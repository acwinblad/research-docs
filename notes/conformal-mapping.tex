\section{Complex Functions and Differentiation}
A brief recap on complex functions and differentiation is written and will be used in the following section.
We start with a complex plane $z$ which is made up of the $x$-axis and $y$-axis
\begin{equation}
  z = x+iy.
\end{equation}

Given a complex function $f(z)$ becomes
\begin{equation}
  f(z) = f(x+iy).
\end{equation}

Which leads us to then write it as
\begin{equation}
  f(z) = u(x,y) +iv(x,y).
\end{equation}

Now for differentiation, a complex function $f(z)$ is differentiable, the proof we skip for now.
It's derivative is 
\begin{align}
  f'(z) &= \dfrac{\partial u}{\partial x} + i \dfrac{\partial v}{\partial x} = \dfrac{\partial f}{\partial x} \\
        &= \dfrac{\partial v}{\partial y} - i \dfrac{\partial u}{\partial y} = -i \dfrac{\partial f}{\partial y}.
\end{align}

Taking note of the Cauchy-Riemann equations
\begin{align}
  \dfrac{\partial u}{\partial x} &= \dfrac{\partial v}{\partial y} \\
  \dfrac{\partial u}{\partial y} &= - \dfrac{\partial v}{\partial x}.
\end{align}

This allows us to conveniently write the derivative as
\begin{align}
  \dfrac{\partial f}{\partial x} &= \dfrac{\partial u}{\partial x} - i \dfrac{\partial u}{\partial y} \\
  &= \dfrac{\partial v}{\partial y} + i \dfrac{\partial v}{\partial x},
\end{align}

Which will be used in the following section.

\section{Conformal Mapping of our Tight-Binding Hamiltonian}

Complex functions allow us to use harmonic functions to solve the two-dimensional Laplace equation.
Conformal mapping allows us to map from one plane to another.
In general we are mapping a complex $z$-plane to a complex $w$-plane to solve some sort of equation.
Once solved in the $w$-plane we can write the solution back in terms of the $z$-plane.
For Laplace's equation it is written as
\begin{equation}
  \nabla^2 \psi = |f'(z)|^2\nabla^2\Psi.
\end{equation}

This factor, or simply the Jacobian, is needed to be accounted for mapping our Hamiltonian.
Second order partial derivatives are given in the text without proof.
We have first order partial derivatives that need to be accounted for.
Recall, we have the following first order partial derivatives
\begin{align}
  \pm \psi_x + i \psi_y. 
\end{align}

Using the chain rule we arrive at
\begin{align}
  \pm \psi_x + i \psi_y &= \pm \psi_u u_x \pm \psi_v v_x + i \psi_u u_y + i \psi_v v_y \\
  &= (\pm u_x + i u_y) \psi_u + i(\mp i v_x + v_y) \psi_y \\
  &=
  \begin{cases}
    (u_x + i u_y) \psi_u + i (v_y - i v_x) \psi_v & (+) \\
    -(u_x - i u_y) \psi_u + i (v_y + i v_x) \psi_v & (-)
  \end{cases} \\
  &=
  \begin{cases}
    \bar{f}'(z) (\psi_u + i \psi_v) & (+) \\
    f'(z) (-\psi_u + i \psi_v) & (-).
  \end{cases}
\end{align}

Now that we have the off-diagonal terms we can write our full Hamiltonian,
\begin{align}
  \Ham = 
  \begin{bmatrix}
    -|f'(z)|^2\dfrac{\nabla^2}{2m} - \mu & \Delta f'(z) (-\partial_u + i \partial_v) \\
    \Delta \bar{f}'(z)(\partial_u +i\partial_v) & |f'(z)|^2\dfrac{\nabla^2}{2m} + \mu
  \end{bmatrix}
\end{align}

This Hamiltonian looks almost identical to the previous version but with new factor terms on each element.
Here is where the situation is a bit trickier than the Laplace equation.
Remember it is
\begin{equation}
  \nabla^2 \psi = |f'(z)|^2\nabla^2\Psi.
\end{equation}

One can simple solve the right hand size and invert back to the $z$-plane without worrying about a Jacobian factor.

\subsection{Choosing an Analytic Function to Map to}

One choice of mapping is using $w = \ln(z)$.
In the case of a semi-inifite plane lying from $x<0$, we have a straight line from $-\infty<y<\infty$, or a wedge ranging from $-\pi/2<\phi<\pi/2$.
Under this mapping the $w$-plane geometry has a semi-infinite ribbon ranging from $-\infty<u<\infty$ and $-\pi/2<v<\pi/2$, where $u = \ln(x^2+y^2)^{1/2}$ and $v = \arctan(y/x)$.
The Jacobian can be found from $|f'(z)|^2$ by letting
\begin{align*}
  f'(z) &= \dfrac{1}{z} = \dfrac{z^*}{|z|^2} = \dfrac{e^{-i\phi}}{r} = e^{-(u+iv)} \\
  J &= |f'(z)|^2 = \dfrac{1}{r^2} = e^{-2u}
\end{align*}

We now arrive at a Hamiltonian in the $w$-plane that is
\begin{align}
  \Ham = 
  \begin{bmatrix}
    -e^{-2u}\dfrac{\nabla^2}{2m} - \mu & \Delta e^{-(u+iv)} (-\partial_u + i \partial_v) \\
    \Delta e^{-(u-iv)}(\partial_u +i\partial_v) & e^{-2u}\dfrac{\nabla^2}{2m} + \mu
  \end{bmatrix}
\end{align}

As a sanity check we take the solution from the $z$-plane and transform it to the $w$-plane to see if we can get a zero energy solution out. 
In the $z$-plane the zero energy edge state with $k_y = 0$ has the following form
\begin{equation}
  \Psi(x) = \mathcal N (e^{-\lambda_1 x}-e^{-\lambda_2 x}) \phi_{x,+}
\end{equation}

Whereas, in the $w$-plane it is
\begin{equation}
  \Psi(u,v) = \mathcal N' (e^{-\lambda_1 e^u \cos{v}}-e^{-\lambda_2 e^u \cos{v}}) \phi
\end{equation}

Applying the transformed Hamiltonian onto the new eigen state we should get zero.
Skipping some derivative steps we arrive at 
\begin{align*}
  H_{top row}\Psi(u,v) &= -e^{-2u} \left(\dfrac{\lambda_j^2}{2m} e^{2u} e^{-\lambda_j e^u \cos{v}}\right) -\mu e^{-\lambda_j e^u \cos{v}} \\
  &- \Delta e^{-(u+iv)} (\lambda_j\cos{v} + i \lambda_j \sin{v}) e^u e^{-\lambda_j e^u \cos{v}} \\
  H_{top row}\Psi(u,v) &= -\left(\dfrac{\lambda_j^2}{2m} +\mu - \Delta \lambda_j \right) e^{-\lambda_j e^u \cos{v}}
\end{align*}

In the $z$-plane it is easy to show
\begin{equation}
  \lambda = m\Delta \pm \sqrt{m^2\Delta^2 - 2m\mu}
\end{equation}

\section{Derivation of a Wedge Corner State using Conformal Mapping}

As noted before we used $w = \ln{z}$ on a semi-infinite plane lying in the negative $x$-plane.
We will use the same mapping on a wedge geometry.
We define the wedge so it is symmetric about the $x$-axis and opening in the positive direction.
Thus, we define $\Psi(r,\phi_0/2 \leq \phi \leq -\phi_0/2) = 0$.
We will also let the function go to zero as $r$ goes to infinity, $\Psi(r\to\infty,\phi)=0$.
For our specific problem $\phi_0 = \pi / 3$, however, we will solve for a general case.
The Hamiltonian for a wedge in the $z$-plane is
\begin{align}
  \Ham = 
  \begin{bmatrix}
    -\dfrac{\nabla^2}{2m} - \mu & \Delta (-\partial_u + i \partial_v) \\
    \Delta (\partial_u +i\partial_v) & \dfrac{\nabla^2}{2m} + \mu
  \end{bmatrix}
\end{align}

Again, using the previous mapping the transformed Hamiltonian is
\begin{align}
  \Ham = 
  \begin{bmatrix}
    -e^{-2u}\dfrac{\nabla^2}{2m} - \mu & \Delta e^{-(u+iv)} (-\partial_u + i \partial_v) \\
    \Delta e^{-(u-iv)}(\partial_u +i\partial_v) & e^{-2u}\dfrac{\nabla^2}{2m} + \mu
  \end{bmatrix}
\end{align}

Now, in the $w$-plane we need to solve for a general solution.
A good guess at an ansatz would to use
\begin{align*}
  \psi(x,y) = \exp{a x + i b y} \\
  \psi(u,v) = \exp(\lambda e^u \cos{v} + n e^u \sin{v} )
\end{align*}

Applying our Hamiltonian to the trial wave function we find a relation between the two quantum numbers given as
\begin{align*}
  \lambda^2 + n^2 & = \left(m\Delta \pm \sqrt{m^2\Delta^2 - 2m\mu}\right)^2 = \beta_{\pm}^2 \\
  \lambda &= \pm \left(\beta_{\pm}^2 - n^2 \right)^{1/2}
\end{align*}


