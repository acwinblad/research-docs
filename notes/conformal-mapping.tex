\section{Complex Functions and Differentiation}
A brief recap on complex functions and differentiation is written and will be used in the following section.
We start with a complex plane $z$ which is made up of the $x$-axis and $y$-axis
\begin{equation}
  z = x+iy.
\end{equation}

Given a complex function $f(z)$ becomes
\begin{equation}
  f(z) = f(x+iy).
\end{equation}

Which leads us to then write it as
\begin{equation}
  f(z) = u(x,y) +iv(x,y).
\end{equation}

Now for differentiation, a complex function $f(z)$ is differentiable, the proof we skip for now.
It's derivative is 
\begin{align}
  f'(z) &= \dfrac{\partial u}{\partial x} + i \dfrac{\partial v}{\partial x} = \dfrac{\partial f}{\partial x} \\
        &= \dfrac{\partial v}{\partial y} - i \dfrac{\partial u}{\partial y} = -i \dfrac{\partial f}{\partial y}.
\end{align}

Taking note of the Cauchy-Riemann equations
\begin{align}
  \dfrac{\partial u}{\partial x} &= \dfrac{\partial v}{\partial y} \\
  \dfrac{\partial u}{\partial y} &= - \dfrac{\partial v}{\partial x}.
\end{align}

This allows us to conveniently write the derivative as
\begin{align}
  \dfrac{\partial f}{\partial x} &= \dfrac{\partial u}{\partial x} - i \dfrac{\partial u}{\partial y} \\
  &= \dfrac{\partial v}{\partial y} + i \dfrac{\partial v}{\partial x},
\end{align}

Which will be used in the following section.

\section{Conformal Mapping of our Tight-Binding Hamiltonian}

Complex functions allow us to use harmonic functions to solve the two-dimensional Laplace equation.
Conformal mapping allows us to map from one plane to another.
In general we are mapping a complex $z$-plane to a complex $w$-plane to solve some sort of equation.
Once solved in the $w$-plane we can write the solution back in terms of the $z$-plane.
For Laplace's equation it is written as
\begin{equation}
  \nabla^2 \psi = |f'(z)|^2\nabla^2\Psi.
\end{equation}

This factor comes from the chain rule and will be a needed to be accounted for mapping our Hamiltonian.
Second order partial derivative is given in text without proof.
We have first order partial derivatives that need to be accounted for.
Recall, we have the following first order partial derivatives
\begin{align}
  \pm \psi_x + i \psi_y. 
\end{align}

Using the chain rule we arrive at
\begin{align}
  \pm \psi_x + i \psi_y &= \pm \psi_u u_x \pm \psi_v v_x + i \psi_u u_y + i \psi_v v_y \\
  &= (\pm u_x + i u_y) \psi_u + i(\mp i v_x + v_y) \psi_y \\
  &=
  \begin{cases}
    (u_x + i u_y) \psi_u + i (v_y - i v_x) \psi_v & (+) \\
    -(u_x - i u_y) \psi_u + i (v_y + i v_x) \psi_v & (-)
  \end{cases} \\
  &=
  \begin{cases}
    \bar{f}'(z) (\psi_u + i \psi_v) & (+) \\
    f'(z) (-\psi_u + i \psi_v) & (-).
  \end{cases}
\end{align}

Now that we have the off-diagonal terms we can write our full Hamiltonian,
\begin{align}
  \Ham = 
  \begin{bmatrix}
    -|f'(z)|^2\dfrac{\nabla^2}{2m} - \mu & \Delta f'(z) (-\partial_u + i \partial_v) \\
    \Delta \bar{f}'(z)(\partial_u +i\partial_v) & |f'(z)|^2\dfrac{\nabla^2}{2m} + \mu
  \end{bmatrix}
\end{align}

This Hamiltonian looks almost identical to the previous version but with new factor terms on each element.
Here is where the situation is a bit trickier than the Laplace equation.
Remember it is
\begin{equation}
  \nabla^2 \psi = |f'(z)|^2\nabla^2\Psi.
\end{equation}

One can simple solve the right hand size and invert back to the $z$-plane without worrying about the magnitude squared of the complex functions derivative.

\subsection{which map}

