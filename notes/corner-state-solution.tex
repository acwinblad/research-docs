\chapter{Solution to the corner state of the \textit{p}-wave superconducting triangular island}
\section{Conserved quantites within the system}

Let's begin by writing out the Hamiltonian without an in-plane magnetic field and determine what operator shares a common eigenbasis with the Hamiltonian.
\begin{equation}
  \Ham = \left(\dfrac{k^2}{2m} - \mu\right) \sz + \Delta (\bm{k} \times \bm{\sigma}) \cdot \hat{z}
\end{equation}
Where the effective order parameter $\Delta = \alpha\Delta_s/(2V_z)$, effective chemical potential $\mu = V_z-\mu_0$, and $m$ is the effective mass. 
Based on the geometry of the problem we know we have rotational symmetry about the z-axis. 
We also have Rashba spin-orbit coupling.
This gives two good guesses as to which quantities are conserved in the system.
Orbital angular momentum and spin along the z-axis. 
More conventually we will use the total angular momentum operator $J_z = L_z + S_z$.
What's left is to prove that our $J_z$ operator commutes with the Hamiltonian.
For convenience I will include $\hbar$ and $\bm{p}$ where needed.

\begin{align*}
  [J_z,\Ham] &= [L_z,\Ham] + [S_z,\Ham] \\
\end{align*}
\begin{align*}
  [L_z,\Ham] &= (\bm{r}\times\bm{p})\cdot\hat{z}\left[\left(\dfrac{p^2}{2m} - \mu\right) \sz + \Delta (\bm{p} \times \bm{\sigma}) \cdot \hat{z}\right] \\ 
  &-\left[\left(\dfrac{p^2}{2m} - \mu\right) \sz + \Delta (\bm{p} \times \bm{\sigma}) \cdot \hat{z}\right](\bm{r}\times\bm{p})\cdot\hat{z} \\
  &= (r_xp_y-r_yp_x)\left[\left(\dfrac{p^2}{2m} - \mu\right) \sz + \Delta (\bm{p} \times \bm{\sigma}) \cdot \hat{z}\right] \\ 
  &-\left[\left(\dfrac{p^2}{2m} - \mu\right) \sz + \Delta (\bm{p} \times \bm{\sigma}) \cdot \hat{z}\right](r_xp_y-r_yp_x) \\
\end{align*}
\begin{align*}
  &= (r_xp_y-r_yp_x)\left[\Delta (\bm{p} \times \bm{\sigma}) \cdot \hat{z}\right] \\ 
  &-\left[\Delta (\bm{p} \times \bm{\sigma}) \cdot \hat{z}\right](r_xp_y-r_yp_x) \\
  &= (r_xp_y-r_yp_x)\left[\Delta (p_x\sy-p_y\sx)\right] \\ 
  &-\left[\Delta (p_x\sy-p_y\sx)\right](r_xp_y-r_yp_x) \\
  &= i\hbar\Delta (p_x\sx + p_y\sy) = i\hbar\Delta\bm{p}\cdot\bm{\sigma}
\end{align*}

\begin{align*}
  [S_z,\Ham] &= \dfrac{\hbar}{2}(\sz\Ham - \Ham\sz) \\
  &= \dfrac{\hbar\Delta}{2}(\sz(p_x\sy - p_y\sx) - (p_x\sy - p_y\sx)\sz) \\
  &= \dfrac{\hbar{\Delta}{2}(-2ip_x\sx - 2ip_y\sy) = -i\hbar\Delta\bm{p}\cdot\bm{\sigma} \\
  [J_z,\Ham] &= 0.
\end{align*}

Now that we have shown $J_z$ is a conserved quantity we will use it to find a common eigenvector between it and the Hamiltonian.
Recall in spherical coordinates that 
\begin{equation}
  J_z\ket{\Psi} = (-i\hbar\partial_{\phi} + \dfrac{\hbar}{2}\sz)\ket{\Psi} =  \hbar m_j\ket{\Psi}.
\end{equation}
Next we let $\bra{\bm{r}}\ket{\Psi} = R(r)\Phi(\phi)$.
Which gives us the separable differential equation
\begin{align*}
  \hbar m_j \Phi R &= (-i\hbar\Phi_{\phi}+\dfrac{\hbar}{2}\sz\Phi)R \\
  \Phi_{\phi} &= i(m_j-\sz/2)\Phi \\
  \Phi(\phi) &= e^{im_j\phi} [ A e^{-i\phi/2}, B e^{i\phi/2} ]^T \\
  \Phi(\phi) &= [ A e^{iM\phi}, B e^{i(M+1)\phi} ]^T
\end{align*}

where it is clear $m_j$ is half integer, it more convenient to write it as $M$ which is a whole integer and represent the orbital angular momentum eigenvalue.

\section{Radial dependence}

With the general angular solution found we should be to turn the original Hamiltonian into a separable equation. 
The Hamiltonian in cylindrical polar coordinates acting on $\Psi = R(r)\Phi(\phi)$ looks like
\begin{align*}
  \Ham\Psi &= 
  \begin{bmatrix}
    -\dfrac{\nabla^2}{2m} -\mu & \Delta e^{-i\phi}\left(-\partial_r + \dfrac{i\partial_{\phi}}{r}\right) \\
    \Delta e^{i\phi}\left(\partial_r + \dfrac{i\partial_{\phi}}{r}\right) & \dfrac{\nabla^2}{2m}+\mu
  \end{bmatrix} R
  \begin{bmatrix}
    Ae^{iM\phi} \\
    Be^{i(M+1)\phi}
  \end{bmatrix}  = ER
  \begin{bmatrix}
    Ae^{iM\phi} \\
    Be^{i(M+1)\phi}
  \end{bmatrix}.
\end{align*}

Applying the differential operators, multiplying both sides by $2mr^2$, and assuming $R$ is a two component vector we find

\begin{equation*}
  \small
  \begin{bmatrix}
    -(r^2\partial_{rr} +r\partial_r +(2m\mu r^2 -M^2)) & -2m\Delta r^2\left(\partial_r + \dfrac{(M+1)}{r}\right) \\
    2m\Delta r^2\left(\partial_r - \dfrac{M}{r}\right) & r^2\partial_{rr} +r\partial_r +(2m\mu r^2 -(M+1)^2)
  \end{bmatrix} 
  \begin{bmatrix}
    AR_1 \\
    BR_2 \\
  \end{bmatrix} 
  = 2mE r^2
  \begin{bmatrix}
    AR_1 \\
    BR_2 \\
  \end{bmatrix} 
\end{equation*}

The diagonal terms are Bessel differential equations.
Knowing the differential form and knowing the function should be zero at the origin we choose a set of Bessel functions.
A trial solution of
\begin{equation}
  \Psi(r,\phi) = 
  \begin{bmatrix}
    A J_M(\ke r) e^{iM\phi} \\
    B J_{M+1}(\ke r) e^{i(M+1)\phi}
  \end{bmatrix},
\end{equation}

where $\ke = \sqrt{2m(\mu + \epsilon)}$, is used; though this definition is not necessary it may prove useful later if needed.
For convenience we write out some of the Bessel function identities to be used
\begin{align*}
  J_{\nu}'(kx) &+ \dfrac{\nu}{x}J_{\nu}(kx) = kJ_{\nu-1}(kx) \\
  J_{\nu}'(kx) &- \dfrac{\nu}{x}J_{\nu}(kx) = - kJ_{\nu+1}(kx) \\
  J_{\nu}''(kx) &= (\nu^2 - (kx)^2)\dfrac{J_{\nu}(kx)}{x^2} - \dfrac{J_{\nu}'(kx)}{x}.
\end{align*}

Plugging into the Hamiltonian we have two equations that give
\begin{equation*}
  \begin{bmatrix}
  (\ke^2-2m\mu) & -2m\Delta\ke \\
  -2m\Delta\ke & -(\ke^2-2m\mu)
  \end{bmatrix} 
  \begin{bmatrix}
    A \\
    B  
  \end{bmatrix}
  = 2mE
  \begin{bmatrix}
    A \\
    B  
  \end{bmatrix}.
\end{equation*}

This is now a simple eigenvalue problem now which is readily solved. 
\begin{equation*}
  \begin{vmatrix}
    (\ke^2-2m(\mu+E)) & -2m\Delta\ke \\
    -2m\Delta\ke & -(\ke^2-2m(\mu-E)) 
  \end{vmatrix} 
  =
  0
\end{equation*}

Solving for the energy we arrive at 
\begin{equation}
  E = \pm\dfrac{\sqrt{\ke^4+\ke^2(4m^2\Delta^2-4m\mu)+4m^2\mu^2}}{2m}.
\end{equation}

We can also rearrange to see that the wavenumber is
\begin{equation}
  \ke = \pm_1 \sqrt{2m(\mu-m\Delta^2)\pm_2\sqrt{4m^2(\mu-m\Delta^2)^2 - 4m^2(\mu^2-E^2)}}
\end{equation}

Here we see there are 4 wavenumbers that satisfy the eigenvalue problem. 
Though due to linear dependence and boundary conditions we can disregard the negative wavenumbers. 
Leaving us with
\begin{equation}
  \kpm = \sqrt{2m(\mu-m\Delta^2)\pm\sqrt{4m^2(\mu-m\Delta^2)^2 - 4m^2(\mu^2-E^2)}}
\end{equation}

\section{Questions}

