\chapter{Majorana fermions in a tunable semiconductor device}
\section{Applying an on axis Magnetic Field}

A method for making a Majorana fermion tunable device has been shown in two forms, one by D. Sau (REFERENCE) and J. Alicea (REFERENCE). A zinc-blende semiconductor quantum well grown along the (100) direction is considered. We start with the relevant noninteracting Hamiltonian

\begin{equation}
  \mathcal{H}_0 = \sum\limits_{\vec{k}} c_{\vec{k}}^\dagger \left[\frac{k^2}{2m} - \mu + \alpha ( \sigma^x k_y - \sigma^y k_x) \right] c_{\vec{k}}
\end{equation}
where $m$ is the effective mass, $\mu$ is the chemical potential, $\alpha$ is the Rashba spin-orbit(REFERENCED in Alicea's paper as ref 23) coupling strength, and $\sigma^i$ are the Pauli matrices that act on the spin degrees of freedom in $c_{\vec{k}}$. We have set $\hbar=1$ throughout.

We next introduce a ferromagnetic insulator and a magnetic field. The ferromagnetic insulator has magnetization pointing perpendicular to the 2D semiconductor. While the magnetic field will point parallel to the 2D  semiconductor. We assume this will induce a Zeeman interaction

\begin{equation}
  \mathcal{H}_Z = \sum\limits_{\vec{k}} c_{\vec{k}}^\dagger \left[V_x \sigma^x + V_y \sigma^y + V_z \sigma^z \right] c_{\vec{k}}
\end{equation}
but neglible orbital coupling. If we look at the combined Hamiltonian it becomes obvious there is a constant energy plus the energy eigenvalues of the Pauli matrices terms. We can easily solve the eigenvalue problem of

\begin{equation}
  \begin{bmatrix}
    V_z & V_x + \alpha k_y - i(V_y - \alpha k_x) \\
    V_x + \alpha k_y + i(V_y -\alpha k_x) & -V_z
  \end{bmatrix}
\end{equation}

or in a more compact form using $\beta(\vec{k}) = k_y + V_x/\alpha$ and $\gamma(\vec{k}) = k_x - V_y/\alpha$ and $\eta^2(\vec{k})=\beta^2(\vec{k})+\gamma^2(\vec{k})$ and we produce

\begin{equation}
  \begin{bmatrix}
    V_z    &    \alpha(\beta(\vec{k}) + i\gamma(\vec{k})) \\
    \alpha(\beta(\vec{k}) - i\gamma(\vec{k})) &    -V_z
  \end{bmatrix}
\end{equation}

giving $\epsilon_{\pm}' = \pm \sqrt{V_z^2+\alpha^2\eta^2(\vec{k})}$ with eigenvectors

\begin{align}
  u_+(\vec{k})  =  
  \left( \begin{array}{l}
      A_\uparrow(\vec{k}) \\ 
      -A_\downarrow(\vec{k}) \dfrac{\beta(\vec{k}) - i \gamma(\vec{k})}{\eta(\vec{k})}
  \end{array} \right)
  \\ \\
  u_-(\vec{k})  =  
  \left( \begin{array}{l}
      B_\uparrow(\vec{k}) \dfrac{\beta(\vec{k}) + i \gamma(\vec{k})}{\eta(\vec{k})}  \\ 
      B_\downarrow(\vec{k})
  \end{array} \right)
\end{align}

One can find $A_{\sigma}=A_{\sigma}^*$ and $B_{\sigma}=B_{\sigma}^*$ and the coefficients are

\begin{align}
  A_{\uparrow}(\vec{k}) &= \dfrac{-\alpha\eta(\vec{k})}{\sqrt{2\epsilon_+'(\vec{k})}} \sqrt{\dfrac{1}{\epsilon_+'(\vec{k})-V_z}} \\
A_{\downarrow}(\vec{k}) &= \sqrt{\dfrac{\epsilon_+'(\vec{k})-V_z}{2\epsilon_+'(\vec{k})}} \\
B_{\uparrow}(\vec{k}) &= \sqrt{\dfrac{\epsilon_-'(\vec{k})+V_z}{2\epsilon_-'(\vec{k})}} \\ 
B_{\downarrow}(\vec{k}) &= \dfrac{\alpha\eta(\vec{k})}{\sqrt{2\epsilon_-'(\vec{k})}} \sqrt{\dfrac{1}{\epsilon_-'(\vec{k})+V_z}}
\end{align}

If in the case of $V_x=V_y=0$ we arrive back at solution Sau $\mathit{et\ al.}$ calculate for eigenvalues, vectors, and coefficients. The expressions for $A_{\uparrow,\downarrow}$ and $B_{\uparrow,\downarrow}$ can be written in convenient terms as

\begin{align}
  f_p(\vec{k}) &= A_{\uparrow}(\vec{k})A_{\downarrow}(-\vec{k}) = B_{\uparrow}(-\vec{k})B_{\downarrow}(\vec{k}) \\
  &= \dfrac{-\alpha \eta(\vec{k})}{2\sqrt{\epsilon_+'(\vec{k})\epsilon_+'(-\vec{k})}} \sqrt{\dfrac{\epsilon_+'(-\vec{k})-V_z}{\epsilon_+'(\vec{k})-V_z}}
\end{align}

When putting the semiconductor in contact with an $s$-wave superconductor a pairing term is generated by the proximity effect. The full Hamiltonian becomes $\mathcal{H} = \mathcal{H}_0 + \mathcal{H}_Z + \mathcal{H}_{SC}$ with

\begin{align}
  \mathcal{H}_{SC} = \sum\limits_{\vec{k}} \Delta c_{\uparrow,\vec{k}}^\dagger c_{\downarrow,-\vec{k}}^\dagger + H.c.
\end{align}

We now want to write the pairing potential in terms of $c_{\pm}$ using a basis transformation.

\begin{align}
  c_{\uparrow,\vec{k}} &= \bra{\uparrow}\ket{u_+(\vec{k})} c_{\vec{k},+} + \bra{\uparrow}\ket{u_-(\vec{k})} c_{\vec{k},-} \\
  &= A_{\uparrow}(\vec{k}) c_{\vec{k},+} + B_{\uparrow}(\vec{k})\dfrac{\beta(\vec{k})+i\gamma(\vec{k})}{\eta(\vec{k})} c_{\vec{k},-} \\
  c_{\downarrow,-\vec{k}} &= \bra{\downarrow}\ket{u_+(-\vec{k})} c_{-\vec{k},+} + \bra{\downarrow}\ket{u_-(-\vec{k})} c_{-\vec{k},-} \\
  &= -A_{\downarrow}(-\vec{k})\dfrac{\beta(-\vec{k})-i\gamma(-\vec{k})}{\eta(-\vec{k})} c_{-\vec{k},+} + B_{\downarrow}(-\vec{k}) c_{-\vec{k},-}
\end{align}

with the adjoints being
\begin{align}
  c_{\uparrow,\vec{k}}^{\dagger} &= A_{\uparrow}(\vec{k}) c_{\vec{k},+}^{\dagger} + B_{\uparrow}(\vec{k})\dfrac{\beta(\vec{k})-i\gamma(\vec{k})}{\eta(\vec{k})} c_{\vec{k},-}^{\dagger} \\
  c_{\downarrow,-\vec{k}}^{\dagger} &= -A_{\downarrow}(-\vec{k})\dfrac{\beta(-\vec{k})+i\gamma(-\vec{k})}{\eta(-\vec{k})} c_{-\vec{k},+}^{\dagger} + B_{\downarrow}(-\vec{k}) c_{-\vec{k},-}^{\dagger}
\end{align}

Continue reducing the pairing potential which becomes

\begin{equation}
  \begin{split}
    \Delta c_{\uparrow,\vec{k}}^{\dagger}c_{\downarrow,-\vec{k}}^{\dagger}  = \Delta &[ -A_{\uparrow}(\vec{k})A_{\downarrow}(-\vec{k})\dfrac{\beta(-\vec{k})+i\gamma(-\vec{k})}{\eta(-\vec{k})}c_{\vec{k},+}^{\dagger}c_{-\vec{k},+}^{\dagger} \\
    + & B_{\uparrow}(\vec{k})B_{\downarrow}(-\vec{k})\dfrac{\beta(\vec{k})-i\gamma(\vec{k})}{\eta(\vec{k})}c_{\vec{k},-}^{\dagger}c_{-\vec{k},-}^{\dagger}  \\
  +&\left(A_{\uparrow}(\vec{k})B_{\downarrow}(-\vec{k})-B_{\uparrow}(\vec{k})A_{\downarrow}(-\vec{k})\dfrac{\beta(\vec{k})-i\gamma(\vec{k})}{\eta(\vec{k})}\dfrac{\beta(-\vec{k})+i\gamma(-\vec{k})}{\eta(-\vec{k})} \right) c_{\vec{k},+}^{\dagger}c_{-\vec{k},-}^{\dagger} ]
  \end{split}
\end{equation}

We will use a more convenient notation by making the following substitutions

\begin{align}
  \Delta_{++}(\vec{k}) &= -\Delta f_p(\vec{k}) \dfrac{\beta(-\vec{k}) +i\gamma(-\vec{k})}{\eta(-\vec{k})} \\
  \Delta_{--}(\vec{k}) &= \Delta f_p(-\vec{k}) \dfrac{\beta(\vec{k}) -i\gamma(\vec{k})}{\eta(\vec{k})} \\
  \Delta_{+-}(\vec{k}) &= \Delta f_s(\vec{k})
\end{align}

Where 

\begin{align}
  f_s(\vec{k}) = \left(A_{\uparrow}(\vec{k})B_{\downarrow}(-\vec{k})-B_{\uparrow}(\vec{k})A_{\downarrow}(-\vec{k})\dfrac{\beta(\vec{k})-i\gamma(\vec{k})}{\eta(\vec{k})}\dfrac{\beta(-\vec{k})+i\gamma(-\vec{k})}{\eta(-\vec{k})} \right)
\end{align}

The pairing potential Hamiltonian then becomes

\begin{align}
  \mathcal{H}_{SC} = \sum\limits_{\vec{k}} \Delta_{++}c_{\vec{k},+}^{\dagger}c_{-\vec{k},+}^{\dagger} + \Delta_{--}c_{\vec{k},-}^{\dagger}c_{-\vec{k},-}^{\dagger} +\Delta_{+-}c_{\vec{k},+}^{\dagger}c_{-\vec{k},-}^{\dagger} + H.c.
\end{align}

Writing the full Hamiltonian in compact form we will use the following Nambu spinor

\begin{align}
  \Psi = (c_{\vec{k},+},\ c_{\vec{k},-},\ c_{-\vec{k},+}^{\dagger},\ c_{-\vec{k},-}^{\dagger} )^T
\end{align}

Then we write the Hamiltonian as, where we have used the conventional BdG approach of applying the anticommutation relation and reindexing the momentum vetor of the second term to give

\begin{align}
  \mathcal{H} = \dfrac{1}{2}\sum\limits_{\vec{k}} \Psi^{\dagger}H_{BdG}\Psi
\end{align}

with 

\begin{equation}
  H_{BdG} = 
  \begin{bmatrix}
    \epsilon_+(\vec{k}) & 0 & 2\Delta_{++}(\vec{k}) & \Delta_{+-}(\vec{k}) \\
    0 & \epsilon_-(\vec{k}) & -\Delta_{+-}(-\vec{k}) & 2\Delta_{--}(\vec{k}) \\
    2\Delta_{++}^*(\vec{k}) & -\Delta_{+-}^*(-\vec{k}) & -\epsilon_+(-\vec{k}) & 0 \\
    \Delta_{+-}^*(\vec{k}) & 2\Delta_{--}^*(\vec{k}) & 0 & -\epsilon_-(-\vec{k}) \\
  \end{bmatrix}
\end{equation}

where 

\begin{equation}
  \epsilon_{\pm}(\vec{k}) = \dfrac{k^2}{2m} - \mu + \epsilon_{\pm}'(\vec{k})
\end{equation}

We can rearrange our matrix into a more block diagonal form with off terms to give

\begin{equation}
  H_{BdG} = 
  \begin{bmatrix}
    \epsilon_+(\vec{k}) & 2\Delta_{++} & 0 & \Delta_{+-}(\vec{k}) \\
    2\Delta_{++}^* & -\epsilon_+(-\vec{k}) & -\Delta_{+-}^*(-\vec{k}) & 0 \\
    0 & -\Delta_{+-}(-\vec{k}) & \epsilon_-(\vec{k}) & 2\Delta_{--} \\
    \Delta_{+-}^*(\vec{k}) & 0 & 2\Delta_{--}^* & -\epsilon_-(-\vec{k}) \\
  \end{bmatrix}
\end{equation}

Upon studying $V_y=V_x=0$ we see that near the fermi surface the interband pairing has little affect on the band gap. Scaling it's effect from $0 \to 1$ we see the intraband gap appears at a slightly smaller momentum as the interband pairing is turned off. We thus use the approximation $\Delta_{+-}(k_f) \approx 0$. We also set $\mu$ such that is only crosses the lower bands, thus allowing $c_+^{\dagger} \to 0$.

\begin{equation}
  H_{BdG} = 
  \begin{bmatrix}
    \epsilon_-(\vec{k}) & 2\Delta_{--}(\vec{k}) \\
    2\Delta_{--}^*(\vec{k}) & -\epsilon_-(-\vec{k}) \\
  \end{bmatrix}
\end{equation}

Solving for the dispersion relation of the system we arrive at

\begin{align}
  E_{\pm}(\vec{k}) = \dfrac{\epsilon_-'(\vec{k})-\epsilon_-'(-\vec{k})}{2} \pm \sqrt{\dfrac{(\epsilon_-(\vec{k})+\epsilon_-(-\vec{k}))^2}{4}+4\abs{\Delta_{--}(\vec{k})}^2}
\end{align}

\section{Small Applied Magnetic Field Approximation}

To simplify we set $V_y\neq V_x =0$ and look at $V_y\ll V_z$ and $\alpha k_f \ll V_z$ to get an idea of what the effective pairing term will be.

\begin{align}
  &\epsilon_+'(\pm\vec{k}) = V_z\sqrt{1+\dfrac{V_y^2+\alpha^2k^2\mp2\alpha k_xV_y}{V_z^2}} \\
  &\epsilon_+'(\pm\vec{k}) \approx V_z\left(1+\dfrac{V_y^2+\alpha^2k^2\mp2\alpha k_xV_y}{2V_z^2}\right) \\
  &\epsilon_+'(\pm\vec{k}) -V_z \approx \dfrac{V_y^2+\alpha^2k^2\mp2\alpha k_xV_y}{2V_z^2}
\end{align}

\begin{align}
  &\dfrac{\sqrt{\epsilon_+'(\vec{k}) -V_z}}{\eta(\vec{k})} \approx \sqrt{\dfrac{V_y^2+\alpha^2k^2-2\alpha k_xV_y}{2V_z^2}} \dfrac{\alpha}{\sqrt{V_y^2+\alpha^2k^2-2\alpha k_xV_y}} \\
  &\dfrac{\sqrt{\epsilon_+'(\vec{k}) -V_z}}{\eta(\vec{k})} \approx \dfrac{\alpha}{\sqrt{2}V_z} \\
  &\dfrac{\eta(-\vec{k})}{\sqrt{\epsilon_+'(-\vec{k}) -V_z}} \approx \sqrt{\dfrac{2V_z^2}{V_y^2+\alpha^2k^2+2\alpha k_xV_y}} \dfrac{\sqrt{V_y^2+\alpha^2k^2+2\alpha k_xV_y}}{\alpha} \\
  &\dfrac{\eta(-\vec{k})}{\sqrt{\epsilon_+'(-\vec{k}) -V_z}} \approx \dfrac{\sqrt{2}V_z}{\alpha} \\
  &\dfrac{\eta(-\vec{k})}{\sqrt{\epsilon_+'(-\vec{k}) -V_z}} \dfrac{\sqrt{\epsilon_+'(\vec{k}) -V_z}}{\eta(\vec{k})} \approx 1
\end{align}

\begin{align}
  &(\epsilon_+'(-\vec{k})\epsilon_+'(\vec{k}))^{-1/2} \approx V_z^{-1}(1-\delta_-)^{-1/2}(1-\delta_+)^{-1/2} \\
  &(\epsilon_+'(-\vec{k})\epsilon_+'(\vec{k}))^{-1/2} \approx V_z^{-1}(1-\frac{1}{2}(\delta_-+\delta_+)) \\
  &(\epsilon_+'(-\vec{k})\epsilon_+'(\vec{k}))^{-1/2} \approx V_z^{-1}\left(1-\dfrac{V_y^2+\alpha^2k^2}{2V_z^2}\right) \\
\end{align}

With all of the appropriate approximations we can now write out the intraband pairing term as

\begin{align}
  \Delta_{--}(\vec{k}) &\approx -\dfrac{\Delta}{2V_z}\left(1-\dfrac{V_y^2+\alpha^2k^2}{2V_z^2}\right)(\alpha k_y -i(\alpha k_x - V_y) \\
  \Delta_{--}(\vec{k}) &\approx -\dfrac{\Delta}{2V_z}(\alpha k_y - i(\alpha k_x - V_y))
\end{align}

If we instead turn the applied field from $y$ to $x$ we arrive at a similar answer as above. Combining both solutions for any arbitrary magnetic field pointing in the $x$-$y$ plane we arrive at

\begin{align}
  \Delta_{--}(\vec{k}) \approx -\dfrac{\Delta}{2V_z}((\alpha k_y + V\cos{\phi})-i(\alpha k_x - V\sin{\phi}))
\end{align}

Where $V=\sqrt{V_x^2+V_y^2}$ and $\phi=\arg(V_x+iV_y)$

\section{Perturbation of Band Gap due to Magnetic Field}

Let us now consider what has more of an affect on the energy band gap, the diagonal or off-diagonal terms in the Hamiltonian. To start we seperate the Hamiltonian in to two terms using only an applied field in the $y$-direction,

\begin{align}
  H_{BdG} = H_0 + H_y 
\end{align}

Where 

\begin{align}
  H_0 &= 
  \begin{bmatrix}
    \epsilon_0(k) & 2\Delta_0(\vec{k}) \\
    2\Delta_0^*(\vec{k}) & -\epsilon_0(k) \\
  \end{bmatrix} \\
  H_y &= 
  \begin{bmatrix}
    \epsilon_y(k_x) & 2\Delta_y \\
    2\Delta_y^* & -\epsilon_y(-k_x) \\
  \end{bmatrix}
\end{align}

Here we define

\begin{align}
  \epsilon_0(k) &= \dfrac{k^2}{2m}-\mu-V_z-\dfrac{\alpha^2k^2}{2V_z} \\
  \Delta_0(\vec{k}) &= -\dfrac{\alpha\Delta}{2V_z}(k_y-ik_x) \\
  \epsilon_y(\pm k_x) &= \dfrac{V_y}{V_z}(\pm\alpha k_x -\frac{1}{2}V_y) \\
  \Delta_y &= -\dfrac{i\Delta V_y}{2V_z}
\end{align}

To start we look at when $\epsilon_0(k_0) = 0$, which is the momentum value we would see an energy band gap appear. Determining the orthonormal eigensystem of the base Hamiltonian gives us 

\begin{align}
  \ket{\pm =\pm 2|\Delta_0|} = \dfrac{1}{\sqrt{2}}
    \begin{bmatrix}
      \mp\dfrac{\Delta_0}{|\Delta_0|} \\
      1
    \end{bmatrix}
\end{align}

We then perform the basis transformation 

\begin{align}
  \matrixelement{+}{H_y}{+} &= \dfrac{\alpha k_x V_y}{V_z} \\
  \matrixelement{+}{H_y}{-} &= \dfrac{V_y}{V_z}(\frac{1}{2}V_y+i\Delta) \\
  \matrixelement{-}{H_y}{+} &= \dfrac{V_y}{V_z}(\frac{1}{2}V_y-i\Delta) \\
  \matrixelement{-}{H_y}{-} &= \dfrac{\alpha k_x V_y}{V_z}
\end{align}

Which can be written in a more compact form as

\begin{align}
  H_y &= \dfrac{V_y}{V_z} 
  \begin{bmatrix}
    \alpha k_x & \frac{1}{2}V_y+i\Delta \\
    \frac{1}{2}V_y-i\Delta & \alpha k_x \\
  \end{bmatrix} 
\end{align}

Here we claim that all elements of the matrix are of equal importance due to them all being within the same order of magnitude. 
