In this dissertation we discuss how gauge potentials can be used as a key ingredient for inducing topological phase transitions in condensed matter systems, such as conductors, insulators, and superconductors.
The first chapter covers some important background physics: Maxwell's equations, gauge invariance, minimal coupling, and Peierls phase, etc.
It then presents a review of how one can realize Majorana fermions (MFs) in superconductors and their importance to topological quantum computing.
In the end of chapter 1, we give an overview of the basics of Landau levels (LLs) and their relation to the Chern number.
Chapter 2 presents a theoretical proposal for inducing topological phase transitions that allow for MFs to be hosted and rotated along the corners of a hollow equilateral triangle, which can serve as a basic building block for topological quantum logic gates.
This provides a potential new avenue for achieving a topological quantum computation where a network of interconnected triangular islands allows for braiding of MFs.
Chapter 3 we show using Floquet theory and high-frequency expansion, that oblique incident, circularly polarized light can give rise to spectral features analogous to Landau levels in the quantum Hall effect (QHE), where the effective magnetic field is related to the electric field of the laser light.
Outside of having the electric field as a useful parameter for achieving a QHE device, this finding enables us to explore non-equilibrium systems exhibiting topological phenomena in the absence of spatial periodicity.
Chapter 4 concludes and discusses further implications of the work in this dissertation.
