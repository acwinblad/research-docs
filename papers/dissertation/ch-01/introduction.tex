\section{Introduction}

In this dissertation we discuss how gauge potentials can be used as a key ingredient for inducing topological phase transitions in condensed matter systems, such as conductors, insulators, and superconductors.
We will cover some important background physics: Maxwell's equation, gauge invariance, minimal coupling, and Peierls phase.
A review of how one can achieve Majorana fermions in superconductors is shown and their importance to topological quantum computing.
Followed by some basics of Landau level in relation to the Chern number, a parameter that indicates if a system is in a non-trivial topological phase.
Then, applying these concepts to superconductors and conductors, for 2D electron gases (2DEG) and Dirac systems, we see topological phenomena occur.

In the case of a superconductor we can induce topological phase transitions that allow for Majorana Fermions to be hosted and rotated along the corners of a hollow equilateral triangle, a basic building block for a topological quantum logic gate.
This provides a potential new avenue for achieving a topological quantum computation where a network of interconnected triangular islands allows for braiding of Majorana fermions.

For 2DEG and Dirac systems we show oblique incident circularly polarized light can using Floquet theory can achieve Landau Levels, or quantum Hall effect, where the effective magnetic field is related to the electric field of the laser light.
Outside of having the electric field as a useful parameter for achieving a QHE device this lets us explore non-equilibrium systems which is a burgeoning field of interest in condensed matter physics.
