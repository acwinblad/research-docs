\section{Landau levels and quantum Hall effect}

\subsection{Landau levels in condensed matter systems}
We are interested in using non-uniform circularly polarized laser light to induce QHE in 2DEG and Dirac systems and determine whether the energy levels are Landau level-like (LLL).
In the classical case of charged QHE the charged particles in the system are quantized in cyclotron orbits due to perpendicular magnetic field, these energies are called Landau levels (LLs).
To understand why LLs appear and QHE arises we need to first solve the Hamiltonian associated with a 2DEG and Dirac systems in the presence of a perpendicular magnetic field.
We can start with the square lattice tight-binding Hamiltonian for a 2DEG

\begin{equation}
  \ham = -\sum_{\langle j,l \rangle} t\cc_j c_{l} + h.c.,
\end{equation}
and in momentum space

\begin{equation}
  \ham = -\sum_{\vec{p}} 2t \left( \cos{(p_x a)} + \cos{(p_y a)} \right) \cc_{\vec{p}} c_{\vec{p}}.
\end{equation}
Then in the limit of small momenta $p$ we arrive at

\begin{align}
  \ham &= -\sum_{\vec{p}} 2t \left( 2 - \dfrac{p_x^2a^2}{2} -\dfrac{p_y^2a^2}{2} \right) \cc_{\vec{p}} c_{\vec{p}}, \nonumber \\
  \ham(\vec{p}) &= \dfrac{p_x^2 + p_y^2}{2m},
\end{align}
we have thus arrived at Schrodingers equation for a 2DEG in the limit of small momenta.
Let us assume a 2DEG in the $x$-$y$ plane and has a magnetic field that points in the positive $\hat{z}$ direction, $\vec{B} = B\hat{z}$ or $\vec{A} = Bx\hat{y}$.
The Hamiltonian in momentum space then becomes

\begin{equation}
  \ham = \dfrac{1}{2m} \left( \op{p}_x^2 + (\op{p}_y - qB \op{x})^2 \right)
\end{equation}
Recall $[\op{r}_{\alpha}, \op{p}_{\beta}] = i\hbar \delta_{\alpha,\beta}$, that means our magnetic term commutes with $\op{p_y}$, so let us assume that $\Psi(x,y) = e^{ik_y y} \psi(x)$.
Acting the Hamiltonian on the ansatz wavefunction yields

\begin{align}
  \ham\Psi &= \dfrac{1}{2m} \left( \op{p}_x^2 + (qB \op{x} - \hbar k_y)^2 \right)e^{ik_y y} \psi(x) \nonumber \\
  \ham & = \dfrac{1}{2m} \left( \op{p}_x^2 + q^2 B^2 \op{x}^2 \right)
\end{align}
where we let $x - \tfrac{\hbar k_y}{mqB} \rightarrow x$, since it is just a shift in $x$ coordinates.
Notice that we arrive at the expression for a quantum harmonic oscillator.
A derivation for the energy solutions can be found in \ref{appendix:qho}.
With the energy solutions

\begin{equation}
  E_n = \dfrac{\hbar q B}{m} \left(n+\dfrac{1}{2}\ \right) = \hbar \omega \left(n + \dfrac{1}{2} \right)
\end{equation}

An alteration to the lattice model can have slightly different results.
Using a honeycomb lattice, provided by graphene, gives the following Hamiltonian

\begin{equation}
  \ham = -t \sum_{\substack{j, l \\ \alpha \beta}} \cc_{j\alpha} c_{l\beta} + h.c.,
\end{equation}
with lattice vectors $\vec{a}_1 = \sqrt{3} a \hat{x}$ and $\vec{a}_2 = \tfrac{\sqrt{3}}{2} a \hat{x} + \tfrac{3}{2} a \hat{y}$.
In momentum space
\[
  \ham = -t \sum_{\vec{p}}
  \begin{bmatrix}
    0 & 1 + e^{i\vec{p}\cdot \vec{a}_1 } + e^{i \vec{p}\cdot \vec{a}_2} \\
    1 + e^{-i\vec{p}\cdot \vec{a}_1 } + e^{-i \vec{p}\cdot \vec{a}_2} & 0
  \end{bmatrix},
\]
\[
  \ham(\vec{p}) =
  \begin{bmatrix}
    0 & t(\vec{p}) \\
    t^*(\vec{p}) & 0
  \end{bmatrix},
\]
where the hopping can be rewritten as

\begin{equation}
  t(\vec{p}) = -t e^{i\sqrt{3} p_x a / 2} \left( 2\cos{\left( \dfrac{\sqrt{3} p_x a }{ 2 } \right)} + e^{i 3 p_y a /2 } \right)
\end{equation}
which gives the following energy spectrum

\begin{equation}
  E(\vec{p}) = \pm t \sqrt{3 + 2\cos{\left(\sqrt{3}p_x a\right)} + 4\cos{\left(\dfrac{\sqrt{3}p_x a}{2}\right)}\cos{\left(\dfrac{3p_y a}{2}\right)} }.
\end{equation}
There are several high symmetry points on the corners of the Brillouin zone, one such point is $\vec{K} = \tfrac{4\pi}{3\sqrt{3}a} \hat{x} $.
Going back to the Hamiltonian and expanding about $\vec{K}$ with small $\vec{q}$, $\vec{q} = \vec{p} + \vec{K}$, gives the following hopping amplitude

\begin{align}
  t(\vec{q}) &= -t e^{i\sqrt{3} q_x a / 2} e^{i\sqrt{3} K a / 2} \left( 2\cos{\left( \dfrac{\sqrt{3} q_x a }{2}  + \dfrac{\sqrt{3} K a}{2} \right)} + e^{i 3 p_y a /2 } \right) \nonumber \\
  %t(\vec{q}) &= -t e^{i\sqrt{3} q_x a / 2} e^{i 2\pi/3} \left( 2\cos{\left( \dfrac{\sqrt{3} q_x a }{2}  + \dfrac{2\pi}{3} \right)} + e^{i 3 p_y a /2 } \right) \nonumber \\
  %t(\vec{q}) &= -t e^{i\sqrt{3} q_x a / 2} e^{i 2\pi/3} \left( -\cos{\left( \dfrac{\sqrt{3} q_x a }{2}\right)} - \sqrt{3}\sin{\left( \dfrac{\sqrt{3}q_x a}{3} \right)} + e^{i 3 p_y a /2 } \right) \nonumber \\
  %t(\vec{q}) &\approx t e^{i 2\pi/3} \left(\dfrac{3q_x a}{2} - \dfrac{i3q_y a}{2} \right) \nonumber \\
\end{align}
and in the small momenta $q$ limit
\begin{align}
  t(\vec{q}) &\approx v_F e^{i 2\pi/3} \left(q_x - iq_y \right), \nonumber \\
  t^*(\vec{q}) &\approx v_F e^{-i 2\pi/3} \left(q_x + iq_y \right), \nonumber
\end{align}
where we keep the leading order in $\vec{q}$ and $v_F = \tfrac{3ta}{2}$.
Using a gauge transformation and redefining $\vec{q} \rightarrow \vec{p}$ we arrive at the Dirac equation

\begin{equation}
  \ham(\vec{p}) = v_F \bm{\sigma}\cdot \vec{p}.
\end{equation}
With graphene spanning the $x$-$y$ plane in the presence of a magnetic field $\vec{B} = B\hat{z}$, $\vec{A} =  Bx\hat{y}$, the Dirac equation becomes

\begin{equation}
  \ham(\vec{p}) = v_F \bm{\sigma}\cdot \left(\vec{p} - q\vec{A} \right).
\end{equation}
A derivation for the energy solution can be found in \ref{appendix:dirac}.
The quantized energy solutions for a 2D Dirac equation in the presence of perpendicular magnetic field is

\begin{equation}
  E_n = v_F \sqrt{2 n \hbar qB }
\end{equation}

We see the energy of both systems produce discrete quantized energies for charged particles in cyclotron orbits with no depenedence on momenta, these are Landau levels.
It is also important to note these Landau levels are highly degenerate flat bands, which will lend to the discussion of bulk insulating states.

\subsection{Quantized Hall conductivity and Chern number}

Here we will go over the relationship between quantized Hall conductivity and Chern number, which is described as

\begin{equation}
  \sigma_{xy} = - C\dfrac{e^2}{h}, \quad C \in \mathbb{Z}.
\end{equation}
Let us consider a 2D system with translation symmetry in the $x$ and $y$ axis with lattice constants $l_x$ and $l_y$, respectively.
The Brillouin zone boundaries are
\begin{equation}
  k_x = \dfrac{\pi}{l_x} [-1,1) \quad \text{and} \quad k_y = \dfrac{\pi}{l_y} [-1,1),
\end{equation}
where we can think of the periodicity in $k_x$ and $k_y$ creating a torus, $\vec{T}$, in 3D space.
We now introduce the Kubo formula, which is a linear response to a physical observable by a time-dependent perturbation, for conductivity as

\begin{equation} \label{eq:kubo}
  \sigma_{xy} = i\hbar \sum_{E_a < E_F < E_b} \int_{\vec{T}} \dfrac{d^2k}{(2\pi)^2} \dfrac{\langle u_{\vec{k}}^a | J_y | u_{\vec{k}}^b \rangle \langle u_{\vec{k}}^b | J_x | u_{\vec{k}}^a \rangle - \langle u_{\vec{k}}^a | J_x | u_{\vec{k}}^b \rangle \langle u_{\vec{k}}^b | J_y | u_{\vec{k}}^a \rangle}{{(E_b - E_a)}^2}.
\end{equation}
The $a$ and $b$ terms represent dispersion bands below and above the Fermi energy, respectively, and a basic requirement the bands be separated to allow for an insulator.
Recall, current density can be defined by $\vec{J} = (e/\hbar) \partial_{\vec{k}} H$.
So long as $H$ is written in a basis that current density is non-zero we can continue.
Plugging current density in Eq. \eqref{eq:kubo} gives

\begin{equation}
  \sigma_{xy} = \dfrac{ie^2}{h} \sum_{E_a < E_F < E_b} \int_{\vec{T}} \dfrac{d^2k}{2\pi} \dfrac{\langle u_{\vec{k}}^a | \partial_{k_y} H | u_{\vec{k}}^b \rangle \langle u_{\vec{k}}^b | \partial_{k_x} H | u_{\vec{k}}^a \rangle - \langle u_{\vec{k}}^a | \partial_{k_x} H | u_{\vec{k}}^b \rangle \langle u_{\vec{k}}^b | \partial_{k_y} H | u_{\vec{k}}^a \rangle}{{(E_b - E_a)}^2}.
\end{equation}
Using the product rule on the following expression $\langle \alpha | \partial_j (H | \beta\rangle)$ and using $\sum_b = \vec{1} - \sum_a |u_{\vec{k}}^a\rangle \langle u_{\vec{k}}^a |$ lets us simplify the previous expression to

\begin{equation}
  \sigma_{xy} = \dfrac{e^2}{h} \sum_{a} \int_{\vec{T}} \dfrac{d^2k}{2\pi} i\left(\langle \partial_{k_y} u_{\vec{k}}^a | \partial_{k_x} u_{\vec{k}}^a \rangle - \langle \partial_{k_x} u_{\vec{k}}^a | \partial_{k_y} u_{\vec{k}}^a \rangle \right) = \dfrac{e^2}{h} \sum_{a} \int_{\vec{T}} \dfrac{d^2k}{2\pi} \mathcal{F}_{xy}
\end{equation}
where we recognize the integral is the negative Chern number integral, which is always integer.
The Hall conductivity becomes

\begin{equation}
  \sigma_{xy} = -\dfrac{e^2}{h} \sum_a C_a = -C \dfrac{e^2}{h}.
\end{equation}
This leads to Hall conductivity becoming quantized and increases as the Fermi energy fills more bands.

\subsection{Laughlin pump}

