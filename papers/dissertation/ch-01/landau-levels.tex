\section{Landau levels in condensed matter}

We are also interested in producing Landau levels in 2DEG and Dirac models using non-uniform circularly polarized laser light.
To understand how Landau levels appear we need to solve the Hamiltonian associated with a 2DEG in the presence of a Magnetic field.
We can start with the square lattice tight-binding Hamiltonian for a 2DEG

\begin{equation}
  \ham = -\sum_{\langle j,l \rangle} t\cc_j c_{l} + h.c.,
\end{equation}
and in momentum space

\begin{equation}
  \ham = -\sum_{\vec{p}} 2t \left( \cos{(p_x a)} + \cos{(p_y a)} \right) \cc_{\vec{p}} c_{\vec{p}}.
\end{equation}
Then in the limit of small momenta $k$ we arrive at

\begin{align}
  \ham &= -\sum_{\vec{p}} 2t \left( 2 - \dfrac{p_x^2a^2}{2} -\dfrac{p_y^2a^2}{2} \right) \cc_{\vec{p}} c_{\vec{p}}, \nonumber \\
  \ham(\vec{p}) &= \left(\dfrac{p_x^2 + p_y^2}{2m}\right),
\end{align}
we have thus arrived at Schrodingers equation for a 2DEG in the limit of small momenta.
Let us assume a 2DEG in the $x$-$y$ plane and has a magnetic field that points in the positive $\hat{z}$ direction, $\vec{B} = B\hat{z}$ or $\vec{A} = Bx\hat{y}$.
The Hamiltonian in momentum space then becomes

\begin{equation}
  \ham = \dfrac{1}{2m} \left( \op{p}_x^2 + (\op{p}_y - qB \op{x})^2 \right)
\end{equation}
Recall $[\op{r}_{\alpha}, \op{p}_{\beta}] = i\hbar \delta_{\alpha,\beta}$, that means our magnetic term commutes with $\op{p_y}$, so let us assume that $\Psi(x,y) = e^{ik_y y} \psi(x)$.
Acting the Hamiltonian on the ansatz wavefunction yields

\begin{align}
  \ham\Psi &= \dfrac{1}{2m} \left( \op{p}_x^2 + (qB \op{x} - \hbar k_y)^2 \right)e^{ik_y y} \psi(x) \nonumber \\
  \ham & = \dfrac{1}{2m} \left( \op{p}_x^2 + q^2 B^2 \op{x}^2 \right) \nonumber \\
  &= \dfrac{\op{p}_x^2}{2m} + \dfrac{1}{2} m \omega^2 \op{x}^2,
\end{align}
where we let $x - \tfrac{\hbar k_y}{mqB} \rightarrow x$, since it is just a shift in $x$ coordinates.
Notice that we arrive at the expression for a quantum harmonic oscillator.
A derivation for the energy solutions can be found in \ref{appendix:qho}.
With the energy solutions

\begin{equation}
  E_n = \hbar \omega \left(n + \dfrac{1}{2} \right) = \dfrac{\hbar q B}{m} \left(n+\dfrac{1}{2}\ \right).
\end{equation}

An alteration to the lattice model can have slightly different results.
Using a honeycomb lattice, provided by graphene, gives the following Hamiltonian

\begin{equation}
  \ham = -t \sum_{\substack{j, l \\ \alpha \beta}} \cc_{j\alpha} c_{l\beta} + h.c.,
\end{equation}
with lattice vectors $\vec{a}_1 = \sqrt{3} a \hat{x}$ and $\vec{a}_2 = \tfrac{\sqrt{3}}{2} a \hat{x} + \tfrac{3}{2} a \hat{y}$.
In momentum space
\[
  \ham = -t \sum_{\vec{p}}
  \begin{bmatrix}
    0 & 1 + e^{i\vec{p}\cdot \vec{a}_1 } + e^{i \vec{p}\cdot \vec{a}_2} \\
    1 + e^{-i\vec{p}\cdot \vec{a}_1 } + e^{-i \vec{p}\cdot \vec{a}_2} & 0
  \end{bmatrix},
\]
\[
  \ham(\vec{p}) =
  \begin{bmatrix}
    0 & t(\vec{p}) \\
    t^*(\vec{p}) & 0
  \end{bmatrix},
\]
where the hopping can be rewritten as

\begin{equation}
  t(\vec{p}) = -t e^{i\sqrt{3} p_x a / 2} \left( 2\cos{\left( \dfrac{\sqrt{3} p_x a }{ 2 } \right)} + e^{i 3 p_y a /2 } \right)
\end{equation}
which gives the following energy spectrum

\begin{equation}
  E(\vec{p}) = \pm t \sqrt{3 + 2\cos{\left(\sqrt{3}p_x a\right)} + 4\cos{\left(\dfrac{\sqrt{3}p_x a}{2}\right)}\cos{\left(\dfrac{3p_y a}{2}\right)} }.
\end{equation}
There are several high symmetry points on the corners of the Brillouin zone, one such point is $\vec{K} = \tfrac{4\pi}{3\sqrt{3}a} \hat{x} $.
Going back to the Hamiltonian and expanding about $\vec{K}$ with small $\vec{q}$, $\vec{q} = \vec{p} + \vec{K}$, gives the following hopping amplitude

\begin{align}
  t(\vec{q}) &= -t e^{i\sqrt{3} q_x a / 2} e^{i\sqrt{3} K a / 2} \left( 2\cos{\left( \dfrac{\sqrt{3} q_x a }{2}  + \dfrac{\sqrt{3} K a}{2} \right)} + e^{i 3 p_y a /2 } \right) \nonumber \\
  t(\vec{q}) &= -t e^{i\sqrt{3} q_x a / 2} e^{i 2\pi/3} \left( 2\cos{\left( \dfrac{\sqrt{3} q_x a }{2}  + \dfrac{2\pi}{3} \right)} + e^{i 3 p_y a /2 } \right) \nonumber \\
  t(\vec{q}) &= -t e^{i\sqrt{3} q_x a / 2} e^{i 2\pi/3} \left( -\cos{\left( \dfrac{\sqrt{3} q_x a }{2}\right)} - \sqrt{3}\sin{\left( \dfrac{\sqrt{3}q_x a}{3} \right)} + e^{i 3 p_y a /2 } \right) \nonumber \\
  t(\vec{q}) &\approx t e^{i 2\pi/3} \left(\dfrac{3q_x a}{2} - \dfrac{i3q_y a}{2} \right) \nonumber \\
  t(\vec{q}) &= v_f e^{i 2\pi/3} \left(q_x - iq_y \right), \nonumber \\
  t^*(\vec{q}) &= v_f e^{-i 2\pi/3} \left(q_x + iq_y \right), \nonumber
\end{align}
where we keep the leading order in $\vec{q}$ and $v_f = \tfrac{3ta}{2}$.
Using a gauge transformation and redefining $\vec{q} \rightarrow \vec{p}$ we arrive at the Dirac equation

\begin{equation}
  \ham(\vec{p}) = v_f \bm{\sigma}\cdot \vec{p}.
\end{equation}
With graphene spanning the $x$-$y$ plane in the presence of a magnetic field $\vec{B} = B\hat{z}$, $\vec{A} =  Bx\hat{y}$, the Dirac equation becomes

\begin{equation}
  \ham(\vec{p}) = v_f \bm{\sigma}\cdot \left(\vec{p - q\vec{A}} \right).
\end{equation}
A derivation for the energy solution can be found in \ref{appendix:dirac}.
The quantized energy solutions for a 2D Dirac equation in the presence of perpendicular magnetic field is

\begin{equation}
  E_n = v_f \sqrt{2 \hbar qB n}
\end{equation}

