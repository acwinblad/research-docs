\section{Maxwell's equations}
In this dissertation we use gauge potentials as the key ingredient for inducing topological phase transitions in condensed matter systems, such as superconductors and insulators.
A quick review of gauge potentials and gauge invariance will help set the stage for the manipulations we do in later chapters.
Let us start with Maxwell theory, given by the following equations,

\begin{align}
  \nabla \cdot \vec{E} &= \dfrac{1}{\epsilon_0}\rho \label{eq:maxwell1} \\
  \nabla \cdot \vec{B} &= 0 \label{eq:maxwell2} \\
  \nabla \cross \vec{E} &= -\partial_t \vec{B} \label{eq:maxwell3} \\
  \nabla \cross \vec{B} &= \mu_0 \vec{J} + \mu_0 \epsilon_0 \partial_t \vec{E} \label{eq:maxwell4}.
\end{align}
We want to write Maxwell's equations in terms of potentials, $V$ and $\vec{A}$.
The magnetic field in terms of a vector potential is $\vec{B} = \nabla \cross \vec{A}$.
This lets us rewrite Eq.~\eqref{eq:maxwell3} as

\begin{align}
  \nabla \cross \left( \vec{E} + \partial_t \vec{A} \right) = 0 \\
  \vec{E} = -\nabla V - \partial_t \vec{A}.
\end{align}
As a check, if $\vec{A} = \text{const}$, then $\vec{E} = -\nabla V$ as expected.
With a new definition of the electric field we look at Eq.~\eqref{eq:maxwell1} to arrive at

\begin{align}
  \nabla \cdot \vec{E} &= \nabla \cdot \left( -\nabla V - \partial_t \vec{A} \right) \\
  &= -\nabla^2 V - \partial_t \nabla \cdot \vec{A} = \dfrac{1}{\epsilon_0}\rho.
  \label{eq:electric-as-potential}
\end{align}
Now we manipulate Eq.~\eqref{eq:maxwell4} to give

\begin{align}
  \nabla \cross \vec{B} &= \nabla \cross (\nabla \cross \vec{A}) \\
  &= \nabla (\nabla \cdot \vec{A}) - \nabla^2 \vec{A} \\
  &= \mu_0 \vec{J} + \mu_0 \epsilon_0 \left( -\nabla \partial_t V - \partial_t^2 \vec{A} \right).
\end{align}
Which we rearrange to

\begin{equation}
  -\mu_0 \vec{J} = \nabla^2 \vec{A} - \mu_0 \epsilon_0 \partial_t^2 \vec{A} - \nabla \left( \nabla \cdot \vec{A} + \mu_0 \epsilon_0 \partial_t V \right).
  \label{eq:current-as-potential}
\end{equation}

We have thus shown Maxwell's equations in terms of potentials.

\section{Gauge transformations}
We now move on to gauge transformations.
Suppose $\vec{A}' = \vec{A}+\bm{\alpha}$ and $V' = V+\beta$.
Both vector potentials give the same magnetic field,

\begin{align}
  \vec{B} = \nabla \cross \vec{A} = \nabla \cross \vec{A}' &= \nabla \cross (\vec{A} + \bm{\alpha}) \\
  \nabla \cross \bm{\alpha} &= 0 \\
  \bm{\alpha} &= \nabla \lambda.
\end{align}
The two potentials should also give the same electric field,

\begin{align}
  \vec{E} &= -\nabla V - \partial_t \vec{A} \\
  &= -\nabla V' - \partial_t \vec{A}' \nonumber \\
  &= -\nabla V - \nabla \beta - \partial_t \vec{A} - \partial_t \bm{\alpha} \nonumber \\
  \text{then} \quad 0 &= \nabla \beta + \partial_t \bm{\alpha}\\
  &= \nabla \beta + \partial_t \nabla\lambda \nonumber \\
  &= \nabla ( \beta + \partial_t \lambda ) \nonumber \\
  \beta &= -\partial_t \lambda + k(t) \nonumber \\
  \beta &= -\partial_t \tilde{\lambda} \nonumber \\
  \vec{A}' &= \vec{A} + \nabla \lambda \\
  V' &= V - \partial_t \tilde{\lambda}
\end{align}
From the above set of expressions we arrive at the general gauge transformations of potentials.
We make note that a change in $V$ and $\vec{A}$ does not change the electric and magnetic fields, gauge invariant, and are used to adjust the divergence of $\vec{A}$.
This allows to solve the scalar and vector potentials readily depending on the gauge.
We will next go over two most common gauge choices, Coulomb and Lorenz.

\subsection{Coulomb gauge}
The Coulomb gauge is used in the case of magnetostatics.
We assert the following for a Coulomb gauge, $\nabla \cdot \vec{A} = 0$, which makes $\nabla^2 V = -\tfrac{1}{\epsilon_0} \rho$.
Recall Eq.~\eqref{eq:current-as-potential}, it simplifies to

\begin{align}
  \left(\nabla^2 - \mu_0 \epsilon_0 \partial_t^2 \right) \vec{A} = -\mu_0 \vec{J} + \mu_0 \epsilon_0 \nabla \partial_t V \nonumber \\
  \square^2 \vec{A} = -\mu_0 \vec{J} + \mu_0 \epsilon_0 \nabla \partial_t V
\end{align}
where we have used $\square$ as the d'Alembertian.

\subsection{Lorenz gauge}
Here we instead assert $\nabla \cdot \vec{A} = -\mu_0 \epsilon_0 \partial_t V$, which makes $\square^2 \vec{A} = -\mu_0 \epsilon_0 \vec{J}$.
Recall Eq.~\eqref{eq:electric-as-potential}, it simplifies to

\begin{align}
  \nabla^2 V + \partial_t (\nabla \cdot \vec{A}) &= \nabla^2 V - \mu_0 \epsilon_0 \partial_t^2 V \nonumber \\
  \square^2 V &= -\dfrac{1}{\epsilon_0} \rho
\end{align}

We can now clearly see how the two gauges affect our potentials.
The Lorenz gauge has the advantage of treating both scalar and vector potential with the same d'Alembertian operator, thus readily able to solve both scalar and vector potentials.
The Coulomb gauge allows for an easily calculable scalar potential, however, it is murkier solve for the vector potential.
REFERENCE GRIFFITHS introduction to electrodynamics 4th Edition
