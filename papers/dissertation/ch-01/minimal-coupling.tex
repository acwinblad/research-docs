\section{Minimal coupling and canonical momentum}
With gauge potentials and their invariance shown we next review how a gauge potential couples to the momentum operator, also known as minimal coupling.
Minimal coupling comes from the following substitution

\begin{equation}
  -i\hbar\nabla \rightarrow -i\hbar\nabla - q\vec{A},
\end{equation}
which can be derived from the canonical momentum operator when a charged particle is present in a gauge potential.
Rigorously speaking, minimal coupling ignores higher order multipole moments.
This also allows the system to have a local gauge invariance under U(1) transformations, i.e.
$\vec{A} \rightarrow \vec{A} + \nabla\lambda$  ~\cite{altlandCondensedMatterField2023}.

Next, we derive the canonical momentum operator.
Starting with the Lagrangian for charged particle coupled to a gauge potential,

\begin{align}
  \mathcal{L} &= T- U \nonumber \\
  \mathcal{L} &= \dfrac{1}{2} m \dot{r}^2 - qV + q \dot{\vec{r}} \cdot \vec{A}(\vec{r},t),
\end{align}
where $T = \tfrac{1}{2} m \dot{r}^2$ and $U = qV - q \dot{\vec{r}} \cdot \vec{A}(\vec{r},t)$.
Recall
\begin{align}
  \vec{p}_{\text{can}} &= \dfrac{\partial\mathcal{L}}{\partial \dot{\vec{r}}} \nonumber \\
  %\vec{p} &= m \dot{\vec{r}} + q \left( \partial_{\dot{x}} + \partial_{\dot{y}} + \partial_{\dot{z}} \right) \left(\dot{x} A_x + \dot{y} A_y + \dot{z} A_z \right) \nonumber \\
  %\vec{p} &= m \vec{v} + q \vec{A} \nonumber \\
  \vec{p}_{\text{can}} &= \vec{p}_{\text{kin}} + q \vec{A}.
  \label{eq:canonical-momentum}
\end{align}
With the canonical momentum defined, the Hamiltonian becomes

\begin{align}
  \ham &= \vec{p}_{\text{can}} \cdot \dot{\vec{r}} - \mathcal{L} \nonumber \\
  %&= (m \dot{\vec{r}} + q \vec{A}) \cdot \dot{\vec{r}} - \dfrac{1}{2} m \dot{r}^2 + q V - q \dot{\vec{r}}\cdot\ \vec{A} \nonumber \\
  %&= \dfrac{1}{2} m \dot{r}^2 + q V \nonumber \\
  %&= \dfrac{1}{2m}  \vec{p}_{kin}^2 + q V \nonumber \\
  \ham &= \dfrac{1}{2m} (\vec{p}_{\text{can}} - q \vec{A})^2 + q V.
\end{align}
Thus, a charged particle in the presence of a gauge potential has the following minimal coupling,
$i\hbar\nabla - q\vec{A}$.

We will now show the Hamiltonian is gauge invariant.
While one can include a scalar potential in the Hamiltonian in the derivation below, we ignore it to keep things simple.
Supposing
$\ham|\psi \rangle = \epsilon |\psi \rangle$
and
$\vec{A}' = \vec{A}+\nabla\lambda$, the Hamiltonian acting on the wavevector is

\begin{equation}
  \ham |\psi\rangle = \dfrac{1}{2m} (\hat{p} - q \vec{A}' + q \nabla \lambda)^2 |\psi\rangle = \epsilon |\psi\rangle.
\end{equation}
We assert

\begin{align}
  %\ham' |\psi'\rangle &= \dfrac{1}{2m} (\hat{p} - q \vec{A}')^2 |\psi'\rangle, \nonumber \\
  \ham' |\psi'\rangle &= \dfrac{1}{2m} (\hat{p} - q \vec{A} - q \nabla \lambda)^2 |\psi'\rangle = \epsilon |\psi'\rangle.
\end{align}
Let
$|\psi'\rangle = U |\psi\rangle$,
where $U$ is a unitary operator such that
$U^{\dagger} U = \hat{1}$.
Position and momentum expectation values should be the same under both gauge choices.
Starting with position operator we have

\begin{align}
  \langle \psi' | \hat{r} | \psi'\rangle &= \langle \psi | U^{\dagger} \hat{r} U | \psi\rangle = \langle \psi | \hat{r} | \psi\rangle
  %\langle \psi' | \hat{r} | \psi'\rangle &= \langle \psi | U^{\dagger} \hat{r} U | \psi\rangle = \langle \psi | \hat{r} | \psi\rangle \nonumber \\
  %U^{\dagger} \hat{r} U &= \hat{r} \nonumber \\
  %\hat{r} U &= U \hat{r} \nonumber \\
\end{align}
which gives the following useful commutation relation:
$[\hat{r},U] &= 0$, leading to
$[\vec{A}{(\vec{r})},U] = 0$.
For the momentum operator we find

\begin{align}
  \langle \psi' | \hat{p}-q \vec{A}' | \psi'\rangle &= \langle \psi | U^{\dagger} (\hat{p} - q \vec{A}') U | \psi\rangle = \langle \psi | \hat{p} - q \vec{A} | \psi\rangle
  %\langle \psi' | \hat{p}-q \vec{A}' | \psi'\rangle &= \langle \psi | U^{\dagger} (\hat{p} - q \vec{A}') U | \psi\rangle = \langle \psi | \hat{p} - q \vec{A} | \psi\rangle \nonumber \\
  %U^{\dagger} \hat{p} U - U^{\dagger} q\vec{A}'U &= \hat{p} - q \vec{A} \nonumber \\
  %\hat{p} U - q\vec{A} U - q\nabla\lambda U &= U \hat{p} - U q \vec{A} \nonumber \\
\end{align}
that gives the following commutation relation
$[\hat{p},U] = -i\hbar\partial_{\hat{r}} U &= q\nabla\lambda U$.
This leads us to find
$U = \exp[iq\lambda/\hbar]$.
The gauged Hamiltonian is related to the original one by

\begin{align}
  %\ham' |\psi'\rangle &= (\hat{p} - q \vec{A} - q \nabla \lambda)^2 |\psi'\rangle \nonumber \\
  %&= (\hat{p} - q \vec{A} - q \nabla \lambda) \cdot (\hat{p} U - q \vec{A} U - q \nabla \lambda U) |\psi\rangle \nonumber \\
  %&= (\hat{p} - q \vec{A} - q \nabla \lambda) \cdot U (\hat{p} - q \vec{A}) |\psi\rangle \nonumber \\
  %&= U (\hat{p} - q \vec{A})^2 |\psi\rangle \nonumber \\
  \ham' |\psi'\rangle &= U \epsilon |\psi\rangle.
\end{align}
We see the local phase of the wavefunction is changed but it still represents the same energy shown by

\begin{align}
  \langle \psi'| \ham' |\psi'\rangle &= \langle \psi | U^{\dagger} U \epsilon |\psi \rangle = \epsilon, \nonumber \\
  \langle \psi | \ham |\psi \rangle &= \langle \psi | \epsilon |\psi \rangle = \epsilon. \end{align}
Thus,
$U^{\dagger} \ham' U = \ham$, and gauge invariance is shown ~\cite{kittelIntroductionSolidState2018}.
