\section{Peierls phase in tight-binding models}
When working with condensed matter systems we often either work with free particles using Schrodinger's or Dirac's equation or tight-binding models describing how much energy is needed for a particle to ``hop'' from one lattice to the next.
In tight-binding models there is typically no momentum term to use minimal coupling to introduce the gauge potential, but we can find a basis transformation that is equivalent.
There are a few different names this can go by, Aharonov-Bohm effect, Berry phase, geometric phase, or Peierls phase.
There are a few ways to derive Peierls phase and we will use the differential geometry approach.
Before, we showed minimal coupling and now we would like to express it in terms of a covariant derivative

\begin{equation}
  D_{\mu} = \partial_{\mu} - i A_{\mu}.
\end{equation}
Let us now envision how a wavefunction will evolve in the presence of a gauge potential field.
Using the covariant derivative with parallel transport along curves we can obtain an expression for the phase accumulation on the wave function.
The covariant derivative should vanish if it has been parallel transported along the curve $\mathcal{C}$ defined by points $x$ and $x' = x + v t$.
The expression is as follows
$\nabla_{v} s \rightarrow & t v^{\mu} D_{\mu} s_{x(t)} = 0$.
This turns out to be a first order ordinary differential equation
\begin{equation}
  \dot{s}_{x(t)} - i \dot{x}^{\mu}(t) A_{x(t),\mu} s_{x(t)} = 0 \nonumber \\
\end{equation}
with the following solution

\begin{equation}
  s_{x(t)} = s_{x(0)} \exp \left[ i\int_{\mathcal{C}} dx^{\mu} A_{x(t'),\mu} \right]
\end{equation}
and in general rewrite it as the following expression $\psi(t) = \psi(0) \exp \left[ \tfrac{iq}{\hbar} \int_{\mathcal{C}} \vec{A}(\vec{r})\cdot d\vec{l} \right]$  ~\cite{altlandCondensedMatterField2023}.

%\begin{align}
%  \nabla_{v} s = 0 \rightarrow & t v^{\mu} D_{\mu} s_{x(t)} = 0 \\
%  &= t \dot{x}^{\mu}(t) (\partial_{\mu} - i A_u) s_{x(t)} \nonumber \\
%  &= t \dfrac{\partial {x}^{\mu}}{\partial t} \dfrac{\partial}{\partial x_{\mu}} s_{x(t)} - it\dot{x}^{\mu}(t) A_{x(t),\mu} s_{x(t)} \nonumber \\
%  \rightarrow & \dot{s}_{x(t)} - i \dot{x}^{\mu}(t) A_{x(t),\mu} s_{x(t)} = 0 \nonumber \\
%  %&= \partial_t \left( \exp\left[-i\int^t_0 \dot{x}^{\mu} A_{x(t'),\mu} dt' \right] s_{x(t)} \right) = 0 \nonumber \\
%  \rightarrow & s_{x(t)} = s_{x(0)} \exp \left[ i\int^t_0 \dot{x}^{\mu} A_{x(t'),\mu} dt' \right] \nonumber \\
%  & s_{x(t)} = s_{x(0)} \exp \left[ i\int_{\mathcal{C}} dx^{\mu} A_{x(t'),\mu} \right] \nonumber
%\end{align}

Given the following tight-binding Hamiltonian

\begin{equation}
  \ham_t = -t \sum_{\langle j, l \rangle} \cc_j c_{l} + h.c.,
\end{equation}
a gauge potential is applied to the system making the following Peierls phase transform, a unitary transform, to its creation/annihilation operators

\begin{align}
  %c_j \rightarrow c_j \exp \left[{\frac{iq}{\hbar} \int^{\vec{r}_j} \vec{A}\cdot d\vec{l}} \right] \nonumber \\
  \cc_j c_l \rightarrow \cc_j c_l \exp \left[{\frac{iq}{\hbar} \int^{\vec{r}_l}_{\vec{r}_j} \vec{A}\cdot d\vec{l}} \right].
\end{align}
The Hamiltonian in the new basis takes the following form

\begin{equation}
  \ham_t = \sum_{\langle j, l \rangle} -t_{j,l} \cc_j c_{l} + h.c.,
\end{equation}
where $t_{j,l} = t \exp \left[{\frac{iq}{\hbar} \int^{\vec{r}_l}_{\vec{r}_j} \vec{A}\cdot d\vec{l}} \right]$.

