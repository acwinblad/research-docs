\section{Majorana fermions and Topological superconductors}

In brief, Majorana Ettore hypothesised the existence of Majorana fermions, by finding a unique (real) wavefunction solution to Dirac's equation, who found a general (complex) wavefunction solution.
The Standard Model predicts neutrinos to be Majorana fermions yet that has yet to be measured.
One can achieve a Majorana fermion as a linear combination of two fermions, also known as a Bogoliubov quasiparticle, inside a non-trivial topological $p$-wave superconductor.
We have yet to physically realize a $p$-wave superconductor experimentally, however, we can use heterostructures in proximity to an $s$-wave superconductor to achieve an effective $p$-wave superconduting interface, which is explained and referenced in later chapters.
Majorana fermions are dictated by non-Abelian exchange statistics, which allows for building a universal quantum computer, hence why they are highly sought after.
Another boon of using a non-trivial topological superconductor is the ability to protect Majorana fermions from local perturbations.

One of our goals in this disseration is present a new type of topological quantum logic gate that utilizes triangular geometry and rotation of a vector potential field to host and braid Majorana fermions.

Let us go ahead and show how Kitaev derived Majorana zero modes, Majorana fermions, on a 1D spinless $p$-wave superconductor.
Start with a 1d spinless $p$-wave superconductor tight-binding Hamiltonian

\begin{equation}
  \ham = \sum_j^{N-1} (-t \cc_j c_{j+1} + \Delta c_j c_{j+1} + h.c.) - \sum_j^{N} \mu \cc_j c_j,
\end{equation}
where $t$ is hopping amplitude, superconducting order parameter $\Delta = |\Delta|$ for simplicity, $\mu$ is chemical potential, and $\cc (c)$ is the creation (annihilation) operator for a complex fermion.
We next use a basis tranformation to convert to the Majorana fermion basis, where $\cc_j = \tfrac{1}{2} (a_j - i b_j)$, $\left\{ a^{\dagger}_j, a_{j'} \right\} = \left\{ a_j, a_{j'} \right\} = 2\delta_{j,j'}$ since they are Majorana fermions, and $\left\{a_j,b_j'\right\} = 0$.
After some algebra we arrive at

\begin{equation}
  \ham = \dfrac{i}{2} \sum_j \left( -\mu a_j b_j + (t+\Delta) b_j a_{j+1} + (-t+\Delta) a_j b_{j+1} \right).
\end{equation}
In the trivial phase, no Majorana fermions, $\mu \neq 0$ and $t=\Delta=0$,

\begin{equation}
  \ham = -\mu\dfrac{1}{2} \sum_j a_j b_j.
\end{equation}
The non-trivial phase, with Majorana fermions present, $\mu = 0$, and $t = \Delta > 0$,

\begin{equation}
  \ham = it \sum_j b_j a_{j+1}.
\end{equation}
Notice the terms $a_0$ and $b_N$ are missing in the non-trivial Hamiltonian, we can say there is a non-localized zero energy mode present in the system defined by $f = \tfrac{1}{2}(a_0 + i b_N)$, hence the name Majorana zero modes.
Now, slightly outside the Kitaev limit for non-trivial topology, we can limit $|\mu| < 2t$ and $t = |Delta| >0$ and still achieve non-trivial topology with Majorana zero modes at the interface of trivial and non-trivial topology.
To understand why this is still true we can determine the topological invariant for the system, also known as the Majorana number, a type of Winding number for 1D superconducting systems.
Calculating the Majorana number is straight forward enough, its proof on the other hand is not, this can be found in the appendix REFERENCE appendix here.
Write the Hamiltonian in the Majorana basis, $A = -iU \ham U^{\dagger}$, then take the sign of the Pfaffian,

\begin{equation}
  \mathcal{M} = \text{sgn} [\text{Pf} (A)].
\end{equation}
This calculation can be reduced down if we can write the Hamiltonian in momentum space.
Doing so lets us consider the $k=0$ and $k=\pi$ points only in the brillouin zone, due to symmetry.
\begin{equation}
  \mathcal{M} =
  \begin{cases}
    \text{sgn} [\text{Pf} (A_{k=0}) \text{Pf} (A_{k=\pi})], &\text{if L is even}, \\
    \text{sgn} [\text{Pf} (A_{k=0})], &\text{if L is odd},
  \end{cases}
\end{equation}
where L is the number of lattice sites from our lattice Hamiltonian.
We find that under the Kitaev limit, if $|\mu|< 2t$, then $\mathcal{M} = -1$, and if $|\mu| > 2t$, then $\mathcal{M} = 1$.
When a section of the material is in a non-trivial topology and either the other material is trivial or vacuum (also trivial), Majorana zero modes will be localized at the differing topological interfaces, this is also known as bulk-edge correspondence and will be used later in our topological quantum logic gate.
As a last note, when $|\mu| = 2t$ this is a critical point and where the gap opens and closes, it is not an ideal region of parameter space for our needs.

\Blue{Add in some schematics of Kitaev chain in the various cases described}

