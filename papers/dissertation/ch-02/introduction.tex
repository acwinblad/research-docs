\section{Introduction}

For more than twenty years, Majorana zero modes (MZM) in condensed matter systems have been highly sought after due to their potential for serving as building blocks of topological quantum computation, thanks to their inherent robustness against decoherence and non-Abelian exchange statistics \cite{ivanovNonAbelianStatisticsHalfQuantum2001, kitaevFaulttolerantQuantumComputation2003, nayakNonAbelianAnyonsTopological2008, aliceaNonAbelianStatisticsTopological2011, aasenMilestonesMajoranaBasedQuantum2016}. MZM were originally proposed to be found in half-quantum vortices of two-dimensional (2D) topological \textit{p}-wave superconductors and at the ends of 1D spinless \textit{p}-wave superconductors \cite{readPairedStatesFermions2000, kitaevUnpairedMajoranaFermions2001}. Whether a pristine \textit{p}-wave superconductor \cite{brisonPWaveSuperconductivityDVector2021} has been found is still under debate. However, innovative heterostructures proximate to ordinary $s$-wave superconductors have been proposed to behave as effective topological superconductors in both 1D and 2D. These include, for example, semiconductor nanowires subject to magnetic fields \cite{mourikSignaturesMajoranaFermions2012, rokhinsonFractionalJosephsonEffect2012, dengAnomalousZeroBiasConductance2012}, ferromagnetic atomic spin chains \cite{choyMajoranaFermionsEmerging2011, brauneckerInterplayClassicalMagnetic2013, klinovajaTopologicalSuperconductivityMajorana2013,nadj-pergeProposalRealizingMajorana2013,nadj-pergeObservationMajoranaFermions2014,schneiderPrecursorsMajoranaModes2022}, 3D topological insulators \cite{fuSuperconductingProximityEffect2008, hosurMajoranaModesEnds2011, potterEngineeringMathitipSuperconductor2011, veldhorstMagnetotransportInducedSuperconductivity2013}, quantum anomalous Hall insulators \cite{chenQuasionedimensionalQuantumAnomalous2018, zengQuantumAnomalousHall2018, xieCreatingLocalizedMajorana2021}, quasi-2D spin-orbit-coupled superconductors with a perpendicular Zeeman field \cite{oregHelicalLiquidsMajorana2010, sauGenericNewPlatform2010, lutchynSearchMajoranaFermions2011, potterTopologicalSuperconductivityMajorana2012, liTwodimensionalChiralTopological2016, leiUltrathinFilmsSuperconducting2018}, and planar Josephson junctions \cite{black-schafferMajoranaFermionsSpinorbitcoupled2011, pientkaSignaturesTopologicalPhase2013, hellTwoDimensionalPlatformNetworks2017, fornieriEvidenceTopologicalSuperconductivity2019, renTopologicalSuperconductivityPhasecontrolled2019, scharfTuningTopologicalSuperconductivity2019, zhouPhaseControlMajorana2020}, etc. It has been a challenging task to decisively confirm the existence of MZM in the various experimental systems due to other competing mechanisms that can potentially result in similar features as MZM do in different probes \cite{xuExperimentalDetectionMajorana2015, albrechtExponentialProtectionZero2016, sunMajoranaZeroMode2016, wangEvidenceMajoranaBound2018, jackObservationMajoranaZero2019, fornieriEvidenceTopologicalSuperconductivity2019, renTopologicalSuperconductivityPhasecontrolled2019, mannaSignaturePairMajorana2020}. Other proposals for constructing Kitaev chains through a bottom-up approach, based on, e.g. magnetic tunnel junctions proximate to spin-orbit-coupled superconductors \cite{fatinWirelessMajoranaBound2016}, and quantum dots coupled through superconducting links \cite{sauRealizingRobustPractical2012,leijnseParityQubitsPoor2012,dvirRealizationMinimalKitaev2023} are therefore promising. In particular, the recent experiment \cite{dvirRealizationMinimalKitaev2023} of a designer minimal Kitaev chain based on two quantum dots coupled through tunable crossed Andreev reflections (CAR) offers a compelling route towards MZM platforms based on exactly solvable building blocks.

In parallel with the above efforts of realizing MZM in different materials systems, scalable architectures for quantum logic circuits based on MZM have also been intensely studied over the past decades. A major proposal among these studies is to build networks of T-junctions, which are minimal units for swapping a pair of MZM hosted at different ends of a junction, that allow braiding-based TQC \cite{aasenMilestonesMajoranaBasedQuantum2016}. Alternatively, networks based on coupled wires forming the so-called tetrons and hexons, aiming at measurement-based logic gate operations  \cite{karzigScalableDesignsQuasiparticlepoisoningprotected2017}, have also been extensively investigated. To counter the technical challenges of engineering networks with physical wires or atomic chains, various ideas based on effective Kitaev chains, such as quasi-1D systems in thin films \cite{potterMultichannelGeneralizationKitaev2010}, cross Josephson junctions \cite{zhouPhaseControlMajorana2020}, scissor cuts on a quantum anomalous Hall insulator \cite{xieCreatingLocalizedMajorana2021}, and rings of magnetic atoms \cite{liManipulatingMajoranaZero2016}, etc. have been proposed. However, due to the same difficulty of obtaining or identifying genuine MZM in quasi-1D systems mentioned above, it remains unclear how practical these strategies are in the near future.

In this Letter, we propose an alternative structural unit for manipulating MZM, triangular superconducting islands, motivated by the above challenges associated with wire geometries and by the fact that triangular islands routinely appear spontaneously in epitaxial growth \cite{pietzschSpinResolvedElectronicStructure2006} on close-packed atomic surfaces. We first show that a minimal ``Kitaev triangle'' consisting of three sites hosts MZM at different pairs of vertices controlled by Peierls phases on the three edges [Fig.~\ref{fig:triangles} (a)], which can be readily realized using quantum dots. To generalize the minimal model to triangular structures involving more degrees of freedom, we study the topological phase transitions of quasi-1D ribbons driven by Peierls phases, which can be created by magnetic fields or supercurrents \cite{romitoManipulatingMajoranaFermions2012, takasanSupercurrentinducedTopologicalPhase2022}, and use the resulting phase diagram as a guide to construct finite-size triangles with a hollow interior that host MZM  [Fig.~\ref{fig:triangles} (b)]. In the end we discuss possible experimental systems that can realize our proposals and scaled-up networks of triangles for implementing braiding operations of MZM.


