\section{Discussion and conclusion}

Results can be explained with the help of existing experiments \cite{YHW, JWM} and can provide an estimate for the strength of the effective magnetic field to observe LLs and QHE. Analytical structure of Eq.~(\ref{eq:DiracEner}) and Eq.~(\ref{eq:Energy}) is primarily responsible for the LLs spectrum in both the Dirac and Schr\"{o}dinger systems, respectively.
%Therefore, the effective magnetic field strength can be calculated with the help of parameters realized/used/ implemented in these experimental observations.
Although such results are valid for other 2D materials or Schr\"{o}dinger systems, however, for simplicity, we will consider parameters realized for graphene or topological insulators \cite{YHW, JWM}. In these experiments \cite{YHW, JWM}, the strength of the electric field used is $1 \times 10^7$ V/m to $1 \times 10^8$ V/m and the frequency of the light varies from 120 meV to 191 meV.

In case of Dirac electrons, we calculate the effective magnetic field strength using Eq.~(\ref{eq:DiracEner}). For a fixed value of the spatial period of 120 nm and frequency of laser light $\hbar \omega = 191$ meV, the strength of the effective magnetic field is $\approx$ 10 Tesla for electric field strength of $5 \times 10^7$ V/m \cite{JWM}. Moreover, by reducing the spatial period to 12 nm, we obtain the effective magnetic field $\approx$ 98 Tesla for fixed frequency ($\hbar \omega = 191 $ meV) and electric field ($5 \times 10^7$ V/m). This is due to the fact that effective magnetic field is directly proportional to electric field and inversely proportional to the spatial period of light according to Eq.~(\ref{eq:DiracEner}). Similarly, keeping the spatial period constant and increasing the electric field strength, we can increase the strength of the effective magnetic field for larger frequencies only. However, for the frequency $\hbar \omega = 191$ meV, we can not go beyond $5 \times 10^7$ V/m value of the electric field irrespective of the spatial period. This limitation is due to the factor "C" in Eq.~(\ref{eq:DiracEner}). Further, using lower light frequency ($\hbar \omega = 120$ meV) as realized in topological insulators experiments \cite{YHW}, the critical strength of the electric field is $2 \times 10^7$ V/m beyond which magnetic field will become negative. Additionally, for larger frequency of 220 meV, the maximum value of electric field $7 \times 10^7$ V/m can be used. It is also important and interesting to note that negative values of the effective magnetic field at larger strengths of light's electric field are fruitful. This is because positive or negative values of the effective magnetic field means (see Eq.~(\ref{eq:HeffB})) that magnetic field is applied either from positive z-axis or negative z-axis. \textbf{This estimate of parameters is equally valid for frequency space expansion results (Fig. 1) obtained numerically and degenerate perturbation (Fig. 2) analysis.}

In case of Schr\"{o}dinger electrons in conventional 2DEG systems, the effective magnetic field strength can be obtained using Eq.~(\ref{eq:Energy}). For a fixed value of the spatial period of 100 nm and light frequency of $\hbar \omega = 191$ meV, the strength of the effective magnetic field is 1 Tesla using electric field of $5 \times 10^7$ V/m. Further, by reducing the spatial period to 10 nm and keeping the electric field ($5 \times 10^7$ V/m) and frequency of light ($\hbar \omega = 191 $ meV) fixed, we obtain the effective magnetic field $\approx$ 106 Tesla. This can be understood from Eq.~(\ref{eq:Energy}). In contrast to Dirac case, by increasing the strength of electric field of light, we can increase the strength of the effective magnetic field values. For example, for fixed period of 100 nm and frequency ($\hbar \omega = 191 $ meV), the effective magnetic field is 17 Tesla for $2 \times 10^8$ V/m. Next, we see the impact of increasing or decreasing frequency and keeping period (100 nm) and electric field ($5 \times 10^7$ V/m) fixed. For decreasing frequency to 120 meV, we obtain larger effective magnetic field (4.3 Tesla) while increasing frequency leads to smaller effective magnetic field (0.7 Tesla at 220 meV light frequency). \textbf{This estimate of parameters is equally valid for frequency space expansion results (Fig. 1) obtained numerically and degenerate perturbation (Fig. 2) analysis.}

In conclusion, we have shown Floquet LLs and the QHE using two linearly polarized lights for graphene-like 2D and conventional 2DEG systems. While using these laser lights, we  need at least one or both polarized lights to be spatially inhomogeneous. We have presented results using frequency space expansion method, degenerate Floquet perturbation theory, tight binding models and numerical model calculations. All the results are agreed well to show Floquet LLs in experimentally accessible parameters range. Also, it is interesting to note that we are flexible to use different values of the electric field strength, frequency or spatial period of the light to realize QHE and control the strength of the effective magnetic field. Therefore, we believe that Floquet LLs and QHE can be observed in the experiments for moderate strength of the spatially inhomogeneous lights \textbf{as shown in Figs. 1 and 2.} Moreover, we expect the potential to host new nano-electronics in nonequilibrium systems.

