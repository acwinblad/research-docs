\section{Floquet LLs in 2DEG}
Next, similar to Dirac electrons, we consider the case of Schr\"{o}dinger electrons under the application of two linearly polarized laser lights. The unperturbed Hamiltonian for 2DEG is%
\begin{equation}\label{eq:H2DEG}
H=\frac{\pi_{x}^{2}}{2m^{\ast}}+\frac{\pi_{y}^{2}}{2m^{\ast}},
\end{equation}
where $m^{\ast}$ is the effective mass of electron. By changing the Hamiltonian into a time-dependent form by applying two linearly polarized lights
such that $\bm{\pi}\rightarrow \vec{p-eA(t)}$. \ Therefore, Eq.~\eqref{eq:H2DEG} is written as%
\begin{equation}\label{eq:H2time}
H(t)=\frac{1}{2m^{\ast}}[p_{x}+eA_{x}(t)]^{2}+\frac{1}{2m^{\ast}}[p_{y}%
+eA_{y}(t)]^{2},
\end{equation}
where the electric field components for two spatially inhomogeneous linearly polarized laser lights are
\begin{align} \label{eq:E2field}
\vec{E}_{1} &=E\cos (\omega t)\hat{x}, \nonumber \\
\vec{E}_{2}^{\prime}&=E\cos(Kx)\sin (\omega t)\hat{y}.
\end{align}%
It is important to note that second electric field in Eq.~\eqref{eq:E2field} is different from similar field used for Dirac spectrum given in Eq.~\eqref{eq:Efield}. This is due to the fact that Schr\"{o}dinger Hamiltonian is quadratic rather than linear as in case of Dirac electrons. This is basic requirement for the Schr\"{o}dinger electron spectrum to exhibit LLs. The field give in Eq.~\eqref{eq:E2field} lead to the following vector potential $\vec{A}(t)$
\begin{equation}\label{eq:A2vector}
\vec{A}(t)= \left\langle -V_1\sin (\omega t), V_2 \cos (\omega t),0 \right\rangle
\end{equation}%
where we have $V_{1}=\frac{eE}{\omega },V_{2}=V_1\cos(Kx)$. Employing the Floquet theory similar to Dirac electrons, we obtain the effective Hamiltonian as%
\begin{equation}\label{eq:H2effec}
	H_{\rm eff}  =H_{0} -\frac{U}{m^{\ast}}\sin (Kx) p_{y} -\frac{U^2}{4m^{\ast}}V_{1}^{2}\cos(2Kx).
\end{equation}
For Landau Level problem, we usually use the Landau gauge with vector potential
like $A=(0,xB,0),B=B\hat{z}$. In the long wavelength limit, Eq.~\eqref{eq:H2effec} can be simplified to
\begin{equation} \label{eq:H2Feffect}
H_{\rm eff}=\frac{p_{x}^{2}}{2m^{\ast}}+\frac{1}{2m^{\ast}}[p_{y}%
+eBx]^{2}-\frac{U^{2}}{4m^{\ast}},
\end{equation}
where $U=\frac{KV_{1}^{2}}{2m^{\ast}\omega}$, and the effective magnetic field
$B=\frac{K^{2}V_{1}^{2}}{em^{\ast}\omega}$. Eq.~\eqref{eq:H2Feffect} is a standard LL problem in the presence of an external perpendicular
magnetic field. Therefore, by diagonalizing the effective Hamiltonian, the corresponding energy eigenvalues are obtained as%
\begin{equation} \label{eq:Energy}
E_{n}=(n+\frac{1}{2})\hbar\omega_{c}-\frac{U^{2}}{4m^{\ast}},
\end{equation}
where $\omega_{c}=\frac{eB}{m^{\ast}}$. We can see from Eq.~\eqref{eq:Energy} that the effective magnetic field is directly proportional to the strength of the electric field (second order) and inversely proportional to the product of second order spatial period and third order of the frequency of the laser lights $\propto (E^2/(a^2\omega^3))$. This factor of the laser lights can be tuned to enhance the strength of the effective magnetic field in nonequilibrium systems.

