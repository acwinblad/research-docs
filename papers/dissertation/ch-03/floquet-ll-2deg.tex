\section{Floquet Landau level-like bands in 2DEG systems}
In this section we show a 2DEG system with a standing, non-uniform, circularly polarized light becomes an effective 2DEG Hamiltonian with an effective magnetic field composed of the electric field component of light.
We consider the case of Schr\"{o}dinger charged particles in the presence of a gauge potential as
\begin{equation}\label{eq:H2deg}
  \ham(t) = \dfrac{1}{2m^*} \left( \vec{p} + e \vec{A}(t)\right)^2
\end{equation}
where $m^*$ is the effective charge mass.
We again consider a similar setup with three linearly polarized lasers.
The relevant electric field components at the 2DEG interface are
\begin{align} \label{eq:E2field}
  \vec{E}_{1} &= E \cos{\omega t}\ \hat{x}, \nonumber \\
  \vec{E}_{2} &= \vec{E}_2^f + \vec{E}_2^b = -E\cos{(K x)} \sin{\omega t}\ \hat{y}.
\end{align}%
Unlike the Dirac system, all lasers have them same frequency.
This ensures a LL-like behavior from the nonzero commutation of $x$ and $p_x$ terms during the high-frequency expansion.
The electric field relates to the following gauge potential, via $\vec{E} = -\partial_t \vec{A}$ as

\begin{equation}\label{eq:A2deg}
  \vec{A}(t)= \dfrac{E}{\omega} \left\langle -\sin \omega t, \cos{(Kx)} \cos{\omega t} \right\rangle.
\end{equation}%
Substituting Eq. \eqref{eq:A2deg} into Eq. \eqref{eq:H2deg}, we arrive at

\begin{align}\label{eq:H2time}
  \ham(t) = \dfrac{1}{2m^*} \Bigl[ &p_x^2 + p_y^2 + \dfrac{e^2 E^2}{2\omega^2}(1-\cos{2\omega t}) + \dfrac{e^2 E^2}{2\omega^2}(1+\cos{2\omega t}) \cos^2{(Kx)} \nonumber \\
  &+\dfrac{2eE}{\omega} \left(p_y \cos{(Kx)}\cos{\omega t} -  p_x \sin{\omega t}\right) \Bigr].
\end{align}
Again, due to laser's time-translation symmetry we can apply the Floquet theory, which includes computing the Fourier time-transform using Eq. \eqref{eq:fourier-time-transform}.
This leaves us with modes $m=0,\pm1,\pm2$.
In the high-frequency expansion we still require $\hbar \omega \gg H_{\pm1,\pm2}$.
After applying the high-frequency approximation to first order in $\hbar \omega$, it leads to the zeroth-mode effective Hamiltonian as

\begin{align}\label{eq:H2eff}
  %\ham_{\text{eff}} &= \dfrac{1}{2m^*} \left[ p_x^2 + p_y^2 + \dfrac{e^2 E^2}{\omega^2}\left(1+\cos^2{(Kx)}\right) - \dfrac{2 K e^2 E^2 p_y \sin{(Kx)}}{m^*\omega^3} \right] \nonumber \\
  \ham_{\text{eff}} &= \dfrac{1}{2m^*} \left[ p_x^2 + \left(p_y - \dfrac{K e^2 E^2 \sin{(Kx)}}{m^*\omega^3} \right)^2 + \dfrac{e^2 E^2 \cos^2{(Kx)}}{\omega^2}  - \dfrac{K^2 e^4 E^4 \sin{(Kx)}}{m^{*2}\omega^6} \right]. \nonumber
\end{align}
where we shifted a constant energy out of the effective Hamiltonian and completed the square for the $p_y$ and $x$ terms.
Here we can see the high-frequency expansion terms $H^{F(1)}$ and $H^{F(2)}$, as shown in \ref{app:quantum-floquet-theory}, introduce a periodic potential in the x-direction, this can be cancelled out by applying an external periodic potential of the same strength and wavenumber in the x-direction.
Finally, in the long wavelength, $Kx \ll 1$,

\begin{equation}\label{eq:Heff2deg}
\ham_{\text{eff}}^{\text{2DEG}} &= \dfrac{1}{2m^*} \left[ p_x^2 + {\left(p_y - eB^{\text{2DEG}}x \right)}^2  \right],
\end{equation}
with $B^{\text{2DEG}} = \tfrac{K^2 e E^2 }{m^*\omega^3}$.
The energy spectrum values are
\begin{equation}\label{eq:2DEGenergy}
  \epsilon_n^{\text{2DEG}} = \dfrac{\hbar K^2 e^2 E^2}{m^{*2}\omega^3} \left(n+\dfrac{1}{2}\right),
\end{equation}
which is similar to 2DEG LLs spectrum.
This is another highly degenerate LL-like spectrum due to an effective magnetic field induced by the combination of lasers provided.

The 2DEG system has the same ways to enhance its effective magnetic field and energy spectrum.
First, the electric field is limited by the high-frequency expansion, we find $E \ll (8m^*\hbar\omega^3/e^2)^{1/2}$ to be a reasonable cutoff.
The effective field $B^{\text{2DEG}} \propto \omega^2 \sin^2{(\theta_i)} / v_p^2$, is similar to the Dirac system.
For a 2DEG system, the enhancement to the effective magnetic field follows a squared dependence on the incident angle and phase velocity, differing from the Dirac case, where it scales linearly and by Fermi velocity.
Similarly, increasing the photon energy is one way to enhance the effective magnetic field, with the requirement of using shorter pulses as photon energy and electric field increase.

%We have shown one can achieve a highly degenerate Landau level-like spectrum and QHE in a 2DEG system with an effective magnetic field proportional to electric field  squared and inverserly proportional to the product of spatial period squared and angular frequency cubed $\propto (E^2/(d^2\omega^3))$.
%These factors of the laser lights can be tuned and be enhanced in such nonequilibrium systems.
%The effective magnetic field strength obtained for 2DEG is proportional to $ B \propto E^2/d^2\omega^3$.  %This allows for additional tunability of an effective magnetic field in nonequilibrium systems.
