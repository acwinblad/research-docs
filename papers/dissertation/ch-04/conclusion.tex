In this dissertation, we focused on gauge potentials as a key mechanism for tuning systems from trivial topology to non-trivial topology.
One can either use minimal coupling to introduce a gauge potential in a continuum model or a Peierls phase in a tight-binding model.
Both approaches are equivalent for slowly varying gauge potentials.
This lends a lot of flexibility in tackling a Hamiltonian in either form, continuum or tight-binding.
Another key aspect of gauge potentials is gauge invariance under a gauge transform.
This allows us to transform a Hamiltonian into a basis that is more descriptive of the physics we are interested in.
%This is more apparent in Chapter 2 when examining the exactly solvable minimal model for a superconducting equilateral triangle of three lattice sites.
%Depending on the parameters in the gauge potential we are able to tune the parameters to see when a system changes topology from trivial to non-trivial.

In Chapter 2, we showed through a Peierls phase of a gauge potential in a superconducting tight-binding model in Majorana basis allows one to achieve non-trivial topology.
We showed for a three lattice superconducting equilateral triangle one can apply a gauge potential as a step function exactly solves the Hamiltonian to have two MFs on two of the triangles three vertices.
With a rotation of the gauge potential via a linear interpolation, we can keep the band gap from closing and rotate one MF at a time to the next vertex.
Then, for larger triangular islands, as a chain or with finite thick edges, we can use bulk-edge correspondence to inform us which edge is trivial and non-trivial.
Using a uniform gauge potential across the triangle showed, due to geometry and the gauge potential contributions, individual edges topology could be tuned from trivial to non-trivial topology.
The phase diagrams showed there are swaths of tunable parameters to then make the topology crawl cw or ccw, without band gap closing, around the triangular island, effectively rotating two MFs around a triangle.
With two options for hosting and rotating MFs we finally showed a minimal network of triangles to host and braid 4 MFs, which can be upscaled to increase the number of MFs for braiding operations.
This opens up new routes to achieve fault-tolerant, topological quantum computing.

In Chapter 3, we showed through minimal coupling of a gauge potential, created from laser light, for both Dirac and 2DEG systems, one can induce LL-like spectrum and QHE.
This used Floquet theory and high-frequency expansion, and allows us to approximate an effective magnetic field as a function of electric field and arrive at LL-like Hamiltonians, to which we know the physics of in equilibrium systems.
There are many tunable parameters to enhance the LL-like spectrum and QHE, from photon energy, electric field, phase velocity, incidence angle, to Fermi velocity or effective electron mass.
The estimated values are achievable with current experimental setups found in previous literature and comparable with classic magnetic field QHE.
These avenues open up new frontiers in nonequilibrium physics.

