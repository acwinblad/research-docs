\section{Kitaev chain}
A pair of Majorana fermions can be defined in terms of spinless operators $a_j, a_j^\dag$, where $j$ labels general quantum numbers, as
\begin{eqnarray}\label{eq:atoc}
	&&c_{2j-1} = a_j+ a_j^\dag\\\nonumber
	&&c_{2j} = -i(a_j - a_j^\dag)
\end{eqnarray}
which are hermitian conjugates of themselves. Conversely,
\begin{eqnarray}\label{eq:ctoa}
	&&a_j =\frac{1}{2} (c_{2j-1} + i c_{2j}),\\\nonumber
	&&a_j^\dag = \frac{1}{2}(c_{2j-1} - i c_{2j})
\end{eqnarray}
For a general mean-field Hamiltonian
\begin{eqnarray}
	H = \sum_{jl} h_{jl} a_j^\dag a_l
\end{eqnarray}
where $h$ is a Hermitian matrix, it can be transformed to the Majorana fermion representation as follows:
\begin{eqnarray}\label{eq:BdGMrep}
	H &=& \frac{1}{4}\sum_{jl} h_{jl} (c_{2j-1} - i c_{2j}) ( c_{2l-1} + i c_{2l})\\\nonumber
	&=& \frac{1}{4}\sum_{jl} (h_{jl} c_{2j-1} c_{2l-1} - i h_{jl} c_{2j} c_{2l-1} + ih_{jl} c_{2j-1} c_{2l} + h_{jl} c_{2j} c_{2l}) \\\nonumber
	&=& \frac{i}{4} \sum_{jl} (c_{2j-1}, c_{2j})\begin{pmatrix}
		-ih_{jl} & h_{jl}\\
		-h_{jl} & -ih_{jl}
	\end{pmatrix}\begin{pmatrix}
	c_{2l-1} \\
	c_{2l}
\end{pmatrix}\\\nonumber
&\equiv&\frac{i}{4}\sum_{mn}A_{mn}c_{m}c_n
\end{eqnarray}
where the matrix $A$ anti-Hermitian since
\begin{eqnarray}
	H^\dag = -\frac{i}{4}\sum_{mn}A_{mn}^* c_n c_m =  H
\end{eqnarray}
which leads to $A_{mn}^* = -A_{nm}$. If the Hamiltonian does not preserve particle number, i.e., it is a BdG Hamiltonian, we have, similarly
\begin{eqnarray}
H = \sum_{jl} \left(h_{jl} a_j^\dag a_l + \Delta_{jl} a_j a_l + \Delta^\dag_{jl}  a_j^\dag a_l^\dag\right)
\end{eqnarray}
supposing we do not double count the terms in the normal part of the Hamiltonian. Then
\begin{eqnarray}
H = \frac{i}{4} \sum_{jl} (c_{2j-1}, c_{2j})\begin{pmatrix}
	-ih -i(\Delta + \Delta^\dag) & h + (\Delta-\Delta^\dag )\\
	-h + (\Delta - \Delta^\dag) & -ih + i(\Delta +\Delta^\dag)
\end{pmatrix}_{jl}\begin{pmatrix}
	c_{2l-1} \\
	c_{2l}
\end{pmatrix}\\\nonumber
\end{eqnarray}
On the other hand, the BdG Hamiltonian can be written as
\begin{eqnarray}\label{eq:HBdG}
	H &=&\frac{1}{2} \sum_{j}h_{jj} + \sum_{jl} \left( \frac{1}{2}h_{jl} a_j^\dag a_l - \frac{1}{2}h_{lj} a_j a_l^\dag + \Delta_{jl} a_j a_l + \Delta^\dag_{jl}  a_j^\dag a_l^\dag \right) \\\nonumber
	&=& \frac{1}{2}{\rm Tr}(h) + \sum_{jl} (a_j^\dag, a_j)\begin{pmatrix}
		\frac{1}{2} h & \Delta^\dag \\
		\Delta & -\frac{1}{2}h^T
	\end{pmatrix}_{jl}\begin{pmatrix}
	a_l \\
	a_l^\dag
\end{pmatrix} \\\nonumber
&\equiv & \frac{1}{2}{\rm Tr}(h) + a^\dag \begin{pmatrix}
	\frac{1}{2} h & \Delta^\dag \\
	\Delta & -\frac{1}{2}h^T
\end{pmatrix} a
\end{eqnarray}
where $a\equiv(a_1, a_2,\dots, a_1^\dag, a_2^\dag,\dots)^T$. Suppose the BdG Hamiltonian can be diagonalized by a Bogoliubov transformation:
\begin{eqnarray}\label{eq:bogoliubov}
	\tilde{a}_j^\dag &=& a_{l}^\dag U_{1,lj} + a_{k} U_{2,kj} \\\nonumber
	\tilde{a}_j &=& a_{l} U^*_{1,lj} + a^\dag_{k} U^*_{2,kj}
\end{eqnarray}
Preserving the anticommutation relation suggests
\begin{eqnarray}
	\delta_{ij} &=& \{\tilde{a}_{i}, \tilde{a}_j^\dag \} = \{a_{l} U^*_{1,li} + a^\dag_{k} U^*_{2,ki},  a_{l'}^\dag U_{1,l'j} + a_{k'} U_{2,k'j}  \} \\\nonumber
	&=&  U^*_{1,li} U_{1,lj} + U^*_{2,ki}U_{2,kj} = (U_1^\dag U_1 + U_2^\dag U_2)_{ij} \\\nonumber
	0 &=& \{\tilde{a}_{i}, \tilde{a}_j \} =\{a_{l} U^*_{1,li} + a^\dag_{k} U^*_{2,ki}, a_{l'} U^*_{1,l'j} + a^\dag_{k'} U^*_{2,k'j}  \} \\\nonumber
	&=& U^*_{1,li}U^*_{2,lj} + U^*_{2,ki}U^*_{1,kj} = (U^\dag_1 U_2^* + U^\dag_2 U_1^*)_{ij}\\\nonumber
	0 &=& \{\tilde{a}_{i}^\dag, \tilde{a}_j^\dag \} =\{a_{l}^\dag U_{1,li} + a_{k} U_{2,ki}, a^\dag_{l'} U_{1,l'j} + a_{k'} U_{2,k'j}  \} \\\nonumber
      &=& U_{1,li}U_{2,lj} + U_{2,ki}U_{1,kj} = (U^\dag_1 U_2^* + U^\dag_2 U_1^*)^*_{ij}
\end{eqnarray}
The above identities simply indicate
\begin{eqnarray}
\mathcal{U} \equiv \begin{pmatrix}
		U_1 & U_2 \\
		U_2^* & U_1^*
	\end{pmatrix}
\end{eqnarray}
is a unitary matrix. Therefore one can diagonalize the BdG Hamiltonian using usual unitary matrices obtained from the eigenvectors of the matrix in the last line of Eq.~\ref{eq:HBdG}:
\begin{eqnarray}\label{eq:HBdGdiag}
	H &=& \frac{1}{2}{\rm Tr}(h) + \tilde{a}^\dag U^\dag \begin{pmatrix}
		\frac{1}{2} h & \Delta^\dag \\
		\Delta & -\frac{1}{2}h^T
	\end{pmatrix} U \tilde{a}\\\nonumber
&\equiv& \frac{1}{2}{\rm Tr}(h) + \frac{1}{2}\tilde{a}^\dag \begin{pmatrix}
	\epsilon & 0 \\
	0 & -\epsilon
\end{pmatrix} \tilde{a}\\\nonumber
&=& \frac{1}{2}\sum_{j}h_{jj} +\frac{1}{2} \sum_{j}(\epsilon_j \tilde{a}_j^\dag \tilde{a}_j - \epsilon_j \tilde{a}_j \tilde{a}_j^\dag) \\\nonumber
&=&  \frac{1}{2}\sum_{j}h_{jj} +\sum_{j}\epsilon_j(\tilde{a}_j^\dag \tilde{a}_j - \frac{1}{2})
\end{eqnarray}
where $\epsilon \equiv {\rm Diag}[\epsilon_1, \epsilon_2,\dots]$. To see why $U$ can transform the BdG Hamiltonian into such a diagonal matrix with opposite eigenvalues at the same positions in the upper-left and lower-right blocks, we check these two terms explicitly
\begin{eqnarray}
	&&\frac{1}{2}(\epsilon_{1} \tilde{a}_j^\dag \tilde{a}_j - \epsilon_{2} \tilde{a}_j \tilde{a}_j^\dag)\\\nonumber
	&=& \frac{\epsilon_1}{2} (a_l^\dag U_{1,lj} + a_k U_{2,kj}) (a_{l'} U^*_{1,{l'}j} + a^\dag_{k'} U^*_{2,k'j}) - \frac{ \epsilon_2 }{2} (a_{l'} U^*_{1,{l'}j} + a^\dag_{k'} U^*_{2,k'j})(a_l^\dag U_{1,lj} + a_k U_{2,kj}) \\\nonumber
	&=& \frac{1}{2}(\epsilon_1 U_{1,lj} U_{1,l'j}^* - \epsilon_2 U^*_{2,lj} U_{2,l'j}) a^\dag_{l} a_{l'} + \frac{1}{2}(\epsilon_1 U_{2,kj} U^*_{2,k'j} -  \epsilon_2 U^*_{1,kj} U_{1, k'j}) a_k a_{k'}^\dag \\\nonumber
	&+& \frac{1}{2} (\epsilon_1 U_{1,lj} U_{2,k'j}^* - \epsilon_2 U_{2,lj}^* U_{1,k'j}) a_{l}^\dag a_{k'}^\dag + \frac{1}{2} (\epsilon_1 U_{2,kj} U^*_{1,l'j} -\epsilon_2 U_{1,kj}^* U_{2,l'j} ) a_{k} a_{l'}
\end{eqnarray}
Comparing them with the original BdG Hamiltonian Eq.~\ref{eq:HBdG}, we can conclude that
\begin{eqnarray}
&&	\epsilon_1 U_{1,lj} U_{1,l'j}^* - \epsilon_2 U^*_{2,lj} U_{2,l'j} = - (\epsilon_1 U_{2,l'j} U^*_{2,lj} -  \epsilon_2 U^*_{1,l'j} U_{1, lj}) \\\nonumber
&& \epsilon_1 U_{1,lj} U_{2,k'j}^* - \epsilon_2 U_{2,lj}^* U_{1,k'j} = (\epsilon_1 U_{2,k'j} U^*_{1,lj} -\epsilon_2 U_{1,k'j}^* U_{2,lj} )^*
\end{eqnarray}
or equivalently
\begin{eqnarray}
	&&(\epsilon_1 - \epsilon_2) (U_{1,lj} U_{1,l'j}^* + U_{2,l'j} U^*_{2,lj} )= 0 \\\nonumber
	&& 0 = 0
\end{eqnarray}
The first equation therefore dictates $\epsilon_1 = \epsilon_2$. However, note that $\epsilon_j$ does not have to be all positive. If the original normal state Hamiltonian can be diagonalized into a form
\begin{eqnarray}
	H = \sum_{j} \tilde{\epsilon}_{j} b^\dag_j b_j
\end{eqnarray}
where $\tilde{\epsilon}_{j}$ can be either positive or negative, the state with the lowest possible energy from the system is
\begin{eqnarray}
	|\Omega \rangle =  \prod_{\tilde{\epsilon}_k < 0} b_k^\dag | 0\rangle
\end{eqnarray}
where $|0\rangle$ is the true vacuum with no particles in any sense. This is the ground state, and its parity is therefore determined by the number of single-particle eigenstates that have negative energies.

From the same perspective, one can define the ground state of the BdG Hamiltonian Eq.~\ref{eq:HBdGdiag} \emph{for a specific set of Fermion operators} $\tilde{a}_j$, as
\begin{eqnarray}
	|\Omega_{\rm BdG}\rangle = \prod_{\epsilon_j < 0} \tilde{a}_j^\dag | 0_{\rm BdG} \rangle
\end{eqnarray}
where $| 0_{\rm BdG} \rangle$ is certain ``vacuum state" for the given set of $\tilde{a}_j$:
\begin{eqnarray}
	\tilde{a}_j | 0_{\rm BdG} \rangle = 0,\, \forall j
\end{eqnarray}
The number of particles in the ground state is therefore determined by the number of negative $\epsilon_j$. However, the sign of $\epsilon_j$ in the present case does not have absolute meaning. For example, supposing $\epsilon_{j_0}<0$, the following Bogoliubov transformation
\begin{eqnarray}\label{eq:Bgtranswitch}
	\tilde{a}'_{j_0} = \tilde{a}_{j_0}^\dag \\\nonumber
	{\tilde{a}}^{\prime \dag}_{j_0} = \tilde{a}_{j_0}
\end{eqnarray}
that only interchanges the creation and annihilation operators for a single $j_0$ and keeps the others unchanged, leads to a different ground state
\begin{eqnarray}
	|\Omega'_{\rm BdG}\rangle = \prod_{\epsilon_j < 0, j\neq j_0} \tilde{a}_j^\dag | 0_{\rm BdG} \rangle
\end{eqnarray}
since the number of negative energy eigenstates decreases by one. The transformation in Eq.~\ref{eq:Bgtranswitch} therefore neither preserves particle number nor particle parity. This can be explicitly checked. The particle number operator in the $\tilde{a}_j$ representation is
\begin{eqnarray}
	N = \sum_{j} \tilde{a}_j^\dag \tilde{a}_j
\end{eqnarray}
Therefore
\begin{eqnarray}
	N' = \sum_{j\neq j_0} \tilde{a}_j^\dag \tilde{a}_j + \tilde{a}_{j_0} \tilde{a}^\dag_{j_0} = N + (1-2 \tilde{a}_{j_0}^\dag \tilde{a}_{j_0})
\end{eqnarray}
The latter equation also means that when the $j_0$ state is occupied in $\Omega_{\rm BdG}$, i.e., $\tilde{a}_{j_0}^\dag \tilde{a}_{j_0} |\Omega_{\rm BdG}\rangle = |\Omega_{\rm BdG}\rangle$, the transformation decreases the total number of particles by 1. Conversely, if it is unoccupied ($\epsilon_{j_0} > 0$), or $\tilde{a}_{j_0}^\dag \tilde{a}_{j_0} |\Omega_{\rm BdG}\rangle = 0$, the transformation increases the particle number by one.

The fermion parity operator can be defined by
\begin{eqnarray}
	P = \prod_j (1-2a_j^\dag a_j)
\end{eqnarray}
so that $P = 1$ if the number of occupied states is even, and $-1$ otherwise. Alternatively, since
\begin{eqnarray}
	a_j^\dag a_j = \frac{1}{4}(c_{2j-1} - ic_{2j})(c_{2j-1} + ic_{2j}) = \frac{1}{2} (1 + ic_{2j-1} c_{2j})
\end{eqnarray}
$P$ can be written in the Majorana representation as
\begin{eqnarray}\label{eq:Pm}
	P = \prod_j (-i c_{2j-1} c_{2j}).
\end{eqnarray}
Eq.~\ref{eq:Bgtranswitch} transforms $P$ by changing
\begin{eqnarray}
	1 - 2\tilde{a}^\dag_{j_0} \tilde{a}_{j_0}\rightarrow 	1 - 2\tilde{a}'_{j_0}\tilde{a}^{\prime\dag}_{j_0} = - (1-\tilde{a}^{\prime\dag}_{j_0}\tilde{a}'_{j_0})
\end{eqnarray}
and hence indeed changes $P$.

We can also understand why the BdG Hamiltonian preserves $P$ but not $N$. Due to the following commutation relations:
\begin{eqnarray}
	&&[a_j^\dag a_{l}, a_j^\dag a_j ] = a_j^\dag a_{l} a_j^\dag a_j - a_j^\dag a_j a_j^\dag a_{l} = - a_j^\dag a_l\\\nonumber
	&&[a_j^\dag a_{l}, a_l^\dag a_l ] = a_j^\dag a_{l} a_l^\dag a_l - a_l^\dag a_l a_j^\dag a_{l} = a_j^\dag a_l\\\nonumber
	&&[a_j^\dag a^\dag_{l}, a_j^\dag a_j ] = a_j^\dag a^\dag_{l} a_j^\dag a_j - a_j^\dag a_j a_j^\dag a^\dag_{l} = - a_j^\dag a_l^\dag\\\nonumber
	&&[a_j^\dag a^\dag_{l}, a_l^\dag a_l ] = a_j^\dag a^\dag_{l} a_l^\dag a_l - a_l^\dag a_l a_j^\dag a^\dag_{l} = - a_j^\dag a_l^\dag\\\nonumber
	&&[a_j a_l, a_j^\dag a_j ] = a_j a_{l} a_j^\dag a_j - a_j^\dag a_j a_j a_{l} =  a_j a_l\\\nonumber
	&&[a_j a_{l}, a_l^\dag a_l ] = a_j a_{l} a_l^\dag a_l - a_l^\dag a_l a_j a_{l} = a_j a_l
\end{eqnarray}
and
\begin{eqnarray}
	[a_j^\dag a_{l}, (1-2a_j^\dag a_j)(1-2a_l^\dag a_l) ] &=& (1-2a_j^\dag a_j)[a_j^\dag a_{l}, (1-2a_l^\dag a_l) ] + 	[a_j^\dag a_{l}, (1-2a_j^\dag a_j)](1-2a_l^\dag a_l) \\\nonumber
	&=& -2(1-2a_j^\dag a_j)a_j^\dag a_l + 2a_j^\dag a_l (1-2a_l^\dag a_l) \\\nonumber
	&=& -2 a_j^\dag a_l + 4 a_j^\dag a_l +  2 a_j^\dag a_l - 4 a_j^\dag a_l = 0 \\\nonumber
	[a_j a_{l}, (1-2a_j^\dag a_j)(1-2a_l^\dag a_l) ] &=& (1-2a_j^\dag a_j)[a_j a_{l}, (1-2a_l^\dag a_l) ] + 	[a_j a_{l}, (1-2a_j^\dag a_j)](1-2a_l^\dag a_l) \\\nonumber
	&=& -2(1-2a_j^\dag a_j)a_j a_l - 2a_j a_l (1-2a_l^\dag a_l) \\\nonumber
	&=& -2 a_j a_l -  2 a_j a_l + 4 a_j a_l =0
\end{eqnarray}
we have
\begin{eqnarray}
	[H, N] &=& 2\sum_{jl} (\Delta_{jl} a_j a_l - \Delta^\dag_{jl} a_j^\dag a_l^\dag ) \\\nonumber
	[H, P] &=& 0
\end{eqnarray}

Since $[H,P] = 0$, there are common eigenstates of $H$ and $P$, or that they can be simultaneously diagonalized by some unitary transformation. However, since $P$ is not a one-body operator, its unitary transformation in general cannot be written as multiplications of $2N\times 2N$ matrices as the BdG Hamiltonian. We therefore need to understand how the Bogoliubov transformation Eq.~\ref{eq:bogoliubov} transforms $P$. To this end we first write Eq.~\ref{eq:bogoliubov} into a block form
\begin{eqnarray}
	(\tilde{a}^\dag, \tilde{a}) = (a^\dag, a)\mathcal{U}
\end{eqnarray}
where $a \equiv (a_1, a_2,\dots, a_N)$ and so on. On the other hand, Eqs.~\ref{eq:ctoa} and \ref{eq:atoc} can be written as
\begin{eqnarray}
	(c_o, c_e) = (a^\dag, a) \begin{pmatrix}
		1 & i\\
		1 & -i
	\end{pmatrix},\; (a^\dag, a) = (c_o, c_e) \begin{pmatrix}
	\frac{1}{2} & \frac{1}{2}\\
	-\frac{i}{2} & \frac{i}{2}
\end{pmatrix}
\end{eqnarray}
where $c_o \equiv (c_1, c_3, \dots c_{2N-1})$ and $c_e\equiv(c_2, c_4, \dots, c_{2N})$. The above equations then lead to
\begin{eqnarray}
	(\tilde{c}_o, \tilde{c}_e) &=& (c_o, c_e) \begin{pmatrix}
		\frac{1}{2} & \frac{1}{2}\\
		-\frac{i}{2} & \frac{i}{2}
	\end{pmatrix} \mathcal{U} \begin{pmatrix}
		1 & i\\
		1 & -i
	\end{pmatrix} \\\nonumber
&=& (c_o, c_e) \begin{pmatrix}
	{\rm Re}(U_1 + U_2) & -{\rm Im}(U_1 - U_2) \\
	{\rm Im}(U_1 + U_2) &  {\rm Re}(U_1 - U_2)
\end{pmatrix}\\\nonumber
&=& (c_o, c_e) \mathcal{O}
\end{eqnarray}
or equivalently
\begin{eqnarray}
	\tilde{c}_{2j-1} &=& c_{2k-1} \mathcal{O}_{k,j} + c_{2k}\mathcal{O}_{N+k, j}\\\nonumber
	\tilde{c}_{2j} &=& c_{2k-1} \mathcal{O}_{k,N+j} + c_{2k}\mathcal{O}_{N+k, N+j}
\end{eqnarray}

Since all elements of $\mathcal{O}$ are real,
\begin{eqnarray}
	\mathcal{O}^T \mathcal{O} = \mathcal{O}^\dag \mathcal{O} = \begin{pmatrix}
		\frac{1}{2} & \frac{1}{2}\\
		-\frac{i}{2} & \frac{i}{2}
	\end{pmatrix} \mathcal{U}^\dag \mathcal{U} \begin{pmatrix}
		1 & i\\
		1 & -i
	\end{pmatrix} = \mathbb{I}
\end{eqnarray}
Namely, $\mathcal{O}$ is a real orthogonal matrix. As a result we have
\begin{eqnarray}
	1 = \det(\mathcal{O}^T\mathcal{O}) = (\det{\mathcal{O}})^2
\end{eqnarray}
which necessarily means $\det{\mathcal{O}} = \pm 1$. From linear algebra we know that an arbitrary special orthogonal matrix, i.e., $\det{\mathcal{O}} = +1$, can always be written as
\begin{eqnarray}
	\mathcal{O} = e^A
\end{eqnarray}
where $A = -A^T$ is a real skew-symmetric matrix. But no such general expressions exist for those $\mathcal{O}$ with $\det \mathcal{O} = -1$. For later convenience we reorganize the elements of $\mathcal{O}$ so that they are labeled in the same way as the Majorana operators. Namely
\begin{eqnarray}
	\mathcal{O}_{k,j}&\rightarrow&\mathcal{O}_{2k-1,2j-1}\\\nonumber
	\mathcal{O}_{N+k,j}&\rightarrow&\mathcal{O}_{2k,2j-1}\\\nonumber
	\mathcal{O}_{k,N+j}&\rightarrow&\mathcal{O}_{2k-1,2j}\\\nonumber
	\mathcal{O}_{N+k,N+j}&\rightarrow&\mathcal{O}_{2k,2j}
\end{eqnarray}
This does not affect the orthogonality of $\mathcal{O}$.

To see how $\mathcal{O}$ transforms $P$, we start from Eq.~\ref{eq:Pm} and note that it can be written as
\begin{eqnarray}\label{eq:Ppf}
	P = (-i)^N \prod_{j=1}^N (c_{2j-1}c_{2j}) = {\rm pf}(\mathcal{C})
\end{eqnarray}
where
\begin{eqnarray}
	\mathcal{C} \equiv\begin{pmatrix}
		C_1 & & \\
		& \ddots & \\
		& & C_N
	\end{pmatrix},\; C_j\equiv \begin{pmatrix}
0 & -i c_{2j-1} c_{2j} \\
i c_{2j-1} c_{2j} & 0
\end{pmatrix}
\end{eqnarray}
The pfaffian $\rm pf$ for an arbitrary skew-symmetric matrix is defined as
\begin{eqnarray}
	{\rm pf}(A) = \frac{1}{2^n n!} \sum_{\sigma \in S_{2n}} {\rm sgn}(\sigma) \prod_{i=1}^n a_{\sigma(2i-1), \sigma(2i)}
\end{eqnarray}
where $A$ is a $2n\times 2n$ skew-symmetric matrix, $S_{2n}$ is the permutation group of order $2n$. For skew-symmetric tridiagonal $A$ with $A_{2j-1, 2j} = - A_{2j, 2j-1} = b_j$ and all other elements zero, ${\rm pf}(A) = \prod_{j=1}^n b_j $. Eq.~\ref{eq:Ppf} is valid since all the $-i c_{2j -1} c_{2j}$ commute with one another and can be viewed as c-numbers.

Our goal is to convert the complicated transformation rule of $P$ under $\mathcal{O}$ to something that is more manageable. To this end we generalize the $\mathcal{C}$ matrix above to the following:
\begin{eqnarray}
	\mathcal{C}_{mn} = \begin{cases}
		0 & m=n\\
		-ic_m c_n & m\neq n
	\end{cases}
\end{eqnarray}
Apparently $\mathcal{C} = -\mathcal{C}^T$ and one can still calculate its pfaffian. To simplify the calculation we use the following equivalent definition of the pfaffian:
\begin{eqnarray}
	{\rm pf} (A) = \sum_{\alpha\in \Pi} A_\alpha
\end{eqnarray}
where $A_\alpha$ is
\begin{eqnarray}
	A_\alpha = {\rm sgn}(\pi_\alpha) a_{i_1,j_1} a_{i_2,j_2}\dots  a_{i_n,j_n}
\end{eqnarray}
and the permutation $\pi_\alpha$ and its set $\Pi$ are constructed in the following way: Consider a partition of $\{1,2,\dots,2n\}$ into unordered pairs and define $\alpha$ as such a partition
\begin{eqnarray}
	\alpha = \{(i_1,j_1),(i_2,j_2),\dots,(i_n,j_n)\}
\end{eqnarray}
so that there are $(2n)!/(2^n n!)$ such partitions. The permutation $\pi_\alpha$ is defined as
\begin{eqnarray}
	\pi_\alpha \equiv \begin{pmatrix}
		1 & 2 & 3 & 4 & \dots & 2n-1 & 2n\\
		i_1 & j_1 & i_2 & j_2 & \dots & i_n & j_n
	\end{pmatrix}
\end{eqnarray}
For our $\mathcal{C}$ this means the counterpart of $A_\alpha$ is
\begin{eqnarray}
	{\rm sgn}(\pi_\alpha) (-i)^N c_{m_1} c_{n_1}c_{m_2} c_{n_2}\dots c_{m_N} c_{n_N} = \prod_{j=1}^N (-ic_{2j-1}c_{2j})
\end{eqnarray}
since ${\rm sgn}(\pi_\alpha)$ is exactly compensated by the anticommutation relation of the Majorana fermions. We therefore have
\begin{eqnarray}
P =\frac{2^N N!}{(2N)!} {\rm pf}(\mathcal{C})
\end{eqnarray}

We next consider the transformation of $\mathcal{C}$ by $\mathcal{O}$:
\begin{eqnarray}
	\tilde{\mathcal{C}}_{mn} = -i \tilde{c}_m \tilde{c}_n = -i\sum_{i\neq j} \mathcal{O}_{mi}\mathcal{O}_{nj} c_i c_j = (\mathcal{O}^T \mathcal{C}\mathcal{O})_{mn}
\end{eqnarray}
which is nothing but the usual similarity transformation of the matrix $\mathcal{C}$. We therefore immediately get
\begin{eqnarray}
	\tilde{P} &=& \frac{2^N N!}{(2N)!} {\rm pf}(\tilde{\mathcal{C}}) = \frac{2^N N!}{(2N)!} {\rm pf}(\mathcal{O}^T \mathcal{C}\mathcal{O}) = \frac{2^N N!}{(2N)!} {\rm pf}(\mathcal{C}) \det(\mathcal{O})\\\nonumber
	&=& \det(\mathcal{O}) P
\end{eqnarray}
Therefore the Bogoliubov transformation $\mathcal{O}$ preserves the parity if $\det(\mathcal{O}) = +1$, and changes the parity if $\det(\mathcal{O}) = -1$.

Using the above transformation rule of $P$ under a general Bogoliubov transformation we can now understand the meaning of ground state parity in \cite{kitaevUnpairedMajoranaFermions2001}. Start from an arbitrary state that is an eigenstate of $P$ with even parity, we have
\begin{eqnarray}
	P |\psi\rangle = |\psi\rangle
\end{eqnarray}
Under a Bogoliubov transformation, the state itself is unchanged, but $P\rightarrow \tilde{P}$, since the meaning of particles is different. We then have
\begin{eqnarray}
	\tilde{P} |\psi\rangle = \det(\mathcal{O})P |\psi\rangle = \det(\mathcal{O})|\psi\rangle
\end{eqnarray}
Namely, because the Bogoliubov transformation redefines particles and hence the parity operator, an even-parity state can become an odd-parity state in the new definition of the parity operator. Therefore for a given BdG Hamiltonian, the meaning of its ground state parity must be relative, and we need to choose a reference in order to discuss the parity. Such a reference is the ground state of the ``canonical form'' of the BdG Hamiltonian:
\begin{eqnarray}
	H_{\rm canonical} &=& \sum_{m} \epsilon_m (\tilde{a}_m^\dag \tilde{a}_m - \frac{1}{2}) = \frac{i}{2}\sum_{m}\tilde{c}_{2m-1} \tilde{c}_{2m} \\\nonumber
	&\equiv& \frac{i}{2}\sum_{m}\epsilon_m b'_m b''_{m},\;\; \epsilon_m\geq 0
\end{eqnarray}
where the crucial requirement is that all the eigenenergies are non-negative. For a given BdG Hamiltonian such a canonical form is uniquely fixed, and we can use its ground state as a reference for the parity and the parity operator. The ground state of $H_{\rm canonical} $ is defined by
\begin{eqnarray}
	\tilde{a}_m|\Omega_{\rm canonical}\rangle = 0\;\; \forall m\in[1,N].
\end{eqnarray}
and the ``reference'' or canonical parity operator is
\begin{eqnarray}
	P_{\rm canonical} \equiv \prod_{m=1}^N (-ib'_m b''_m).
\end{eqnarray}
Since there are no $\tilde{a}$ particles in $|\Omega_{\rm canonical}\rangle$, we must have
\begin{eqnarray}
	P_{\rm canonical} |\Omega_{\rm canonical}\rangle = |\Omega_{\rm canonical}\rangle
\end{eqnarray}
Namely $|\Omega_{\rm canonical}\rangle $ has even parity. We can then ask the following question: What is the parity of $|\Omega_{\rm canonical}\rangle$ in the sense of particles in the original BdG Hamiltonian, i.e., $a_j$? This requires us to evaluate
\begin{eqnarray}
	P_{\rm BdG} |\Omega_{\rm canonical}\rangle &\equiv& \prod_{j} (-ic_{2j-1}c_{2j})  |\Omega_{\rm canonical}\rangle = \det(\mathcal{O}) 	P_{\rm canonical} |\Omega_{\rm canonical}\rangle \\\nonumber
	&=& \det(\mathcal{O}) |\Omega_{\rm canonical}\rangle
\end{eqnarray}
Namely, the parity is equal to the determinant of the orthogonal transformation that transforms $c$ to $b'$ and $b''$. More precisely,
\begin{eqnarray}
	\begin{pmatrix}
		b'_1 \\
		b''_1 \\
		\vdots \\
		b'_N \\
		b''_N
	\end{pmatrix} = \mathcal{O} \begin{pmatrix}
	c_1 \\
	c_2 \\
	\vdots \\
	c_{2N-1} \\
	c_{2N}
\end{pmatrix}
\end{eqnarray}
and
\begin{eqnarray}\label{eq:AtransO}
	\mathcal{O} A \mathcal{O}^T = \begin{pmatrix}
		0 & \epsilon_1 & & & \\
		-\epsilon_1 & 0 & & & \\
		& & \ddots & & \\
		& & & 0 & \epsilon_N \\
		& & & -\epsilon_N & 0
	\end{pmatrix}
\end{eqnarray}
where $A$ is introduced in Eq.~\ref{eq:BdGMrep}, and our $\mathcal{O}$ is the matrix $W$ in \cite{kitaevUnpairedMajoranaFermions2001}. Eq.~\ref{eq:AtransO} therefore leads to a convenient formula for calculating $\det(\mathcal{O})$:
\begin{eqnarray}
	{\rm pf}(	\mathcal{O} A \mathcal{O}^T) = 	\det(\mathcal{O}){\rm pf}(A) = {\rm pf}\begin{pmatrix}
		0 & \epsilon_1 & & & \\
		-\epsilon_1 & 0 & & & \\
		& & \ddots & & \\
		& & & 0 & \epsilon_N \\
		& & & -\epsilon_N & 0
	\end{pmatrix} = \prod_m \epsilon_m \geq 0
\end{eqnarray}
Therefore
\begin{eqnarray}
	\det(\mathcal{O}) = \left(\prod_m \epsilon_m\right) [{\rm pf}(A)]^{-1}
\end{eqnarray}
Since $\det(\mathcal{O}) = \pm 1$ we only need the signs of the two quantities on the right hand side of the above equation. If none of the $\epsilon_m$ vanishes, $ \left(\prod_m \epsilon_m\right) > 0$, we finally arrive at
\begin{eqnarray}
	\det(\mathcal{O}) = {\rm sgn}[{\rm pf}(A)]
\end{eqnarray}
Namely,
\begin{eqnarray}
	P_{\rm BdG} |\Omega_{\rm canonical}\rangle &=& \det(\mathcal{O}) |\Omega_{\rm canonical}\rangle \\\nonumber
	&=& {\rm sgn}[{\rm pf}(A)]  |\Omega_{\rm canonical}\rangle
\end{eqnarray}

\section{Gauge potential and gauge invariance}
In this section we address the question of how to understand the Peierls substitution in BdG Hamiltonian.

Although the superconductivity order parameter appears to break the U(1) gauge symmetry, all physical observables are still gauge invariant. More explicitly, consider a general tight-binding BdG Hamiltonian
\begin{eqnarray}\label{eq:HBdGgeneral}
	H = \sum_{ij,\alpha\beta} \left(t_{ij}^{\alpha\beta}c_{i\alpha}^\dag c_{j\beta} + \Delta_{ij,\alpha\beta} c_{i\alpha}c_{j\beta} - \frac{\mu}{2}c_{i\alpha}^\dag c_{i\alpha} + {\rm h.c.} \right) \equiv \frac{1}{2}C^\dag h C
\end{eqnarray}
where $i,j$ label position, $\alpha,\beta$ label any internal degrees of freedom, and $C = (\{c_{i\alpha}\},\{c^\dag_{i\alpha}\})^T$. $H$ has the eigensolutions
\begin{eqnarray}
	H |\psi_n\rangle &=& \epsilon_n |\psi_n\rangle \\\nonumber
	|\psi_n\rangle &=& d^\dag_{\psi_n} |\Omega\rangle = \sum_{i\alpha\sigma}c^{\sigma}_{i\alpha} | \Omega\rangle U_{i\alpha\sigma,n}
\end{eqnarray}
where $|\Omega\rangle$ is the BCS ground state, $\sigma=\pm$ distinguishes the creation (particle) and annihilation (creation for hole) operators, and $U$ is a Bogoliubov transformation matrix which is unitary for fermions. Substituting $|\psi\rangle$ into the eigenequation leads to
\begin{eqnarray}
	U^\dag h U = {\rm Diag}[\{\epsilon_n\}]
\end{eqnarray}

where the pairing potential satisfies the gap equation
\begin{eqnarray}
	\Delta_{ij,\alpha\beta} &=& Z^{-1}{\rm Tr}[V_{j\beta,i\alpha}c_{j\beta}^\dag c^\dag_{i\alpha} e^{-\frac{1}{k_B T} H}] \\\nonumber
	&=& \sum_n f(\epsilon_n) (U^\dag {\mathbb{V}} U)_{nn}
\end{eqnarray}
where $\mathbb{V}$ is a matrix with the only nonzero element being $\mathbb{V}_{j\beta+,i\alpha-} = V_{j\beta, i\alpha}$, $f$ is the Fermi-Dirac distribution function.

We now show that physical observables are gauge invariant. A gauge transformation corresponds to
\begin{eqnarray}
	\mathbf A\rightarrow \mathbf A' = \mathbf A + \nabla \chi
\end{eqnarray}
where $\mathbf A$ is the gauge potential. $\mathbf A$ enters the tight-binding Hamiltonian implicitly through the Peierls substitution:
\begin{eqnarray}\label{eq:peierls}
	c_{i\alpha}^\dag \rightarrow \tilde{c}^\dag_{i\alpha} = e^{-\frac{ie}{\hbar} \int_0^{\mathbf r_i} \mathbf A\cdot d \mathbf l} c_{i\alpha}^\dag
\end{eqnarray}
and we can understand Eq.~\eqref{eq:HBdGgeneral} as that written for certain $\mathbf A$ already absorbed into the definitions of $t$ and $\Delta$. The gauge transformation leads to
\begin{eqnarray}
	c^\dag_{i\alpha} \rightarrow c^\dag_{i\alpha} e^{-\frac{ie}{\hbar}\chi_i}
\end{eqnarray}
The Hamiltonian therefore becomes
\begin{eqnarray}
	H\rightarrow H'&=& \sum_{ij,\alpha\beta}\left[ t_{ij}^{\alpha\beta} e^{-\frac{ie}{\hbar}(\chi_i - \chi_j)} c_{i\alpha}^\dag c_{j\beta} + \Delta_{ij,\alpha\beta} e^{\frac{ie}{\hbar}(\chi_i + \chi_j)}c_{i\alpha}c_{j\beta}  - \frac{\mu}{2}c_{i\alpha}^\dag c_{i\alpha} + {\rm h.c.} \right] \\\nonumber
	&=&\frac{1}{2}C^\dag U_\chi h U_\chi^\dag C
\end{eqnarray}
where
\begin{eqnarray}
	U_\chi = {\rm Diag}[\{e^{-\frac{ie}{\hbar}\chi_i}\},\{e^{\frac{ie}{\hbar}\chi_i}\}]
\end{eqnarray}
As a result, the BdG eigenvalues as well as all other physical observables represented in terms of Bogoliubov quasiparticles are invariant under the gauge transformation.

The above derivation includes, however, an assumption. Namely the pairing potential $\Delta_{ij,\alpha\beta}$ stays unchanged. This is indeed the case, since
\begin{eqnarray}
	\Delta'_{ij,\alpha\beta} &=& Z^{'-1}{\rm Tr}[V_{j\beta,i\alpha} c^\dag_{j\beta}c^{\dag}_{i\alpha} e^{-\frac{ie}{\hbar}(\chi_i + \chi_j)} e^{-\frac{1}{k_B T}H'}] \\\nonumber
	&=& \sum_{n}f(\epsilon_n) (U^\dag U^\dag_\chi U_\chi \mathbb{V} U_\chi^\dag U_\chi U) \\\nonumber
	&=&\Delta_{ij,\alpha\beta}
\end{eqnarray}

%\section{Analytic solution of the Kitaev triangle}
%In this section we present some analytic results related to the Kitaev triangle.
%
%We start from the 1D Kitaev chain Hamiltonian with complex nearest-neighbor hopping $-te^{i\phi}$ and $p$-wave pairing $\Delta e^{i\theta}$ in the Kitaev limit ($t=\Delta > 0, \mu = 0$):
%\begin{eqnarray}
%	H = \sum_{n}\left(- te^{i\phi}c_n^\dag c_{n+1} + \Delta e^{i\theta}c_nc_{n+1} + {\rm h.c.}\right)
%\end{eqnarray}
%In the Majorana fermion basis $a_n = c_n + c_n^\dag$, $b_n = -i(c_n - c_n^\dag)$ the Hamiltonian becomes
%\begin{eqnarray}
%H = -\frac{it}{2} \sum_n \left[(S_\phi - S_\theta) a_n a_{n+1} + (S_\phi + S_\theta)b_n b_{n+1} + (C_\phi - C_\theta) a_n b_{n+1} - (C_\phi + C_\theta)b_na_{n+1}\right]
%\end{eqnarray}
%where $S_\phi\equiv \sin\phi$, $C_\phi\equiv \cos\phi$, etc. Therefore, when $\phi = \theta$, $a_n$ becomes decoupled from $a_{n+1}$ and $b_{n+1}$, and $a_1$ drops out from the Hamiltonian. Similarly, when $\phi = \theta + \pi$, $b_1$ becomes isolated. To find the other MZM, we note that when $\phi = \theta$, terms involving $a_{N}$ and $b_N$ in the Hamiltonian are
%\begin{eqnarray}
%	H_N = -itb_{N-1}(S_\phi b_{N} - C_\phi a_N).
%\end{eqnarray}
%Considering the unitary transformation
%\begin{eqnarray}
%	\begin{pmatrix}
%		a_N' \\
%		b_N'
%	\end{pmatrix} \equiv\begin{pmatrix}
%	C_\phi & - S_\phi\\
%	S_\phi & C_\phi
%\end{pmatrix}\begin{pmatrix}
%a_N\\
%b_N
%\end{pmatrix}
%\end{eqnarray}
%we have
%\begin{eqnarray}
%	H_N = itb_{N-1} a'_N
%\end{eqnarray}
%Therefore the other MZM is $b'_N = S_\phi a_N + C_\phi b_N$. Similarly, when $\phi = \theta + \pi$ the other MZM is $a'_{N} \equiv C_\phi a_N - S_\phi b_N$.
%
%We now consider the three edges of the Kitaev triangle separately. The MZM due to each edge are respectively
%\begin{eqnarray}
%	1-2:&& a_1,\,b_2\\\nonumber
%	2-3:&& b_2,\,\frac{1}{2}a_3 + \frac{\sqrt{3}}{2}b_3\\\nonumber
%	3-1:&& a_1,\, \frac{\sqrt{3}}{2}a_3 + \frac{1}{2} b_3
%\end{eqnarray}
%One can therefore see that the two MZM at site 3 are not compatible with each other. To get the non-MZM eigenstates, we write down the remaining terms of the Hamiltonian in the Majorana basis
%\begin{eqnarray}
%	H &=& -\frac{it}{2}\left( -2b_1 a_2 - \sqrt{3} a_2 a_3 + a_2 b_3 + \sqrt{3} b_1 b_3 - b_1 a_3   \right)\\\nonumber
%	&=&\frac{1}{2}(b_1, a_2, a_3, b_3)h\begin{pmatrix}
%	b_1\\
%	a_2\\
%	a_3\\
%	b_3
%\end{pmatrix}\\\nonumber
%h&\equiv&-it\begin{pmatrix}
%	0 & -1 & -\frac{1}{2} & \frac{\sqrt{3}}{2} \\
%	1 & 0 & -\frac{\sqrt{3}}{2} & \frac{1}{2}\\
%	\frac{1}{2} & \frac{\sqrt{3}}{2} & 0 & 0 \\
%	-\frac{\sqrt{3}}{2} & -\frac{1}{2} & 0 & 0
%\end{pmatrix} = t\left( -\frac{1}{2}\sigma_0 \tau_y - \frac{1}{2}\sigma_z\tau_y -\frac{1}{2}\sigma_y \tau_z + \frac{\sqrt{3}}{2} \sigma_x \tau_y \right)
%\end{eqnarray}
%$h$ has the following symmetry:
%\begin{eqnarray}
%	O = \left(\frac{\sqrt{3}}{2}\sigma_x - \frac{1}{2}\sigma_z\right) \tau_y
%\end{eqnarray}
%We therefore rotate the Hamiltonian so that $O$ becomes diagonal using the following unitary operator
%\begin{eqnarray}
%	U = e^{-\frac{i\pi}{3}\sigma_y}\otimes e^{i\frac{\pi}{4}\tau_x}
%\end{eqnarray}
%which leads to
%\begin{eqnarray}
%	U^\dag O U &=& {\rm Diag}(1,-1,-1,1)
%\end{eqnarray}
%$U$ therefore block-diagonalizes $h$ as
%\begin{eqnarray}
%	U^\dag h U = 	\frac{t}{2}\begin{pmatrix}
%		1 &  &  & -1 \\
%		& -1 & 1 & \\
%		& 1 & -3 & \\
%		-1 & & & 3
%	\end{pmatrix}
%\end{eqnarray}
%which can then be diagonalized by
%\begin{eqnarray}
%	V = \begin{pmatrix}
%		\frac{1+ \sqrt{2}}{\sqrt{4+2\sqrt{2}}} & 0 & \frac{1 - \sqrt{2}}{\sqrt{4-2\sqrt{2}}} & 0 \\
%		0 & \frac{1 + \sqrt{2}}{\sqrt{4 + 2\sqrt{2}}} & 0 & \frac{1-\sqrt{2}}{\sqrt{4 - 2\sqrt{2}}} \\
%		0 & \frac{1}{\sqrt{4+ 2\sqrt{2}}} & 0 & \frac{1}{\sqrt{4 - 2\sqrt{2}}} \\
%		\frac{1}{\sqrt{4+2\sqrt{2}}} & 0 & \frac{1}{\sqrt{4-2\sqrt{2}}} & 0
%	\end{pmatrix}
%\end{eqnarray}
%as
%\begin{eqnarray}
%	V^\dag U^\dag h U V =t\times {\rm Diag}\left( 1 -\frac{\sqrt{2}}{2},  -1 +\frac{\sqrt{2}}{2},  1 +\frac{\sqrt{2}}{2},  -1 -\frac{\sqrt{2}}{2} \right)
%\end{eqnarray}
%We therefore have the two lowest excited states
%\begin{eqnarray}
%	\psi_{+1} &=& (b_1,a_2,a_3,b_3) U \begin{pmatrix}
%	\frac{1+ \sqrt{2}}{\sqrt{4+2\sqrt{2}}} \\
%0  \\
%0 \\
%\frac{1}{\sqrt{4+2\sqrt{2}}}
%	\end{pmatrix} = (b_1,a_2,a_3,b_3)\times \frac{1}{4\sqrt{2+\sqrt{2}}} \begin{pmatrix}
%	1+\sqrt{2}-\sqrt{3} i \\
%	(1+\sqrt{2})i-\sqrt{3}  \\
%	i+ \sqrt{3} + \sqrt{6} \\
%	1 + (\sqrt{3} + \sqrt{6})i
%\end{pmatrix} \\\nonumber
%	\psi_{-1} &=& (b_1,a_2,a_3,b_3) U \begin{pmatrix}
%		0\\
%	\frac{1+ \sqrt{2}}{\sqrt{4+2\sqrt{2}}} \\
%	\frac{1}{\sqrt{4+2\sqrt{2}}} \\
%	0
%\end{pmatrix} = (b_1,a_2,a_3,b_3)\times \frac{1}{4\sqrt{2+\sqrt{2}}} \begin{pmatrix}
%	(1+\sqrt{2})i-\sqrt{3} \\
%	1+\sqrt{2}-\sqrt{3} i \\
%	1+ (\sqrt{3} + \sqrt{6} )i\\
%    i + \sqrt{3} + \sqrt{6}
%\end{pmatrix}
%\end{eqnarray}
%The first excited states can therefore be understood as a hybridization between the ``bulk" states of the 1-2 bond and the fermion on site 3.
%
%We next consider the braiding process and particularize to the $\boldsymbol{\phi}_1\rightarrow \boldsymbol{\phi}_2$ step. The Hamiltonian in the fermion basis becomes
%\begin{eqnarray}
%H &=& - e^{ix}c_1^\dag c_2 + c_1 c_2 + e^{-ix}c_1 c_2^\dag - c_1^\dag c_2^\dag \\\nonumber
%&+&  - e^{-\frac{\pi}{3}i} c_2^\dag c_3 + e^{\frac{2\pi}{3}i} c_2 c_3 + e^{\frac{\pi}{3}i}c_2 c_3^\dag - e^{-\frac{2\pi}{3}i} c_2^\dag c_3^\dag \\\nonumber
%&+&  e^{\left(-\frac{\pi}{3}-x\right)i} c_1 c_3^\dag  - e^{-\frac{2\pi}{3}i} c_1 c_3  - e^{\left(\frac{\pi}{3}+x\right)i} c_1^\dag c_3  + e^{\frac{2\pi}{3}i}  c_1^\dag c_3^\dag
%\end{eqnarray}
%where we have temporarily omitted the energy unit $t$. We then have
%\begin{eqnarray}
%	[c_1^\dag, H] &=& c_2 + e^{-ix} c_2^\dag + e^{\left(-\frac{\pi}{3}-x\right)i} c_3^\dag - e^{-\frac{2\pi}{3}i} c_3 \\\nonumber
%	[c_1, H] &=& -e^{ix}c_2 - c_2^\dag - e^{\left(\frac{\pi}{3}+x\right)i} c_3 + e^{\frac{2\pi}{3}i} c_3^\dag\\\nonumber
%	&=& -e^{ix} \left[ c_2 + e^{-i x}c_2^\dag - e^{-\frac{2\pi}{3}i} c_3 +  e^{\left(-\frac{\pi}{3}-x\right)i} c_3^\dag \right]
%\end{eqnarray}
%Therefore
%\begin{eqnarray}
%	[e^{\frac{ix}{2}}c_1^\dag + e^{-\frac{ix}{2}} c_1, H] = 0
%\end{eqnarray}
%Namely we have an MZM:
%\begin{eqnarray}
%	\tilde{a}_1 \equiv e^{\frac{ix}{2}}c_1^\dag + e^{-\frac{ix}{2}} c_1 = \frac{1}{2}e^{\frac{ix}{2}} (a_1 - ib_1) + \frac{1}{2}e^{-\frac{ix}{2}} (a_1 + ib_1) = C_{\frac{x}{2}} a_1 + S_{\frac{x}{2}} b_1
%\end{eqnarray}
%To find the other MZM, we calculate the commutators between the other fermion operators with the Hamiltonian:
%\begin{eqnarray}
%	[c_2^\dag, H] &=& e^{ix} c_1^\dag - c_1 - e^{-\frac{i\pi}{3}} c_3 + e^{\frac{i\pi}{3}} c_3^\dag \\\nonumber
%	[c_2, H] &=& -e^{-ix} c_1 + c_1^\dag + e^{\frac{i\pi}{3}} c_3^\dag - e^{-\frac{i\pi}{3}} c_3\\\nonumber
%	[c_3^\dag, H] &=& e^{-\frac{i\pi}{3}}c_2^\dag + e^{-\frac{i\pi}{3}} c_2 - e^{\frac{i\pi}{3}} c_1 + e^{i\left(\frac{\pi}{3} + x \right)}c_1^\dag \\\nonumber
%	[c_3, H]	&=& -e^{\frac{i\pi}{3}}c_2 - e^{\frac{i\pi}{3}} c_2^\dag + e^{-\frac{i\pi}{3}} c_1^\dag - e^{-i\left(\frac{\pi}{3} + x \right)}c_1
%\end{eqnarray}
%Therefore
%\begin{eqnarray}
%	[c_2 - c_2^\dag, H] &=& (1-e^{-ix})c_1 + (1-e^{ix})c_1^\dag\\\nonumber
%	[e^{\frac{i\pi}{6}}\left(e^{i\frac{2\pi}{3}} c_3^\dag + c_3\right), H] & = &  e^{\frac{i\pi}{6}} (1-e^{-i\left(\frac{\pi}{3} + x\right)})c_1 + e^{-\frac{i\pi}{6}}( 1- e^{i\left(\frac{\pi}{3} + x\right)})c_1^\dag
%\end{eqnarray}
%However
%\begin{eqnarray}
%-\frac{1-e^{-ix}}{e^{\frac{i\pi}{6}} (1-e^{-i\left(\frac{\pi}{3} + x\right)})} = - \frac{2-2\cos x}{e^{\frac{i\pi}{6}} (1-e^{-i\left(\frac{\pi}{3} + x\right)})(1-e^{ix})} = \frac{1-\cos x}{\cos\left( x+ \frac{\pi}{6}\right)-\frac{\sqrt{3}}{2}}
%\end{eqnarray}
%Thus the following is the other MZM:
%\begin{eqnarray}
%	\tilde{b}_{23} &\equiv& -iN\left(\left[\cos\left( x+ \frac{\pi}{6}\right)-\frac{\sqrt{3}}{2}\right](c_2 - c_2^\dag) + (1-\cos x) \left( e^{\frac{i\pi}{6}} c_3 -e^{-\frac{i\pi}{6}} c_3^\dag\right)\right)\\\nonumber
%	&=& N\left(\left[\cos\left( x+ \frac{\pi}{6}\right)-\frac{\sqrt{3}}{2}\right]b_2 + (1-\cos x)\left( \frac{1}{2} a_3 + \frac{\sqrt{3}}{2} b_3\right)\right)
%\end{eqnarray}
%where $N$ is a normalization factor. When $x=0$ only the first term survives since
%\begin{eqnarray}
%\lim_{x\rightarrow 0} \frac{1-\cos x}{\cos\left( x+ \frac{\pi}{6}\right)-\frac{\sqrt{3}}{2}} = 0
%\end{eqnarray}
%while when $x = -\frac{\pi}{3}$ only the second term survives.
%
%
%
%
%
%%\begin{thebibliography}{99}
%%\bibitem{kitaev} A. Yu Kitaev, Physics-Uspekhi, \textbf{44}, 131 (2001).
%%
%%\end{thebibliography}
