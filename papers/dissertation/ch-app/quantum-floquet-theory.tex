\section{Quantum Floquet theory} \label{app:quantum-floquet-theory}
Floquet theory describes the stationary state of a time-periodic Hamiltonian, much like Bloch theory describes the state of a spatial-periodic or translation symmetric Hamiltonian.
We state the time-periodic Hamiltonian as

\begin{equation}
  H(t) = H(t+T)
\end{equation}

and the Floquet states in general are $|\psi(t)\rangle$.
The time-dependent Schrodinger equation solution with the Floquet states are

\begin{equation}
  i\hbar d_t |\psi(t)\rangle = H |\psi(t)\rangle,
\end{equation}
with $d_t = \tfrac{d}{dt}$.
We see the state solutions have the form

\begin{equation}
  |\psi_n (t) \rangle = |u_n (t)\rangle e^{-i\epsilon_n t / \hbar}
\end{equation}
with $|u_n (t)\rangle = |u_n (t+T)\rangle$ and $\epsilon_n$ is the quasienergy mode.
This lets us write

\begin{equation}
  |u_n (t) \rangle = e^{-im\omega t} |u_{nm} (t) \rangle,
\end{equation}
where $\omega = 2\pi/T$.
The state solution then becomes

\begin{equation}
  |\psi_n (t) \rangle = e^{-i\epsilon_{nm} t / \hbar} |u_{nm} (t) \rangle,
\end{equation}
where we have defined $\epsilon_{nm} = \epsilon_n + m \hbar \omega$.
Looking back at the Schrodinger equation we can write a new quasienergy operator $Q = H - id_t$ and act it on the state solution to give

\begin{equation}
  Q|u_{nm} (t)\rangle = \epsilon_{nm} |u_{nm} (t) \rangle
\end{equation}

\begin{equation}
  H = H^{F(0)} + H^{F(1)} + H^{F(2)} + H^{F(3)} + \dots
\end{equation}
where each term is a Floquet Hamiltonian

\begin{align}
  H^{F(0)} &= 0 \\
  H^{F(1)} &= H_0 \\
  H^{F(2)} &= \sum_{m\neq 0} \dfrac{[H_m, H_{-m}]}{m\hbar\omega} \\
  H^{F(3)} &= \sum_{m\neq 0} \left( \dfrac{[H_{-m} , [H_0, H_m]]}{2(m\hbar\omega)^2} + \sum_{m'\neq 0, m} \dfrac{[H_{-m'}, [H_{m'-m}, H_m]]}{3mm'(\hbar\omega)^2} \right)
\end{align}
We go to third order in general because our 2DEG is non-zero in second order and Dirac is non-zero in third order approximation.
The $H_m$ terms are Fourier time-domain transforms of the time dependent Hamiltonian

\begin{equation}
  H_n = \dfrac{1}{T} \int_0^T H(t) e^{-in\omega t} dt = H_{-n}^{\dagger}
\end{equation}

Some of the commutators are related by transpose or Hermitian conjugate.
As an example in $H^{F(2)}$ the transpose reduces the sum down by

\begin{align}
  \dfrac{[H_m,H_{-m}]}{m\hbar \omega} + \dfrac{[H_{-m}, H_m]}{-m\hbar\omega} &= \dfrac{[H_m,H_{-m}]}{m\hbar \omega} + \dfrac{[H_m, H_{-m}]}{m\hbar\omega} \\
  &= 2\dfrac{[H_m,H_{-m}]}{m\hbar \omega}
\end{align}
Additionally, an example in $H^{F(3)}$ the Hermitian conjugate reduces the sum down by

\begin{align}
  [H_{-m'},[H_{m'-m},H_m]]^{\dagger} &= [[H_{m'-m},H_m]^{\dagger}, H_{-m'}^{\dagger}] \nonumber \\
  &= [ [H_m^{\dagger}, H_{m'-m}^{\dagger}], H_{m'}] \nonumber \\
  &= [ [H_{-m}, H_{m-m'}], H_{m'}] \nonumber \\
  &= [H_{m'}, [H_{m-m'}, H_{-m}]]
\end{align}
or in general the following identity

\begin{equation}
  [A,[B,C]]^{\dagger} = [A^{\dagger}, [B^{\dagger}, C^{\dagger}]].
\end{equation}
With the symmetry in modes we have the following simplification

\begin{equation}
  \dfrac{[H_{-m'},[H_{m'-m},H_m]]}{mm'(\hbar\omega)^2} + \dfrac{[H_{m'},[H_{m-m'},H_{-m}]]}{(-m)(-m')(\hbar\omega)^2} = \dfrac{[H_{-m'},[H_{m'-m},H_m]] + h.c.}{mm'(\hbar\omega)^2}
\end{equation}

We can reduce the second and third perturbation summation terms to
\begin{align}
  H^{F(2)} &= \sum_{m> 0} \dfrac{2[H_m, H_{-m}]}{m\hbar\omega} \\
  H^{F(3)} &= \sum_{m> 0} \left( \dfrac{[H_{-m} , [H_0, H_m]] + h.c.}{2(m\hbar\omega)^2} + \sum_{m'\neq m} \dfrac{[H_{-m'}, [H_{m'-m}, H_m]] + h.c.}{3mm'(\hbar\omega)^2} \right)
\end{align}
