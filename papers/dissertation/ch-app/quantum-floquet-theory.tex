\section{Quantum Floquet theory}

\begin{equation}
  H = H^{F(0)} + H^{F(1)} + H^{F(2)} + H^{F(3)} + \dots
\end{equation}
where each term is a Floquet Hamiltonian

\begin{align}
  H^{F(0)} &= 0 \\
  H^{F(1)} &= H_0 \\
  H^{F(2)} &= \sum_{m\neq 0} \dfrac{[H_m, H_{-m}]}{m\hbar\omega} \\
  H^{F(3)} &= \sum_{m\neq 0} \left( \dfrac{[H_{-m} , [H_0, H_m]]}{2(m\hbar\omega)^2} + \sum_{m'\neq 0, m} \dfrac{[H_{-m'}, [H_{m'-m}, H_m]]}{3mm'(\hbar\omega)^2} \right)
\end{align}
The $H_m$ terms are Fourier time-domain transforms of the time dependent Hamiltonian

\begin{equation}
  H_n = \dfrac{1}{T} \int_0^T H(t) e^{-in\omega t} dt = H_{-n}^{\dagger}
\end{equation}

\subsection{Non-uniform circularly polarized light on 2DEG}

We start with the Schrodinger equation in 2D with a vector potential field

\begin{equation}
  H(t) = \dfrac{1}{2m} \left[ (p_x - qA_x(t))^2 + (p_y - qA_y(t))^2 \right]
\end{equation}
where $\vec{A}(t) = \langle -V_1\sin{\omega t} , V_2\cos{\omega t} \rangle$, $V_1 = E/\omega$ and $V_2 = V_1 \cos{Kx}$.
Which is made up of two electromagnetic wave sources.
The time dependent Hamiltonian becomes

\begin{equation}
  H(t) = \dfrac{1}{2m} \left[ p_x^2 + p_y^2 + 2qV_1\sin{\omega t} p_x - 2qV_2\cos{\omega t} p_y + \dfrac{q^2(V_1^2+V_2^2)}{2} -\dfrac{q^2(V_1^2-V_2^2)}{2}\cos{2\omega t} \right].
\end{equation}

Notice our Hamiltonian has modes $m = [-2,2]$, due to orthogonality we need only compute three integrals.

\begin{align}
  H_0 &= \dfrac{1}{2m} \left(p_x^2 + p_y^2 + \dfrac{q^2}{2}(V_1^2+V_2^2) \right) \\
  H_{\pm1} &= -\dfrac{1}{2m} \left( qV_2 p_y \pm iqV_1p_x \right) \\
  H_{\pm2} &= -\dfrac{1}{2m} \left( \dfrac{q^2}{4} (V_1^2 - V_2^2) \right)
\end{align}
We can now begin writing out all the commutator relations seen above.
Some of the commutators are related by transpose or Hermitian conjugate.
As an example in $H^{F(2)}$ the transpose reduces the sum down by

\begin{align}
  \dfrac{[H_m,H_{-m}]}{m\hbar \omega} + \dfrac{[H_{-m}, H_m]}{-m\hbar\omega} &= \dfrac{[H_m,H_{-m}]}{m\hbar \omega} + \dfrac{[H_m, H_{-m}]}{m\hbar\omega} \\
  &= 2\dfrac{[H_m,H_{-m}]}{m\hbar \omega}
\end{align}
Alternatively, an example in $H^{F(3)}$ the Hermitian conjugate reduces the sum down by

\begin{align}
  [H_{-m'},[H_{m'-m},H_m]]^{\dagger} &= [[H_{m'-m},H_m]^{\dagger}, H_{-m'}^{\dagger}] \nonumber \\
  &= [ [H_m^{\dagger}, H_{m'-m}^{\dagger}], H_{m'}] \nonumber \\
  &= [ [H_{-m}, H_{m-m'}], H_{m'}] \nonumber \\
  &= [H_{m'}, [H_{m-m'}, H_{-m}]]
\end{align}
or in general the following identity

\begin{equation}
  [A,[B,C]]^{\dagger} = [A^{\dagger}, [B^{\dagger}, C^{\dagger}]].
\end{equation}
With the symmetry in modes we have the following simplification

\begin{equation}
  [H_{-m'},[H_{m'-m},H_m]] + [H_{m'},[H_{m-m'},H_{-m}]] = [H_{-m'},[H_{m'-m},H_m]] + h.c.
\end{equation}

We now focus on the $H^{F(2)}$ term which looks like

\begin{equation}
  H^{F(2)} = \dfrac{2}{\hbar\omega} \left( [H_1,H_{-1}] + \tfrac{1}{2} [H_2,H_{-2}] \right)
\end{equation}

\begin{align*}
  [H_1,H_{-1}] &= \dfrac{q^2}{4m^2} [qV_2 p_y + iV_1 p_x, qV_2 p_y - iV_1 p_x] \\
  &= \dfrac{iq^2 V_1 p_y}{4m^2} ([p_x,V_2] - [V_2,p_x]) \\
  &= -\dfrac{iq^2 V_1 p_y}{2m^2} [V_2,p_x] \\
  &= \dfrac{\hbar q^2 V_1 p_y}{2m^2} \partial_x V_2
\end{align*}

\begin{equation}
  [H_2,H_{-2}] &= \dfrac{q^4}{64m^2} [V_1^2 - V_2^2, V_1^2-V_2^2] = 0
\end{equation}

\begin{equation}
  H^{F(2)} = \dfrac{q^2 V_1 p_y}{m^2 \omega} \partial_x V_2
\end{equation}

For the 3rd order terms there are several combinations that end up being zero becuse they require $H_{|m|\geq3}$ which are zero from our earlier Fourier transforms.
We need compute the following commutations:

\begin{align}
  [H_1,[H_0,H_{-1}]] \\
  [H_2,[H_0,H_{-2}]] \\
  [H_1,[H_1, H_{-2}]] \\
  [H_2,[H_{-1}, H_{-1}]] \\
  [H_{-1},[H_2, H_{-1}]].
\end{align}

\begin{align}
  [H_1,[H_0,H_{-1}]] &= \dfrac{q^2}{8m^3} [V_2 p_y + i V_1 p_x, [p_x^2 + p_y^2 + \tfrac{q^2}{2} (V_1^2+V_2^2), V_2 p_y - i V_1 p_x]] \nonumber \\
  &= \dfrac{q^2}{8m^3} [V_2 p_y + i V_1 p_x, p_y[p_x^2,V_2] -i\tfrac{q^2}{2}V_1[V_2^2,p_x]] \nonumber \\
  &= \dfrac{q^2}{8m^3} (p_y^2 [V_2,[p_x^2,V_2]] +iV_1p_y[p_x,[p_x^2,V_2]] -i\tfrac{q^2}{2}V_1p_y[V_2,[V_2^2,p_x]] + \tfrac{q^2}{2}V_1^2[p_x,[V_2^2,p_x]]) \nonumber \\
  [H_1,[H_0,H_{-1}]] + h.c. &= \dfrac{q^2}{4m^3} (p_y^2[V_2,[p_x^2,V_2]] + \tfrac{q^2}{2} V_1^2 [p_x,[V_2^2, p_x]])
\end{align}
notice, the imaginary term cancels out when considering the Hermitian conjugate.

\begin{align}
  [H_2,[H_0,H_{-2}]] &= \dfrac{q^4}{128m^3} [V_1^2 - V_2^2, [p_x^2 + p_y^2 + \tfrac{q^2}{2} (V_1^2 + V_2^2), V_1^2 - V_2^2]] \nonumber \\
  &= -\dfrac{q^4}{128m^3} [ V_1^2 - V_2^2, [p_x^2,V_2^2]] \nonumber \\
  &= \dfrac{q^4}{128m^3} [V_2^2,[p_x^2,V_2^2]] \nonumber \\
  [H_2,[H_0,H_{-2}]] + h.c. &= \dfrac{q^4}{64m^3} [V_2^2,[p_x^2,V_2^2]]
\end{align}

\begin{align}
  [H_1,[H_1, H_{-2}]] &= -\dfrac{q^4}{32m^3} [ V_2 p_y + i V_1 p_x, [V_2 p_y + i V_1 p_x, V_1^2 - V_2^2]] \nonumber \\
  &= \dfrac{q^4}{32m^3} [ V_2 p_y + i V_1 p_x, iV_1 [p_x,V_2^2]] \nonumber \\
  &= \dfrac{q^4}{32m^3} V_1 ( ip_y [V_2,[p_x,V_2^2]] - V_1 [p_x,[p_x,V_2^2]]) \nonumber \\
  [H_1,[H_1,H_{-2}]] + h.c. &= \dfrac{q^4 V_1^2}{16m^3} [p_x,[V_2,p_x^2]]
\end{align}

\begin{align}
  [H_2,[H_{-1},H_{-1}]] &= -\dfrac{q^4}{32m^3} [V_1^2 - V_2^2, [V_2 p_y - i V_1 p_x, V_2 p_y - i V_1 p_x]] \nonumber \\
  &= -\dfrac{q^4}{32m^3} [V_1^2 - V_2^2, -i V_1p_y [V_2,p_x] - i V_1p_y [p_x,V_2]] \nonumber \\
  &= 0
\end{align}

\begin{align}
  [H_{-1}, [H_2, H_{-1}]] &= -\dfrac{q^4}{32m^3} [V_2 p_y - i V_1 p_x, [V_1^2 - V_2^2, V_2 p_y - i V_1 p_x]] \nonumber \\
  &= -\dfrac{q^4}{32m^3} [V_2 p_y - i V_1 p_x, iV_1[V_2^2,p_x]] \nonumber \\
  &= -\dfrac{q^4}{32m^3} V_1( i p_y [V_2,[V_2^2,p_x]] + V_1[p_x,[V_2^2,p_x]]) \nonumber \\
  [H_{-1}, [H_2, H_{-1}]] + h.c. &= -\dfrac{q^4V_1^2}{16m^3} [p_x,[V_2^2,p_x]]
\end{align}

\begin{align}
  H^{F(3)} &= \dfrac{1}{2\hbar^2\omega^2} [H_{-1},[H_0,H_1]] + \dfrac{1}{8\hbar^2\omega^2} [H_{-2},[H_0,H_2]] + \dfrac{1}{6\hbar^2\omega^2} [H_1,[H_1,H_{-2}]] \nonumber \\
  &+ \dfrac{1}{6\hbar^2\omega^2} [H_2,[H_{-1},H_{-1}]] - \dfrac{1}{3\hbar^2\omega^2} [H_{-1},[H_2,H_{-1}]] + h.c. \\
  &= \dfrac{q^2}{8m^3\hbar^2\omega^2} \left( p_y^2 [V_2,[p_x^2,V_2]] + \tfrac{q^2}{2} V_1^2 [p_x,[V_2^2,p_x]]\right) + \dfrac{q^4}{8^3m^3\hbar^2\omega^2} [V_2^2,[p_x^2,V_2^2]] \nonumber \\
  &+ \dfrac{q^4}{96m^3\hbar^2\omega^2} V_1^2 [p_x,[V_2,p_x^2]] + \dfrac{q^4}{48m^3\hbar^2\omega^2} [p_x,[V_2^2,p_x]] \\
  &= \dfrac{q^2 p_y^2}{8m^3\hbar^2\omega^2} [V_2,[p_x^2,V_2]] + \dfrac{9q^4V_1^2}{96m^3\hbar^2\omega^2} [p_x,[V_2^2,p_x]]
\end{align}
The Hamiltonian with $p_y=0$ becomes

\begin{align}
  H &= \dfrac{1}{2m} \left(p_x^2 + \dfrac{q^2}{2}(V_1^2+V_2^2)\right) + \dfrac{9 q^4 V_1^2}{96m^3\hbar^2\omega^2} [p_x,[V_2^2,p_x]] \nonumber \\
  &= \dfrac{1}{2m} \left(p_x^2 + \dfrac{q^2}{2}(V_1^2+V_2^2)\right) - \dfrac{9 q^4 V_1^2}{48m^3\omega^2} \left(\left(\partial_x V_2\right)^2 + V_2 \partial_x^2 V_2 \right)
\end{align}
Letting $V_2 = \tfrac{E}{\omega}\sin{Kx}$, $\partial_x V_2 = \tfrac{EK}{\omega}\cos{Kx}$, and $\partial_x^2 V_2 = -\tfrac{EK^2}{\omega} \sin{Kx}$ we arrive at

\begin{equation}
  H &= \dfrac{1}{2m} \left(p_x^2 + \dfrac{q^2E^2}{2\omega^2}(1+\sin^2{Kx})\right) - \dfrac{9 q^4 E^4 K^2}{48m^3\omega^6} \left( \cos^2{Kx} - \sin^2{Kx} \right)
\end{equation}
In the limit $Kx \ll 1$ it becomes

\begin{align}
  H &= \dfrac{1}{2m} \left(p_x^2 + \dfrac{q^2E^2}{2\omega^2}(1+K^2x^2)\right) - \dfrac{9 q^4 E^4 K^2}{48m^3\omega^6} \left( 1 - 2K^2x^2 \right) \nonumber \\
H &= \dfrac{1}{2m} \left(p_x^2 + \left( \dfrac{q^2E^2K^2}{2\omega^2} + \dfrac{9q^4E^4K^4}{48m^2\omega^6}\right) x^2\right) + \dfrac{q^2E^2}{4m\omega^2} - \dfrac{9 q^4 E^4 K^2}{48m^3\omega^6} \nonumber \\
\end{align}
