\section{Quantum harmonic oscillator}
\label{appendix:qho}

We will quickly derive this energy solution and derive ladder operators.
Rewrite the quantum harmonic oscillator as (and dropping the operator hat)

\begin{equation*}
  H = \dfrac{1}{2m} \left( p_x^2 + m^2 \omega^2 x^2 \right),
\end{equation*}
then complete the square by adding "zero"

\begin{align}
  H &= \dfrac{1}{2m} \left( [m\omega x - i p_x] [m\omega x + i p_x] -im\omega[xp_x - p_x x] \right) \nonumber \\
  &= \dfrac{1}{2m} \left( [m\omega x - i p_x] [m\omega + i p_x] + m \hbar \omega \right) \nonumber \\
  &= \dfrac{1}{2m} \left( \tilde{a}^{\dagger} \tilde{a} + m\hbar \omega \right) \nonumber \\
  &= \hbar \omega \left( \dfrac{\tilde{a}^{\dagger} \tilde{a} }{2m\hbar \omega} + \dfrac{1}{2} \right) \nonumber \\
  &= \hbar \omega \left( a^{\dagger} a + \dfrac{1}{2} \right),
\end{align}
where $a = \tfrac{1}{\sqrt{2}} \left(\sqrt{\tfrac{m\omega}{\hbar}} x + i \tfrac{p_x}{\sqrt{m\hbar\omega}} \right)$.
We have simplified the Hamiltonian into new creation and annihilation operators, called ladder operators, which we will now show how they work.
Also note $[a, a^{\dagger}] = 1$.
Let looks at how the operator commutes with the Hamiltonian

\begin{align}
  [H, a] &= Ha - aH = \hbar \omega \left( \ca a a + \dfrac{a}{2} - a \ca a - \dfrac{a}{2}) \right) \nonumber \\
      &= \hbar \omega (\ca a  - (1 + \ca a) ) a \nonumber \\
      &= - \hbar \omega a, \quad \text{and} \\
  [H, \ca] &= H\ca - \ca H = \hbar \omega \left( \ca a \ca + \dfrac{\ca}{2} - \ca \ca a - \dfrac{\ca}{2}\right) \nonumber \\
      &= \hbar \omega \ca ( a\ca - \ca a) \nonumber \\
      &= \hbar \omega \ca.
\end{align}
Let $H$ act on the wavefunction as

\begin{align}
  H | \psi_n \rangle &= E_n | \psi_n \rangle. \nonumber \\
  H\ca | \psi_n \rangle &= (\ca H + \hbar \omega \ca) |\psi_n\rangle \nonumber \\
  H\ca | \psi_n \rangle &= (E_n +\hbar \omega) \ca |\psi_n\rangle. \nonumber \\
  H a |\psi_n \rangle &= (E_n - \hbar \omega) a |\psi_n\rangle. \nonumber \\
\end{align}
We notice

\begin{align}
  H|\psi_0\rangle &= E_0 |\psi_0\rangle \nonumber \\
  Ha|\psi_0\rangle &= (E_0-\hbar \omega) a|\psi_0\rangle,
\end{align}
however, $E_0$ is the minimum so $E_0-\hbar \omega$ cannot exist and thus

\begin{equation}
  a |\psi_0\rangle = 0
\end{equation}
Again, we look at the ground state energy

\begin{align}
  \langle \psi_0 | H | \psi_0 \rangle &= \langle \psi_0 | \hbar \omega(\ca a +1/2) | \psi_0 \rangle \nonumber \\
  E_0 &= \hbar \omega \langle \psi_0 | \ca a | \psi_0 \rangle + \dfrac{\hbar\omega}{2} \langle \psi_0 | \psi_0 \rangle \nonumber \\
  E_0 &= \dfrac{\hbar \omega}{2}.
\end{align}
Then for the given eigenstates

\begin{equation*}
  \ca|\psi_0\rangle, \quad \ca\ca|\psi_0\rangle, \quad \ca\ca\ca|\psi_0\rangle, \quad \dots
\end{equation*}
with eigenvalues
\begin{equation*}
  \tfrac{3}{2} \hbar\omega, \quad \tfrac{5}{2} \hbar\omega, \quad \tfrac{7}{2} \hbar\omega, \quad \dots
\end{equation*}
Can be generalized to

\begin{equation*}
  |\psi_n\rangle \propto ( \ca )^n|\psi_0\rangle,
\end{equation*}
with the eigenenergy

\begin{equation*}
  E_n = \hbar\omega \left( n + \tfrac{1}{2} \right).
\end{equation*}
With our goal complete we continue on to determine how the ladder operators evolve the state.
We can now renormalize our proportional expression

\begin{align*}
  |\psi_{n+1}\rangle &= c\ca|\psi_n\rangle \nonumber \\
  1 = \langle \psi_{n+1} | \psi_{n+1} \rangle &= |c|^2 (\langle \psi_n|\ca)(\ca|\psi_n\rangle) \nonumber \\
  &= |c|^2 \langle \psi_n | a\ca | \psi_n \rangle \nonumber \\
  &= |c|^2 \langle \psi_n | \dfrac{H}{\hbar\omega} + \dfrac{1}{2} |\psi_n \rangle \nonumber \\
  &= |c|^2 \left(\dfrac{E_n}{\hbar\omega} +\dfrac{1}{2} \right) \nonumber \\
  &= |c|^2 \left(n+1\right) \nonumber \\
  |c| &= \dfrac{1}{\sqrt{n+1}}
\end{align*}
which give the following relation

\begin{equation}
  |\psi_{n+1} \rangle = \dfrac{\ca}{\sqrt{n+1}} |\psi_n \rangle.
\end{equation}
Similarly we find

\begin{equation}
  |\psi_{n-1} \rangle = \dfrac{\ca}{\sqrt{n}} |\psi_n \rangle.
\end{equation}
Thus $\ca a |\psi_n \rangle = n |\psi_n \rangle$.
The energy of the system is definitively

\begin{equation}
  E_n = \hbar\omega \left(n + \dfrac{1}{2} \right)
\end{equation}

\section{Dirac equation in the presence of a magnetic field}
\label{appendix:dirac}
We now focus on how the presence of a magnetic field affects the Dirac equation.
The Dirac Hamiltonian with vector potential

\begin{equation}
  \ham = v_f \bm{\sigma}\cdot(\op{p} - q\op{A})
\end{equation}
Using the previous definition, $\vec{A} = Bx\hat{y}$, the Hamiltonian becomes

\begin{align}
  \ham = v_f \sigma_x \op{p}_x + v_f \sigma_y (\op{p}_y - qB\op{x})
\end{align}
Like Schrodinger's equation we use the same ansatz wavefunction and arrive at

\begin{align}
  \ham = v_f \sigma_x \op{p}_x - v_f \sigma_y (qB\op{x} - \hbar k_y) \nonumber \\
  \ham = v_f \sigma_x \op{p}_x - v_f \sigma_y qB\op{x},
\end{align}
where we recognize the x term is just shifted by a constant.
In matrix form the Hamiltonian looks like

\[
  \ham = i v_f q B
  \begin{bmatrix}
    0 & \op{p}_x + iqB\op{x} \\
    \op{p}_x - iqB\op{x} & 0
  \end{bmatrix}
\]

\[
  \ham = i v_f \sqrt{2\hbar qB}
  \begin{bmatrix}
    0 & \ca\\
    -a & 0
  \end{bmatrix}
\]
The form of the Hamiltonian can be quickly solved by squaring then acting on a wavefunction

\[
  \ham^2 = 2\hbar qB  v_f^2
  \begin{bmatrix}
    \ca a & 0 \\
    0 & a\ca
  \end{bmatrix}
\]
We focus on the first element of the matrix

\begin{align}
  \langle \psi_n | \ham_{11}^2 |\psi_n\rangle &= \langle \psi_n | E_n^2 |\psi_n\rangle \nonumber \\
    &= 2 \hbar qB v_f^2 \langle \psi_n | \ca a |\psi_n\rangle \nonumber \\
    &= 2 \hbar qB v_f^2 \langle \psi_n | n |\psi_n\rangle \nonumber \\
    E_n^2 &= 2 \hbar qB n v_f^2 \nonumber \\
    E_n &= \pm v_f \sqrt{2\hbar qB n}
\end{align}

