\documentclass[10pt, letterpaper]{article}

% Packages:
\usepackage[
    ignoreheadfoot, % set margins without considering header and footer
    top=2 cm, % separation between body and page edge from the top
    bottom=2 cm, % separation between body and page edge from the bottom
    left=2 cm, % separation between body and page edge from the left
    right=2 cm, % separation between body and page edge from the right
    footskip=1.0 cm, % separation between body and footer
    % showframe % for debugging
]{geometry} % for adjusting page geometry
\usepackage{titlesec} % for customizing section titles
\usepackage{tabularx} % for making tables with fixed width columns
\usepackage{array} % tabularx requires this
\usepackage[dvipsnames]{xcolor} % for coloring text
\definecolor{primaryColor}{RGB}{0, 0, 0} % define primary color
\usepackage{bibentry}
\usepackage{enumitem} % for customizing lists
\usepackage{fontawesome5} % for using icons
\usepackage{amsmath} % for math
\usepackage[
    pdftitle={John Doe's CV},
    pdfauthor={John Doe},
    pdfcreator={LaTeX with RenderCV},
    colorlinks=true,
    urlcolor=primaryColor
]{hyperref} % for links, metadata and bookmarks
\usepackage[pscoord]{eso-pic} % for floating text on the page
\usepackage{calc} % for calculating lengths
\usepackage{bookmark} % for bookmarks
\usepackage{lastpage} % for getting the total number of pages
\usepackage{changepage} % for one column entries (adjustwidth environment)
\usepackage{paracol} % for two and three column entries
\usepackage{ifthen} % for conditional statements
\usepackage{needspace} % for avoiding page brake right after the section title
\usepackage{iftex} % check if engine is pdflatex, xetex or luatex

% Ensure that generate pdf is machine readable/ATS parsable:
\ifPDFTeX
    \input{glyphtounicode}
    \pdfgentounicode=1
    \usepackage[T1]{fontenc}
    \usepackage[utf8]{inputenc}
    \usepackage{lmodern}
\fi

\usepackage{charter}

% Some settings:
\raggedright
\AtBeginEnvironment{adjustwidth}{\partopsep0pt} % remove space before adjustwidth environment
\pagestyle{empty} % no header or footer
\setcounter{secnumdepth}{0} % no section numbering
\setlength{\parindent}{0pt} % no indentation
\setlength{\topskip}{0pt} % no top skip
\setlength{\columnsep}{0.15cm} % set column separation
\pagenumbering{gobble} % no page numbering

\titleformat{\section}{\needspace{4\baselineskip}\bfseries\large}{}{0pt}{}[\vspace{1pt}\titlerule]

\titlespacing{\section}{
    % left space:
    -1pt
}{
    % top space:
    0.3 cm
}{
    % bottom space:
    0.2 cm
} % section title spacing

\renewcommand\labelitemi{$\vcenter{\hbox{\small$\bullet$}}$} % custom bullet points
\newenvironment{highlights}{
    \begin{itemize}[
        topsep=0.10 cm,
        parsep=0.10 cm,
        partopsep=0pt,
        itemsep=0pt,
        leftmargin=0 cm + 10pt
    ]
}{
    \end{itemize}
} % new environment for highlights


\newenvironment{highlightsforbulletentries}{
    \begin{itemize}[
        topsep=0.10 cm,
        parsep=0.10 cm,
        partopsep=0pt,
        itemsep=0pt,
        leftmargin=10pt
    ]
}{
    \end{itemize}
} % new environment for highlights for bullet entries

\newenvironment{onecolentry}{
    \begin{adjustwidth}{
        0 cm + 0.00001 cm
    }{
        0 cm + 0.00001 cm
    }
}{
    \end{adjustwidth}
} % new environment for one column entries

\newenvironment{twocolentry}[2][]{
    \onecolentry
    \def\secondColumn{#2}
    \setcolumnwidth{\fill, 4.5 cm}
    \begin{paracol}{2}
}{
    \switchcolumn \raggedleft \secondColumn
    \end{paracol}
    \endonecolentry
} % new environment for two column entries

\newenvironment{threecolentry}[3][]{
    \onecolentry
    \def\thirdColumn{#3}
    \setcolumnwidth{, \fill, 4.5 cm}
    \begin{paracol}{3}
    {\raggedright #2} \switchcolumn
}{
    \switchcolumn \raggedleft \thirdColumn
    \end{paracol}
    \endonecolentry
} % new environment for three column entries

\newenvironment{header}{
    \setlength{\topsep}{0pt}\par\kern\topsep\centering\linespread{1.5}
}{
    \par\kern\topsep
} % new environment for the header

\newcommand{\placelastupdatedtext}{% \placetextbox{<horizontal pos>}{<vertical pos>}{<stuff>}
  \AddToShipoutPictureFG*{% Add <stuff> to current page foreground
    \put(
        \LenToUnit{\paperwidth-2 cm-0 cm+0.05cm},
        \LenToUnit{\paperheight-1.0 cm}
    ){\vtop{{\null}\makebox[0pt][c]{
        \small\color{gray}\textit{Last updated in September 2024}\hspace{\widthof{Last updated in September 2024}}
    }}}%
  }%
}%

% save the original href command in a new command:
\let\hrefWithoutArrow\href

% new command for external links:


\begin{document}
    \nobibliography{resume-publications}
    \bibliographystyle{unsrt}
    \newcommand{\AND}{\unskip
        \cleaders\copy\ANDbox\hskip\wd\ANDbox
        \ignorespaces
    }
    \newsavebox\ANDbox
    \sbox\ANDbox{$|$}

    \begin{header}
        \fontsize{25 pt}{25 pt}\selectfont Aidan Winblad

        \vspace{5 pt}

        \normalsize
        \mbox{Fort Collins, CO}%
        \kern 5.0 pt%
        \AND%
        \kern 5.0 pt%
        \mbox{\hrefWithoutArrow{mailto:acwinblad@gmail.com}{acwinblad@gmail.com}}%
        \kern 5.0 pt%
        \AND%
        \kern 5.0 pt%
        \mbox{\hrefWithoutArrow{tel:+90-541-999-99-99}{+1 (970) 391-8041}}%
        \kern 5.0 pt%
        \AND%
        \kern 5.0 pt%
        \mbox{\hrefWithoutArrow{https://github.com/yourusername}{github.com/acwinblad}}%
    \end{header}

    \vspace{5 pt - 0.3 cm}


    \section{Professional Summary}

        \begin{onecolentry}
          I am a recent physics Ph.D. graduate who specialized in condensed matter physics and with 10+ years of experience in computational modeling and numerical analysis.
          My research topics focused on topological phenomena in low dimensional systems, such as topological superconductors and topological insulators.
          Research skills include analytical and numerical methods to solve partial differential, examine eigenmodes, determine topological states, and presenting findings intuitively.
          Most of my Ph.D. research used Python3 (numpy, scipy, and matplotlib) and some Mathematica  scripting.
          I am published in peer-reviewed journals and have presented at several physics conferences.
          Eager to apply analytical and numerical techniques to simulate and visualize physics.


        \end{onecolentry}

    \section{Work Experience}

        \begin{twocolentry}{
            Aug 2016 – May 2025
        }
            \textbf{Graduate Research and Teacher Assistant}, Colorado State University Physics Department -- Fort Collins, CO\end{twocolentry}

        \vspace{0.10 cm}
        \begin{onecolentry}
            \begin{highlights}
            \item As a teaching assistant, I taught labs and recitations, tutored students, graded coursework and exams, and proctored exams. I am versed in patience and can adapt my teaching communication to cater to different learning needs.
            \item Using analytical and numerical analysis I showed superconducting triangular islands can host and braid Majorana fermions with a rotating gauge potential. Demonstrated braiding of a 4-qubit system on a minimal network of triangular islands. Contributing to novel approaches in topological quantum computing outside of 1D wire junctions.
            \item Through analytical and perturbation techniques, I showed oblique incident circularly polarized light on Dirac and 2D electron gas systems induces quantum Hall effect. Laser light electric field is shown to cause an effective magnetic field experienced by the materials. An important push in the study of non-equilibrium physics.
            \item Both projects involved building large matrices and using a nearest neighbor algorithm. Collaborated within my research group to optimize nearest neighbor algorithms, reducing computation times and improved efficiency of Hamiltonian matrix construction.
            \end{highlights}
        \end{onecolentry}

        \vspace{0.2 cm}

        \begin{twocolentry}{
            June 2015 – Aug 2016
        }
      \textbf{Computational Physicist, Contractor}, Engility -- Fort Sam Houston AFRL, TX\end{twocolentry}

        \vspace{0.10 cm}
        \begin{onecolentry}
            \begin{highlights}
              \item Developing and testing ray tracing physics to simulate laser tissue interactions.
              \item Developing a hot spot and temperature gradient tracker from thermal radiation video data.
              \item Examine damage prediction models, perform experimental validation, and set safety standards.
              \item Writing technical reports, presenting results in presentations, proceedings, and journal articles.
            \end{highlights}
        \end{onecolentry}


    \section{Education}

        \begin{twocolentry}{
            May 2015
        }
      \textbf{Fort Hays State University}, \textit{Hays, KS}, B.S. in Mathematics and B.S. in Physics\end{twocolentry}

        \vspace{0.10 cm}

        \begin{twocolentry}{
            Dec 2019
        }
      \textbf{Colorado State University}, \textit{Fort Collins, CO}, M.S. in Physics\end{twocolentry}

        \vspace{0.10 cm}

        \begin{twocolentry}{
            May 2025
        }
      \textbf{Colorado State University}, \textit{Fort Collins, CO}, Ph.D. in Physics\end{twocolentry}

        \vspace{0.10 cm}

    \section{Publications}

        \begin{onecolumnentry}
          \begin{highlights}
            \item \bibentry{hokr2016}
            \item \bibentry{winblad2024}
            \item \bibentry{winblad2025}
          \end{highlights}
        \end{onecolumnentry}

    \section{Technologies}

        \begin{onecolentry}
          \textbf{Current proficiencies:} Python3 (numpy and matplotlib), Git, i3-wm, \LaTeX,  Vim
        \end{onecolentry}

        \vspace{0.2 cm}

        \begin{onecolentry}
          \textbf{Exposure to:} Bash/Shell, C/C++, FORTRAN 90, Gnuplot, Linux/Unix, Mathematica, Obsidian (markdown), Python2, Zotero
        \end{onecolentry}

        \vspace{0.2 cm}

        \begin{onecolentry}
          \textbf{Limited exposure to:} ALE3D, Arduino, Blender, MATLAB, Objective-C, Rust, Swift, Typescript
        \end{onecolentry}

    %\pagebreak
    \section{Computational Physics Experience}

        \begin{onecolumnentry}
          \begin{highlights}
          \item Graph Theory
            \begin{highlights}
              \item Developed a shortest path algorithm for optimizing foot traffic on campus sidewalks. (C++)
            \end{highlights}
          \item Classical Mechanics and Chaos
            \begin{highlights}
              \item Solved ODE of damped-driven pendulum to show chaotic motion with RK4. (C++)
            \end{highlights}
          \item Electromagnetic Modeling
            \begin{highlights}
            \item Simulated potential and electric field of conductors using SOR. (C++)
              \item Simulated propagation of electromagnetic wave in 1D using FDTD. (C++)
              \item Developed a 2D (r,z) Focused Gaussian Beam Monte-Carlo Multi-Layer method, with ABCD transforms, for modeling energy deposits into tissue for laser tissue interactions. (C++)
            \end{highlights}
          \item Heat Transfer
            \begin{highlights}
              \item Solved Heat equation for a 2D system using SOR. (C++)
            \end{highlights}
          \item Quantum Simulations
            \begin{highlights}
              \item Simulated a Gaussian Wave packet in 1D Schrodinger equation infinite square well using Goldberg method and 2D infinite square well using ADI method. (FORTRAN 90)
              \item Computed eigenvalues and states of Schrodinger equation for 1D harmonic oscillator using FEM. (Python2)
            \end{highlights}
          \item Topological Materials
            \begin{highlights}
              \item Developed numerical methods for modeling topological quantum computing logic gates of superconducting triangular islands, advancing research on new platforms for quantum computation. (Python3)
              \item Modeled quantum Hall effect in non-equilibrium system using Floquet theory of oblique incident circularly polarized light upon Dirac or 2DEG substrates. (Python3)
            \end{highlights}
          \end{highlights}
        \end{onecolumnentry}




\end{document}
