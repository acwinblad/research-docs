\documentclass[12pt, letterpaper]{letter}

% Packages:
\usepackage[
    ignoreheadfoot, % set margins without considering header and footer
    top=2 cm, % separation between body and page edge from the top
    bottom=2 cm, % separation between body and page edge from the bottom
    left=2 cm, % separation between body and page edge from the left
    right=2 cm, % separation between body and page edge from the right
    footskip=1.0 cm, % separation between body and footer
    % showframe % for debugging
]{geometry} % for adjusting page geometry
\usepackage{titlesec} % for customizing section titles
\usepackage{tabularx} % for making tables with fixed width columns
\usepackage{array} % tabularx requires this
\usepackage[dvipsnames]{xcolor} % for coloring text
\definecolor{primaryColor}{RGB}{0, 0, 0} % define primary color
\usepackage{bibentry}
\usepackage{enumitem} % for customizing lists
\usepackage{fontawesome5} % for using icons
\usepackage{amsmath} % for math
\usepackage[
    pdftitle={Aidan Winblad's CV},
    pdfauthor={Aidan Winblad},
    pdfcreator={LaTeX with RenderCV},
    colorlinks=true,
    urlcolor=primaryColor
]{hyperref} % for links, metadata and bookmarks
\usepackage[pscoord]{eso-pic} % for floating text on the page
\usepackage{calc} % for calculating lengths
\usepackage{bookmark} % for bookmarks
\usepackage{lastpage} % for getting the total number of pages
\usepackage{changepage} % for one column entries (adjustwidth environment)
\usepackage{paracol} % for two and three column entries
\usepackage{ifthen} % for conditional statements
\usepackage{needspace} % for avoiding page brake right after the section title
\usepackage{iftex} % check if engine is pdflatex, xetex or luatex
\setlength\parindent{24pt}

% Ensure that generate pdf is machine readable/ATS parsable:
\ifPDFTeX
    \input{glyphtounicode}
    \pdfgentounicode=1
    \usepackage[T1]{fontenc}
    \usepackage[utf8]{inputenc}
    \usepackage{lmodern}
\fi

\usepackage{charter}

% Some settings:
\raggedright
\AtBeginEnvironment{adjustwidth}{\partopsep0pt} % remove space before adjustwidth environment
\pagestyle{empty} % no header or footer
\setcounter{secnumdepth}{0} % no section numbering
\setlength{\parindent}{0pt} % no indentation
\setlength{\topskip}{0pt} % no top skip
\setlength{\columnsep}{0.15cm} % set column separation
\pagenumbering{gobble} % no page numbering

\titleformat{\section}{\needspace{4\baselineskip}\bfseries\large}{}{0pt}{}[\vspace{1pt}\titlerule]

\titlespacing{\section}{
    % left space:
    -1pt
}{
    % top space:
    0.3 cm
}{
    % bottom space:
    0.2 cm
} % section title spacing

\renewcommand\labelitemi{$\vcenter{\hbox{\small$\bullet$}}$} % custom bullet points
\newenvironment{highlights}{
    \begin{itemize}[
        topsep=0.10 cm,
        parsep=0.10 cm,
        partopsep=0pt,
        itemsep=0pt,
        leftmargin=0 cm + 10pt
    ]
}{
    \end{itemize}
} % new environment for highlights


\newenvironment{highlightsforbulletentries}{
    \begin{itemize}[
        topsep=0.10 cm,
        parsep=0.10 cm,
        partopsep=0pt,
        itemsep=0pt,
        leftmargin=10pt
    ]
}{
    \end{itemize}
} % new environment for highlights for bullet entries

\newenvironment{onecolentry}{
    \begin{adjustwidth}{
        0 cm + 0.00001 cm
    }{
        0 cm + 0.00001 cm
    }
}{
    \end{adjustwidth}
} % new environment for one column entries

\newenvironment{twocolentry}[2][]{
    \onecolentry
    \def\secondColumn{#2}
    \setcolumnwidth{\fill, 4.5 cm}
    \begin{paracol}{2}
}{
    \switchcolumn \raggedleft \secondColumn
    \end{paracol}
    \endonecolentry
} % new environment for two column entries

\newenvironment{threecolentry}[3][]{
    \onecolentry
    \def\thirdColumn{#3}
    \setcolumnwidth{, \fill, 4.5 cm}
    \begin{paracol}{3}
    {\raggedright #2} \switchcolumn
}{
    \switchcolumn \raggedleft \thirdColumn
    \end{paracol}
    \endonecolentry
} % new environment for three column entries

\newenvironment{header}{
    \setlength{\topsep}{0pt}\par\kern\topsep\centering\linespread{1.5}
}{
    \par\kern\topsep
} % new environment for the header

\newcommand{\placelastupdatedtext}{% \placetextbox{<horizontal pos>}{<vertical pos>}{<stuff>}
  \AddToShipoutPictureFG*{% Add <stuff> to current page foreground
    \put(
        \LenToUnit{\paperwidth-2 cm-0 cm+0.05cm},
        \LenToUnit{\paperheight-1.0 cm}
    ){\vtop{{\null}\makebox[0pt][c]{
        \small\color{gray}\textit{Last updated in September 2024}\hspace{\widthof{Last updated in September 2024}}
    }}}%
  }%
}%

% save the original href command in a new command:
\let\hrefWithoutArrow\href

% new command for external links:


\begin{document}
    \nobibliography{resume-publications}
    \bibliographystyle{unsrt}
    \newcommand{\AND}{\unskip
        \cleaders\copy\ANDbox\hskip\wd\ANDbox
        \ignorespaces
    }
    \newsavebox\ANDbox
    \sbox\ANDbox{$|$}

    \begin{header}
        \fontsize{25 pt}{25 pt}\selectfont Aidan Winblad

        \vspace{5 pt}

        \fontsize{16 pt}{16 pt}\selectfont Teaching Philosophy

    \end{header}

    \vspace{5 pt}

    \normalsize

    Physics is not for spectating. It demands participation, curiosity, and hands-on interaction to truly grasp its principles. I reject passive, lecture-heavy models in favor of active engagement, which has shown to build deeper understanding, confidence, and retention. While some physics instruction can rely on slideshows or pre-written notes, I've witnessed how these approaches leave students disengaged and wanting. I've witnessed students thrive by doing physics--writing on boards in groups, testing hypotheses in labs, and debating in groups. They move from passive observers into active participants. These instances also allow for collaboration and inclusivity. Collaborative work aims to teach everyone, as peers often explain concepts in ways I might not anticipate or in a way that a fellow student will understand.

    Engagement alone is not enough. Structure and accountability are equally critical. Homework should reinforce intuition without becoming busywork. I assign handwritten problems (no online platforms) and grade them with a  balance of particaption credit and partial credit, ensuring effort is rewarded with maintaining rigor. Exams are frequent and focused, testing recent material to prevent cramming that plagues high-stakes midterms. While arithmetic and unit analysis have their place, they should serve physics, not the other way around. Labs, for instance, are the ideal setting for connecting numerical intuition to physical phenomena, not homework or exams.

    Above all, I strive to create a classroom where physics is embraced as a tool for understanding the world. Many students claim not to be ``math people'' but with patience and active engagement we can promote learning. I want to produce thinkers not calculators and for students to apply physics (or problem solving skills) to everyday challenges in any field. Every student--regardless of background--deserves to experience physics as it is meant to be, as an exploration, a debate, a collective push against the boundaries of what we know. I hope to make it accessible to all students, not just those who've been historically privileged.

\end{document}
