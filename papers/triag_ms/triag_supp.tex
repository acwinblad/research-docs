\documentclass[aps,prb,showpacs,amsmath,amssymb,superscriptaddress]{revtex4-2}

\usepackage{tabularx}
\usepackage{bm}
%\usepackage[demo]{graphicx}
\usepackage{graphicx}

\usepackage{hyperref}
\hypersetup{colorlinks=true,urlcolor= blue,citecolor=blue,linkcolor= blue,bookmarks=true,bookmarksopen=false}

\usepackage{color}

\usepackage{amsmath,mathtools}
\usepackage{multirow}
\usepackage{dcolumn}
\usepackage{amssymb,amscd,xypic,bm,wasysym}
\usepackage{float}
\usepackage{cleveref}
\usepackage[caption=false,position=top,captionskip=0pt,farskip=0pt]{subfig}
\captionsetup[subfigure]{justification=raggedright,singlelinecheck=false}


\newcommand{\Red}[1]{\textcolor{red}{#1}}
%\newcommand{\vb}[1]{\boldsymbol{#1}}
\usepackage{soul}

% reset vec and hat style to a bold type
\let\oldhat\hat
\renewcommand{\hat}[1]{\oldhat{\mathbf{#1}}}
\renewcommand{\vec}[1]{\mathbf{#1}}
% stretches the vertical spacing of arrays/matrices
\renewcommand{\arraystretch}{1.5}
\setlength{\jot}{10pt}

\newcommand{\ham}{\mathcal{H}}
\newcommand{\cc}{c^{\dagger}}
\newcommand{\de}{\Delta}


\begin{document}

\title{Supplemental Material for ``Superconducting triangular islands as a platform for manipulating Majorana zero modes"}

\author{Aidan Winblad}
\affiliation{Department of Physics, Colorado State University, Fort Collins, CO 80523, USA}

\author{Hua Chen}
\affiliation{Department of Physics, Colorado State University, Fort Collins, CO 80523, USA}
\affiliation{School of Advanced Materials Discovery, Colorado State University, Fort Collins, CO 80523, USA}

\maketitle

\section{Many-body Berry phase calculation for the 3-site Kitaev triangle}

In this section we provide details for calculating the many-body Berry phase for braiding two MZM in the Kitaev triangle, as shown in Fig.~2 in the main text. To start we use the Hamiltonian Eq.~(1) in the main text,
\begin{equation}\label{eq:HBdG}
  \ham = \sum_{\langle j l \rangle} (-te^{i\phi_{jl}}\cc_{j} c_l + \de e^{i\theta_{jl}} c_{j} c_l + {\rm h.c.}) - \sum_{j} \mu \cc_j c_j,
\end{equation}
We then write the creation and annihilation operators in the following Fock space basis for three spinless fermions
\begin{align*}
\left(|0\rangle, |1\rangle, \dots, |7\rangle\right) \equiv \,& \{ |n_1,n_2,n_3\rangle \}\\
  =\,&\big(|0,0,0 \rangle, \\
  &|1,0,0 \rangle, |0,1,0 \rangle, |0,0,1 \rangle, \\
  &|0,1,1 \rangle, |1,0,1 \rangle, |1,1,0 \rangle, \\
  &|1,1,1 \rangle\big )
\end{align*}
The creation(annihilation) operators in this space are defined as
\begin{eqnarray}
  \cc_j |n_1,\dots,n_j, \dots\rangle &=& \sqrt{n_j+1}(-1)^{s_j}|n_1,\dots,n_j+1,\dots\rangle, \\\nonumber
  c_j |n_1,\dots,n_j, \dots\rangle &=& \sqrt{n_j} (-1)^{s_j}|n_1,\dots,n_j-1,\dots\rangle,
\end{eqnarray}
where
\begin{eqnarray}
  s_j =
  \begin{cases}
    \sum_{l=1}^{j-1} n_l &  j>1 \\
    0 & j = 1
  \end{cases}
\end{eqnarray}

For the initial configuration corresponding to $\bm \phi_1$ in Eq.~(6) of the main text, diagonalizing the $8\times 8$ BdG Hamiltonian in the above basis leads to two degenerate ground states that can be distinguished by the occupation number of the following fermion operator constructed from the two MZM at the two bottom vertices
\begin{equation}
  c_M \equiv  \dfrac{1}{2} ( a_1 + i b_2 ),\; n_M \equiv c_M^\dag c_M
\end{equation}
The two degenerate ground states for the initial configuration, denoted as $|0\rangle_i$ and $|1\rangle_i$, therefore satisfy
\begin{eqnarray}
    n_M | 0 \rangle_i &=& 0,\\\nonumber
    n_M | 1 \rangle_i &=& |1\rangle_i
\end{eqnarray}
In practice, we first construct the operator $R_{\rm gs}$ as a $8\times 2$ matrix by combining the two column eigenvectors of the two lowest-energy eigenstates of the initial BdG Hamiltonian:
\begin{eqnarray}
    R_{\rm gs}\equiv (\psi_i,\psi_i')
\end{eqnarray}
and then diagonalize the projected $n_M$ operator:
\begin{eqnarray}
    U_n^\dag (R_{\rm gs}^\dag n_M R_{\rm gs}) U_n \equiv R_i^\dag n_M R_i = \begin{pmatrix}
        0 & \\
       & 1
    \end{pmatrix}
\end{eqnarray}

To carry out the Berry phase calculation we next need to adiabatically ``rotate'' the vector potential field by following the linearly interpolated closed parameter path described in the main text, which is discretized into $N + 1$ segments. At any given point labeled by $j$ along the path, we diagonalize the corresponding Hamiltonian and construct the projection operator $P_j$ using the two lowest-energy eigenvectors $\psi_j,\psi_j'$:
\begin{eqnarray}
    P_j \equiv \psi_j\otimes \psi_j^\dag + \psi'_j\otimes \psi'^\dag_j
\end{eqnarray}
where $\otimes$ means tensor product. The $2\times 2$ Berry phase matrix $M_{f\leftarrow i}$ for the given parameter path is then obtained as
\begin{eqnarray}
 M_{f\leftarrow i} = \lim_{N\rightarrow \infty} R_f^\dag P_{N} P_{N-1}\dots P_{1} R_i
\end{eqnarray}
where $R_f = R_i$ since the path is closed.

By using a large enough $N$ we found the converged $M_{f\leftarrow i}$ matrix has only diagonal elements being nonzero, meaning the braiding only changes each ground state by a scalar phase factor. Their values are $(M_{f\leftarrow i})_{00} = e^{i0.37\pi}$ and $(M_{f\leftarrow i})_{11} = e^{i(0.37-0.5)\pi}$.

\Red{Could you double check the numbers? I previously got $e^{i0.118\pi}$ for $|0\rangle$ and $e^{-i0.382\pi}$ for $|1\rangle$.}

We end this section by noting that the parameter path considered for the 3-site Kitaev triangle above is not equivalent to rotating a staggered vector potential but to separately manipulating the Peierls phases along the three edges. We have also done calculations for the latter case and found the two lowest-energy states fail to be degenerate everywhere along the parameter path, leading to non-standard Berry phase values.

\section{Majorana corner modes for finite-width triangles}

\begin{figure}[ht]
  \hspace{28pt}
  \subfloat[]{\includegraphics[width=0.65\textwidth]{./figures/supp/topological-phase-diagram-1pi6-n-3.pdf}} \\
  \vspace{-20pt}
  \subfloat[]{\includegraphics[width=0.46\textwidth]{./figures/supp/spectral-flow-nr-50-w-3-mu-1_6.pdf}} \\
  \hspace{70pt}
  \subfloat[]{\includegraphics[width=0.50\textwidth]{./figures/supp/GS-A-2_74-nr-50-w-3-mu-1_6.pdf}}
  \caption{(a) Topological phase diagram for a $W=3$ hollow triangle obtained by overlapping the $\mathcal{M}(A, \mu)$ plots of 1D chains with $\mathbf A = A\hat{y}$ and $\mathbf A = A(\frac{\sqrt{      3}}{2}\hat{x}+\frac{1}{2}\hat{y})$. Color scheme: white---$\mathcal{M}=1$, dark blue---$\mathcal{M}=-1$, light blue---$\mathcal{M}=0$ (b) Near-gap BdG eigen-energies vs $A$ for a finite triangle with edge length $L=50$, $W=3$, and $\mu=1.6$. (c) BdG eigenfunction $|\Psi|^2$ summed over the two zero modes at $A=2.4709$.}
  \label{fig: supp pd}
\end{figure}

A model that is closer to a realistic hollow triangular island is the finite-width triangular chain or ribbon. An example, illustrated in Figure \ref{fig: supp pd} (c), has its edge length $L=50$ and width $W=3$. The phase diagram Fig.~\ref{fig: supp pd} (a) is created in a similar way as that in Fig.~3 (a) of the main text, assuming a constant vector potential $\mathbf A = A\hat{y}$ and infinitely long $W=3$ ribbons. The spectral flow for the actual triangle with $\mu = 1.6$ in Fig.~\ref{fig: supp pd} (b) show MZM in the correct regions expected from the phase diagram. Fig.~\ref{fig: supp pd} (c) plots the MZM wavefunction for $A=2.7409$ and $\mu=1.6$ that are indeed well localized at the bottom corners.

\begin{figure}[ht]
  \subfloat[]{\includegraphics[width=0.4\textwidth]{./figures/supp/spectral-flow-w-3.pdf}}
  \subfloat[]{\includegraphics[width=0.4\textwidth]{./figures/supp/GS-T-Square-w-3.pdf}} \\
  \subfloat[]{\includegraphics[width=0.4\textwidth]{./figures/supp/GS-T-Circle-w-3.pdf}}
  \subfloat[]{\includegraphics[width=0.4\textwidth]{./figures/supp/GS-T-Diamond-w-3.pdf}}
  \caption{(a) Spectral flow of a hollow triangle with $W=3$, $L=50$, $\mu=1.6$, and $A=2.75$ with increasing rotation angle $\varphi$, defined through $\mathbf A = A(-\sin\varphi \hat{x} + \cos\varphi \hat{y})$. (b-d) BdG eigenfunction $|\Psi|^2$ summed over the two zero modes at $\varphi = 0, \frac{\pi}{6}$, and $\frac{\pi}{3}$, respectively.}
  \label{fig: supp rotation}
\end{figure}

\Red{Shouldn't the eigenenergies at $\varphi = 0$ the same as that at $\varphi = \pi$? Why are they different in Fig.~\ref{fig: supp rotation} (a)?}

We next rotate the uniform vector potential to examine how the MZM move on a hollow triangle. Figure~\ref{fig: supp rotation} shows the spectral flow and eigenfunctions as we rotate $\varphi=0$ to $\varphi=\pi$ counterclockwisely. The two MZM cycle through the three vertices as $\varphi$ increases from 0 to $\pi/3$ in a similar manner as that in Fig.~4 of the main text.  

\section{Braiding MZM in a small network of triangles}

It is now pertinent to see if we can use the previous results to interchange two MFs that are not paired together.
We initialize three triangular chains adjavent to each other, connecting their bottom corners together.
The left and right triangles initialize a constant vector potential with $\varphi=0$, while the middle island will initialize with $\varphi=\dfrac{\pi}{6}$.
Slowly rotating the middle triangle chains vector potential to $\varphi=\dfrac{\pi}{3}$ we see the the middle MFs swap which MF pair they are connected to.

\begin{figure}
  \subfloat[]{\includegraphics[width=0.4\textwidth]{./figures/supp/spectral-flow-braiding.pdf}} \\
  \subfloat[]{\includegraphics[width=0.4\textwidth]{./figures/supp/GS-Braid-0.pdf}}
  \subfloat[]{\includegraphics[width=0.4\textwidth]{./figures/supp/GS-Braid-1.pdf}} \\
  \vspace{-40pt}
  \subfloat[]{\includegraphics[width=0.4\textwidth]{./figures/supp/GS-Braid-2.pdf}}
  \subfloat[]{\includegraphics[width=0.4\textwidth]{./figures/supp/GS-Braid-3.pdf}} \\
  \vspace{-40pt}
  \subfloat[]{\includegraphics[width=0.4\textwidth]{./figures/supp/GS-Braid-4.pdf}}
  \caption{(a) Spectral flow of three triangular chains connected at their bottom certices in a row with $W=1$, $L=50$, and $\mu=1.6$. Each triangle has a vector potential field applied as $\mathbf A = A(\sin\varphi \hat{x} + \cos\varphi \hat{y})$, with $A=2.6$. The left and right trianlges have $\varphi=0$, while the middle triangle rotates $\varphi = \dfrac{\pi}{6}$ to $\varphi = \dfrac{\pi}{3}$. (b)-(f) BdG eigenfunction $|\Psi|^2$ summed over the four zero modes over the rotation of the middle triangles $\varphi$. }
\end{figure}

\bibliography{triag_ref}

\end{document}
