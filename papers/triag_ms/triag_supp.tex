\documentclass[aps,prb,showpacs,amsmath,amssymb,superscriptaddress]{revtex4-2}

\usepackage{tabularx}
\usepackage{bm}
%\usepackage[demo]{graphicx}
\usepackage{graphicx}

\usepackage{hyperref}
\hypersetup{colorlinks=true,urlcolor= blue,citecolor=blue,linkcolor= blue,bookmarks=true,bookmarksopen=false}

\usepackage{color}

\usepackage{amsmath,mathtools}
\usepackage{multirow}
\usepackage{dcolumn}
\usepackage{amssymb,amscd,xypic,bm,wasysym}
\usepackage{float}
\usepackage{cleveref}
\usepackage[caption=false,position=top,captionskip=0pt,farskip=0pt]{subfig}
\captionsetup[subfigure]{justification=raggedright,singlelinecheck=false}


\newcommand{\Red}[1]{\textcolor{red}{#1}}
%\newcommand{\vb}[1]{\boldsymbol{#1}}
\usepackage{soul}

\begin{document}
	
\title{Supplementary Material for ``Superconducting triangular islands as a platform for manipulating Majorana fermions"}
	
\author{Aidan Winblad}
\affiliation{Department of Physics, Colorado State University, Fort Collins, CO 80523, USA}	

\author{Hua Chen}
\affiliation{Department of Physics, Colorado State University, Fort Collins, CO 80523, USA}
\affiliation{School of Advanced Materials Discovery, Colorado State University, Fort Collins, CO 80523, USA}

\maketitle

\section{1D spinless $p$-wave superconductor with a linear vector potential}
In this section we discuss the topological phases of a 1D spinless $p$-wave superconductor subject to a linear vector potential:
\begin{eqnarray}
	H = \int dx \left\{   \psi^\dag(x) \left[ \frac{(p+ebx)^2}{2m} -\mu \right]\psi(x) + \Delta\left[  \psi^\dag(x) \partial_x \psi^\dag(x) + {\rm h.c.} \right] \right\}
\end{eqnarray} 
where $p=-i\hbar \partial_x$ is the 1D momentum operator, $b$ is the strength of a nominal magnetic field, $e>0$ is the absolute value of electron charge, $\mu$ is the chemical potential, $\Delta$ is the strength of the $p$-wave pairing potential. Such a vector potential generally exists when a magnetic field along $z$ is applied and can be removed by choosing a different gauge for the vector potential in the absence of superconductivity. This can also be seen from the fact that the $\Delta = 0$ Hamiltonian has eigensolutions 
\begin{eqnarray}
	\psi_k(x) = \frac{1}{\sqrt{2\pi L}} e^{i k x - \frac{ieb}{\hbar}x^2},\; \epsilon_k = \frac{\hbar^2 k^2}{2m}.
\end{eqnarray}
But when $\Delta \neq 0$, the Hamiltonian is not diagonal in this basis any more. 

The eigenequation in Nambu spinor basis is 
\begin{eqnarray}
	\begin{pmatrix}
		\frac{(-i\hbar \partial_x + eb x)^2}{2m} - \mu & \Delta \partial_x \\
		-\Delta \partial_x & -\frac{(-i\hbar \partial_x - eb x)^2}{2m} + \mu
	\end{pmatrix} \Psi = \epsilon \Psi
\end{eqnarray}

\bibliography{triag_ref}

\end{document}
