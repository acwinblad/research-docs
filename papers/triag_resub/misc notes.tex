\documentclass[aps,physrev,amsmath,amssymb]{revtex4-2}

\usepackage{tabularx}
\usepackage{bm}
%\usepackage[demo]{graphicx}
\usepackage{graphicx}

\usepackage{hyperref}
\hypersetup{colorlinks=true,urlcolor= blue,citecolor=blue,linkcolor= blue,bookmarks=true,bookmarksopen=false}

\usepackage{color}

\usepackage{amsmath,mathtools}
\usepackage{multirow}
\usepackage{dcolumn}
\usepackage{amssymb,amscd,xypic,bm,wasysym}
\usepackage{float}
\usepackage{cleveref}
\usepackage[caption=false,position=top,captionskip=0pt,farskip=0pt]{subfig}
\captionsetup[subfigure]{justification=raggedright,singlelinecheck=false}


\newcommand{\Red}[1]{\textcolor{red}{#1}}
%\newcommand{\vb}[1]{\boldsymbol{#1}}
\usepackage{soul}

\begin{document}

\title{Miscellaneous results related to Majorana triangles}
\author{Aidan Winblad and Hua Chen}

\maketitle

\section{Vector potential and gauge invariance}
In this section we address the question of how to understand the Peierls substitution in BdG Hamiltonian. 

Although the superconductivity order parameter appears to break the U(1) gauge symmetry, all physical observables are still gauge invariant. More explicitly, consider a general tight-binding BdG Hamiltonian 
\begin{eqnarray}\label{eq:HBdGgeneral}
	H = \sum_{ij,\alpha\beta} \left(t_{ij}^{\alpha\beta}c_{i\alpha}^\dag c_{j\beta} + \Delta_{ij,\alpha\beta} c_{i\alpha}c_{j\beta} - \frac{\mu}{2}c_{i\alpha}^\dag c_{i\alpha} + {\rm h.c.} \right) \equiv \frac{1}{2}C^\dag h C
\end{eqnarray}
where $i,j$ label position, $\alpha,\beta$ label any internal degrees of freedom, and $C = (\{c_{i\alpha}\},\{c^\dag_{i\alpha}\})^T$. $H$ has the eigensolutions
\begin{eqnarray}
	H |\psi_n\rangle &=& \epsilon_n |\psi_n\rangle \\\nonumber
	|\psi_n\rangle &=& d^\dag_{\psi_n} |\Omega\rangle = \sum_{i\alpha\sigma}c^{\sigma}_{i\alpha} | \Omega\rangle U_{i\alpha\sigma,n} 
\end{eqnarray}
where $|\Omega\rangle$ is the BCS ground state, $\sigma=\pm$ distinguishes the creation (particle) and annihilation (creation for hole) operators, and $U$ is a Bogoliubov transformation matrix which is unitary for fermions. Substituting $|\psi\rangle$ into the eigenequation leads to
\begin{eqnarray}
	U^\dag h U = {\rm Diag}[\{\epsilon_n\}]
\end{eqnarray}

where the pairing potential satisfies the gap equation
\begin{eqnarray}
	\Delta_{ij,\alpha\beta} &=& Z^{-1}{\rm Tr}[V_{j\beta,i\alpha}c_{j\beta}^\dag c^\dag_{i\alpha} e^{-\frac{1}{k_B T} H}] \\\nonumber
	&=& \sum_n f(\epsilon_n) (U^\dag {\mathbb{V}} U)_{nn}
\end{eqnarray}
where $\mathbb{V}$ is a matrix with the only nonzero element being $\mathbb{V}_{j\beta+,i\alpha-} = V_{j\beta, i\alpha}$, $f$ is the Fermi-Dirac distribution function. 

We now show that physical observables are gauge invariant. A gauge transformation corresponds to
\begin{eqnarray}
	\mathbf A\rightarrow \mathbf A' = \mathbf A + \nabla \chi
\end{eqnarray}
where $\mathbf A$ is the vector potential. $\mathbf A$ enters the tight-binding Hamiltonian implicitly through the Peierls substitution:
\begin{eqnarray}\label{eq:peierls}
	c_{i\alpha}^\dag \rightarrow \tilde{c}^\dag_{i\alpha} = e^{-\frac{ie}{\hbar} \int_0^{\mathbf r_i} \mathbf A\cdot d \mathbf l} c_{i\alpha}^\dag
\end{eqnarray} 
and we can understand Eq.~\eqref{eq:HBdGgeneral} as that written for certain $\mathbf A$ already absorbed into the definitions of $t$ and $\Delta$. The gauge transformation leads to 
\begin{eqnarray}
	c^\dag_{i\alpha} \rightarrow c^\dag_{i\alpha} e^{-\frac{ie}{\hbar}\chi_i}
\end{eqnarray}
The Hamiltonian therefore becomes
\begin{eqnarray}
	H\rightarrow H'&=& \sum_{ij,\alpha\beta}\left[ t_{ij}^{\alpha\beta} e^{-\frac{ie}{\hbar}(\chi_i - \chi_j)} c_{i\alpha}^\dag c_{j\beta} + \Delta_{ij,\alpha\beta} e^{\frac{ie}{\hbar}(\chi_i + \chi_j)}c_{i\alpha}c_{j\beta}  - \frac{\mu}{2}c_{i\alpha}^\dag c_{i\alpha} + {\rm h.c.} \right] \\\nonumber
	&=&\frac{1}{2}C^\dag U_\chi h U_\chi^\dag C
\end{eqnarray}
where 
\begin{eqnarray}
	U_\chi = {\rm Diag}[\{e^{-\frac{ie}{\hbar}\chi_i}\},\{e^{\frac{ie}{\hbar}\chi_i}\}]
\end{eqnarray}
As a result, the BdG eigenvalues as well as all other physical observables represented in terms of Bogoliubov quasiparticles are invariant under the gauge transformation.

The above derivation includes, however, an assumption. Namely the pairing potential $\Delta_{ij,\alpha\beta}$ stays unchanged. This is indeed the case, since
\begin{eqnarray}
	\Delta'_{ij,\alpha\beta} &=& Z^{'-1}{\rm Tr}[V_{j\beta,i\alpha} c^\dag_{j\beta}c^{\dag}_{i\alpha} e^{-\frac{ie}{\hbar}(\chi_i + \chi_j)} e^{-\frac{1}{k_B T}H'}] \\\nonumber
	&=& \sum_{n}f(\epsilon_n) (U^\dag U^\dag_\chi U_\chi \mathbb{V} U_\chi^\dag U_\chi U) \\\nonumber
	&=&\Delta_{ij,\alpha\beta}
\end{eqnarray}

\section{Analytic solution of the Kitaev triangle}
In this section we present some analytic results related to the Kitaev triangle. 

We start from the 1D Kitaev chain Hamiltonian with complex nearest-neighbor hopping $-te^{i\phi}$ and $p$-wave pairing $\Delta e^{i\theta}$ in the Kitaev limit ($t=\Delta > 0, \mu = 0$):
\begin{eqnarray}
	H = \sum_{n}\left(- te^{i\phi}c_n^\dag c_{n+1} + \Delta e^{i\theta}c_nc_{n+1} + {\rm h.c.}\right)
\end{eqnarray}
In the Majorana fermion basis $a_n = c_n + c_n^\dag$, $b_n = -i(c_n - c_n^\dag)$ the Hamiltonian becomes
\begin{eqnarray}
H = -\frac{it}{2} \sum_n \left[(S_\phi - S_\theta) a_n a_{n+1} + (S_\phi + S_\theta)b_n b_{n+1} + (C_\phi - C_\theta) a_n b_{n+1} - (C_\phi + C_\theta)b_na_{n+1}\right]
\end{eqnarray}
where $S_\phi\equiv \sin\phi$, $C_\phi\equiv \cos\phi$, etc. Therefore, when $\phi = \theta$, $a_n$ becomes decoupled from $a_{n+1}$ and $b_{n+1}$, and $a_1$ drops out from the Hamiltonian. Similarly, when $\phi = \theta + \pi$, $b_1$ becomes isolated. To find the other MZM, we note that when $\phi = \theta$, terms involving $a_{N}$ and $b_N$ in the Hamiltonian are
\begin{eqnarray}
	H_N = -itb_{N-1}(S_\phi b_{N} - C_\phi a_N).
\end{eqnarray}
Considering the unitary transformation
\begin{eqnarray}
	\begin{pmatrix}
		a_N' \\
		b_N'
	\end{pmatrix} \equiv\begin{pmatrix}
	C_\phi & - S_\phi\\
	S_\phi & C_\phi 
\end{pmatrix}\begin{pmatrix}
a_N\\
b_N
\end{pmatrix}
\end{eqnarray}
we have 
\begin{eqnarray}
	H_N = itb_{N-1} a'_N
\end{eqnarray}
Therefore the other MZM is $b'_N = S_\phi a_N + C_\phi b_N$. Similarly, when $\phi = \theta + \pi$ the other MZM is $a'_{N} \equiv C_\phi a_N - S_\phi b_N$. 

We now consider the three edges of the Kitaev triangle separately. The MZM due to each edge are respectively
\begin{eqnarray}
	1-2:&& a_1,\,b_2\\\nonumber
	2-3:&& b_2,\,\frac{1}{2}a_3 + \frac{\sqrt{3}}{2}b_3\\\nonumber
	3-1:&& a_1,\, \frac{\sqrt{3}}{2}a_3 + \frac{1}{2} b_3
\end{eqnarray}
One can therefore see that the two MZM at site 3 are not compatible with each other. To get the non-MZM eigenstates, we write down the remaining terms of the Hamiltonian in the Majorana basis
\begin{eqnarray}
	H &=& -\frac{it}{2}\left( -2b_1 a_2 - \sqrt{3} a_2 a_3 + a_2 b_3 + \sqrt{3} b_1 b_3 - b_1 a_3   \right)\\\nonumber
	&=&\frac{1}{2}(b_1, a_2, a_3, b_3)h\begin{pmatrix}
	b_1\\
	a_2\\
	a_3\\
	b_3
\end{pmatrix}\\\nonumber
h&\equiv&-it\begin{pmatrix}
	0 & -1 & -\frac{1}{2} & \frac{\sqrt{3}}{2} \\
	1 & 0 & -\frac{\sqrt{3}}{2} & \frac{1}{2}\\
	\frac{1}{2} & \frac{\sqrt{3}}{2} & 0 & 0 \\
	-\frac{\sqrt{3}}{2} & -\frac{1}{2} & 0 & 0
\end{pmatrix} = t\left( -\frac{1}{2}\sigma_0 \tau_y - \frac{1}{2}\sigma_z\tau_y -\frac{1}{2}\sigma_y \tau_z + \frac{\sqrt{3}}{2} \sigma_x \tau_y \right)
\end{eqnarray}
$h$ has the following symmetry:
\begin{eqnarray}
	O = \left(\frac{\sqrt{3}}{2}\sigma_x - \frac{1}{2}\sigma_z\right) \tau_y
\end{eqnarray}
We therefore rotate the Hamiltonian so that $O$ becomes diagonal using the following unitary operator
\begin{eqnarray}
	U = e^{-\frac{i\pi}{3}\sigma_y}\otimes e^{i\frac{\pi}{4}\tau_x}
\end{eqnarray}
which leads to 
\begin{eqnarray}
	U^\dag O U &=& {\rm Diag}(1,-1,-1,1)
\end{eqnarray}
$U$ therefore block-diagonalizes $h$ as
\begin{eqnarray}
	U^\dag h U = 	\frac{t}{2}\begin{pmatrix}
		1 &  &  & -1 \\
		& -1 & 1 & \\
		& 1 & -3 & \\
		-1 & & & 3 
	\end{pmatrix}
\end{eqnarray}
which can then be diagonalized by
\begin{eqnarray}
	V = \begin{pmatrix}
		\frac{1+ \sqrt{2}}{\sqrt{4+2\sqrt{2}}} & 0 & \frac{1 - \sqrt{2}}{\sqrt{4-2\sqrt{2}}} & 0 \\
		0 & \frac{1 + \sqrt{2}}{\sqrt{4 + 2\sqrt{2}}} & 0 & \frac{1-\sqrt{2}}{\sqrt{4 - 2\sqrt{2}}} \\
		0 & \frac{1}{\sqrt{4+ 2\sqrt{2}}} & 0 & \frac{1}{\sqrt{4 - 2\sqrt{2}}} \\
		\frac{1}{\sqrt{4+2\sqrt{2}}} & 0 & \frac{1}{\sqrt{4-2\sqrt{2}}} & 0
	\end{pmatrix}
\end{eqnarray}
as
\begin{eqnarray}
	V^\dag U^\dag h U V =t\times {\rm Diag}\left( 1 -\frac{\sqrt{2}}{2},  -1 +\frac{\sqrt{2}}{2},  1 +\frac{\sqrt{2}}{2},  -1 -\frac{\sqrt{2}}{2} \right)
\end{eqnarray}
We therefore have the two lowest excited states
\begin{eqnarray}
	\psi_{+1} &=& (b_1,a_2,a_3,b_3) U \begin{pmatrix}
	\frac{1+ \sqrt{2}}{\sqrt{4+2\sqrt{2}}} \\
0  \\
0 \\
\frac{1}{\sqrt{4+2\sqrt{2}}} 
	\end{pmatrix} = (b_1,a_2,a_3,b_3)\times \frac{1}{4\sqrt{2+\sqrt{2}}} \begin{pmatrix}
	1+\sqrt{2}-\sqrt{3} i \\
	(1+\sqrt{2})i-\sqrt{3}  \\
	i+ \sqrt{3} + \sqrt{6} \\
	1 + (\sqrt{3} + \sqrt{6})i
\end{pmatrix} \\\nonumber
	\psi_{-1} &=& (b_1,a_2,a_3,b_3) U \begin{pmatrix}
		0\\
	\frac{1+ \sqrt{2}}{\sqrt{4+2\sqrt{2}}} \\
	\frac{1}{\sqrt{4+2\sqrt{2}}} \\
	0
\end{pmatrix} = (b_1,a_2,a_3,b_3)\times \frac{1}{4\sqrt{2+\sqrt{2}}} \begin{pmatrix}
	(1+\sqrt{2})i-\sqrt{3} \\
	1+\sqrt{2}-\sqrt{3} i \\
	1+ (\sqrt{3} + \sqrt{6} )i\\
    i + \sqrt{3} + \sqrt{6}
\end{pmatrix}
\end{eqnarray}
The first excited states can therefore be understood as a hybridization between the ``bulk" states of the 1-2 bond and the fermion on site 3.  

We next consider the braiding process and particularize to the $\boldsymbol{\phi}_1\rightarrow \boldsymbol{\phi}_2$ step. The Hamiltonian in the fermion basis becomes
\begin{eqnarray}
H &=& - e^{ix}c_1^\dag c_2 + c_1 c_2 + e^{-ix}c_1 c_2^\dag - c_1^\dag c_2^\dag \\\nonumber
&+&  - e^{-\frac{\pi}{3}i} c_2^\dag c_3 + e^{\frac{2\pi}{3}i} c_2 c_3 + e^{\frac{\pi}{3}i}c_2 c_3^\dag - e^{-\frac{2\pi}{3}i} c_2^\dag c_3^\dag \\\nonumber
&+&  e^{\left(-\frac{\pi}{3}-x\right)i} c_1 c_3^\dag  - e^{-\frac{2\pi}{3}i} c_1 c_3  - e^{\left(\frac{\pi}{3}+x\right)i} c_1^\dag c_3  + e^{\frac{2\pi}{3}i}  c_1^\dag c_3^\dag
\end{eqnarray}
where we have temporarily omitted the energy unit $t$. We then have
\begin{eqnarray}
	[c_1^\dag, H] &=& c_2 + e^{-ix} c_2^\dag + e^{\left(-\frac{\pi}{3}-x\right)i} c_3^\dag - e^{-\frac{2\pi}{3}i} c_3 \\\nonumber
	[c_1, H] &=& -e^{ix}c_2 - c_2^\dag - e^{\left(\frac{\pi}{3}+x\right)i} c_3 + e^{\frac{2\pi}{3}i} c_3^\dag\\\nonumber
	&=& -e^{ix} \left[ c_2 + e^{-i x}c_2^\dag - e^{-\frac{2\pi}{3}i} c_3 +  e^{\left(-\frac{\pi}{3}-x\right)i} c_3^\dag \right]
\end{eqnarray}
Therefore
\begin{eqnarray}
	[e^{\frac{ix}{2}}c_1^\dag + e^{-\frac{ix}{2}} c_1, H] = 0
\end{eqnarray}
Namely we have an MZM:
\begin{eqnarray}
	\tilde{a}_1 \equiv e^{\frac{ix}{2}}c_1^\dag + e^{-\frac{ix}{2}} c_1 = \frac{1}{2}e^{\frac{ix}{2}} (a_1 - ib_1) + \frac{1}{2}e^{-\frac{ix}{2}} (a_1 + ib_1) = C_{\frac{x}{2}} a_1 + S_{\frac{x}{2}} b_1
\end{eqnarray}
To find the other MZM, we calculate the commutators between the other fermion operators with the Hamiltonian:
\begin{eqnarray}
	[c_2^\dag, H] &=& e^{ix} c_1^\dag - c_1 - e^{-\frac{i\pi}{3}} c_3 + e^{\frac{i\pi}{3}} c_3^\dag \\\nonumber
	[c_2, H] &=& -e^{-ix} c_1 + c_1^\dag + e^{\frac{i\pi}{3}} c_3^\dag - e^{-\frac{i\pi}{3}} c_3\\\nonumber
	[c_3^\dag, H] &=& e^{-\frac{i\pi}{3}}c_2^\dag + e^{-\frac{i\pi}{3}} c_2 - e^{\frac{i\pi}{3}} c_1 + e^{i\left(\frac{\pi}{3} + x \right)}c_1^\dag \\\nonumber
	[c_3, H]	&=& -e^{\frac{i\pi}{3}}c_2 - e^{\frac{i\pi}{3}} c_2^\dag + e^{-\frac{i\pi}{3}} c_1^\dag - e^{-i\left(\frac{\pi}{3} + x \right)}c_1
\end{eqnarray}
Therefore
\begin{eqnarray}
	[c_2 - c_2^\dag, H] &=& (1-e^{-ix})c_1 + (1-e^{ix})c_1^\dag\\\nonumber
	[e^{\frac{i\pi}{6}}\left(e^{i\frac{2\pi}{3}} c_3^\dag + c_3\right), H] & = &  e^{\frac{i\pi}{6}} (1-e^{-i\left(\frac{\pi}{3} + x\right)})c_1 + e^{-\frac{i\pi}{6}}( 1- e^{i\left(\frac{\pi}{3} + x\right)})c_1^\dag
\end{eqnarray}
However
\begin{eqnarray}
-\frac{1-e^{-ix}}{e^{\frac{i\pi}{6}} (1-e^{-i\left(\frac{\pi}{3} + x\right)})} = - \frac{2-2\cos x}{e^{\frac{i\pi}{6}} (1-e^{-i\left(\frac{\pi}{3} + x\right)})(1-e^{ix})} = \frac{1-\cos x}{\cos\left( x+ \frac{\pi}{6}\right)-\frac{\sqrt{3}}{2}}
\end{eqnarray}
Thus the following is the other MZM:
\begin{eqnarray}
	\tilde{b}_{23} &\equiv& -iN\left(\left[\cos\left( x+ \frac{\pi}{6}\right)-\frac{\sqrt{3}}{2}\right](c_2 - c_2^\dag) + (1-\cos x) \left( e^{\frac{i\pi}{6}} c_3 -e^{-\frac{i\pi}{6}} c_3^\dag\right)\right)\\\nonumber
	&=& N\left(\left[\cos\left( x+ \frac{\pi}{6}\right)-\frac{\sqrt{3}}{2}\right]b_2 + (1-\cos x)\left( \frac{1}{2} a_3 + \frac{\sqrt{3}}{2} b_3\right)\right)
\end{eqnarray}
where $N$ is a normalization factor. When $x=0$ only the first term survives since
\begin{eqnarray}
\lim_{x\rightarrow 0} \frac{1-\cos x}{\cos\left( x+ \frac{\pi}{6}\right)-\frac{\sqrt{3}}{2}} = 0 
\end{eqnarray}
while when $x = -\frac{\pi}{3}$ only the second term survives.

\end{document}
