\documentclass[11pt,letterpaper]{article}

\usepackage{graphicx}
\usepackage{bbm}
%\usepackage{mathrsfs}
\usepackage{amsmath}
%\usepackage{txfonts}
\usepackage{amssymb,amscd,xypic,bm}
%\usepackage{hyperref}
\usepackage{geometry}
\usepackage{multirow}
\usepackage{tabularx}
\usepackage{dcolumn}
\newcolumntype{Y}{>{\centering\arraybackslash}X}
\geometry{left=2.5cm,right=2.5cm,top=2.5cm,bottom=2.5cm}
\parskip 10pt

\makeatletter
\newsavebox{\@brx}
\newcommand{\llangle}[1][]{\savebox{\@brx}{\(\m@th{#1\langle}\)}%
	\mathopen{\copy\@brx\mkern2mu\kern-0.9\wd\@brx\usebox{\@brx}}}
\newcommand{\rrangle}[1][]{\savebox{\@brx}{\(\m@th{#1\rangle}\)}%
	\mathclose{\copy\@brx\mkern2mu\kern-0.9\wd\@brx\usebox{\@brx}}}
\makeatother

\title{Tight-binding models of Floquet quantum Hall effect}
\author{M. Tahir and Hua Chen}
\date{}

\begin{document}

\maketitle

\section{Introduction}
In this note we present details of how to set up the tight-binding models for Floquet quantum Hall effect.


\section{General framework of Floquet theory}

In this section we review the basic results of the Floquet theory and how to recast it into a matrix diagonalization problem. The discussion in this section is mostly following \cite{eckardt_2015}.

For a time-periodic Hamiltonian $H(t) = H(t+T)$ with period $T$, the time evolution of a wavefunction governed by it is described by the Schr\"{o}dinger equation
\begin{eqnarray}\label{eq:SchrHt}
i\hbar \partial_t \psi(t) = H(t) \psi(t).
\end{eqnarray}
Floquet theorem states that $\psi(t)$ must satisfy
\begin{eqnarray}
\psi(t+T) = \psi(t) e^{-i \frac{\epsilon T}{\hbar}},
\end{eqnarray}
where $\epsilon$ is a real number of energy units, or equivalently
\begin{eqnarray}
\psi(t) = e^{-i \frac{\epsilon t}{\hbar}} u_{\epsilon}(t),
\end{eqnarray}
where $u_{\epsilon}(t) = u_{\epsilon}(t+T)$. 

Here we give a proof that is closely analogous to that of the Bloch theorem, based on plane wave expansion. An arbitrary wavefunction can be expanded into plane waves
\begin{eqnarray}
\psi(t) = \sum_{\epsilon} c_\epsilon e^{-i \frac{\epsilon t}{\hbar}},
\end{eqnarray}
where $\epsilon\in \mathbb{R}$, while a time-periodic function $H(t)$ can only be written as a discrete Fourier series
\begin{eqnarray}
H(t) = \sum_n H_n e^{i n \omega t},
\end{eqnarray}
where $\omega = 2\pi /T$, and $H_n = \frac{1}{T} \int_0^T H(t) e^{-i n \omega t} dt$. Substituting the two expansions above into Eq.~\ref{eq:SchrHt} gives 
\begin{eqnarray}
0 &=& \sum_\epsilon \left[ \sum_n H_n e^{-i \frac{(\epsilon - n \hbar \omega) t}{\hbar}} c_\epsilon - \epsilon c_\epsilon  e^{-i \frac{\epsilon t}{\hbar}} \right] \\\nonumber
&=& \sum_\epsilon \left[ \sum_n H_n c_{\epsilon + n\hbar \omega} - \epsilon c_\epsilon  \right] e^{-i \frac{\epsilon t}{\hbar}},
\end{eqnarray}
which leads to
\begin{eqnarray}\label{eq:cepseqn}
\sum_n H_n c_{\epsilon + n\hbar \omega} - \epsilon c_\epsilon = 0.
\end{eqnarray}
For an arbitrary $\epsilon\in \mathbb{R}$ we can define $\tilde{\epsilon} \in [-\hbar \omega /2, \hbar \omega /2)$ so that $\epsilon = \tilde{\epsilon} + m \hbar \omega$. It is apparent that Eq.~\ref{eq:cepseqn} only couples $c_{\tilde{\epsilon} + m\hbar \omega}$ belonging to the same $\tilde{\epsilon}$. We thus define
\begin{eqnarray}
c_{\tilde{\epsilon} + m\hbar \omega} \equiv c_{m \tilde{\epsilon}},
\end{eqnarray}
so that Eq.~\ref{eq:cepseqn} becomes a set of coupled equations for $c_{m \tilde{\epsilon}}$, $m \in \mathbb{Z}$:
\begin{eqnarray}\label{eq:cepsteqn}
\sum_n (H_n  - m\hbar \omega \delta_{n0} ) c_{m+n, \tilde{\epsilon}} = \tilde{\epsilon} c_{m \tilde{\epsilon}}.
\end{eqnarray}
Eq.~\ref{eq:cepseqn} is the eigenvalue problem of the infinite-dimensional matrix $\bar{Q}$ with the matrix elements
\begin{eqnarray}
\bar{Q}_{m,m+n} = H_n - m \hbar \omega\delta_{n0},
\end{eqnarray}
which is also the quasienergy operator in \cite{eckardt_2015}. In practice the number of eigenvalues $\tilde{\epsilon}$ is determined by the dimension of $H(t)$. The solutions of Eq.~\ref{eq:SchrHt} are therefore
\begin{eqnarray}\label{eq:psitildee}
\psi_{\tilde{\epsilon}} (t) = \sum_m c_{m \tilde{\epsilon}} e^{-i\frac{(\tilde{\epsilon} + m \hbar \omega)t}{\hbar}} = e^{-i\frac{\tilde{\epsilon} t}{\hbar}} \sum_m c_{m \tilde{\epsilon}} e^{-i m  \omega t} \equiv e^{-i\frac{\tilde{\epsilon} t}{\hbar}} u_{\tilde{\epsilon}}(t).
\end{eqnarray}

The proof above also gives a useful device for calculating the Floquet states $\psi_{\tilde{\epsilon}} (t) $ based on plane wave expansion. In general $H_n$ can be a complicated operator depending on, e.g. position, spin, etc., and $c_{m \tilde{\epsilon}}$ is a function depending on these quantum numbers. One can choose a representation that makes $H_0$ diagonal, such as the Bloch representation, leading to the eigenvalues $\epsilon_{n \bm k}$ of the time-averaged Hamiltonian ($H_0$). When $H_n$ is 0 for all $n \neq 0$, we have $\tilde{\epsilon} = \epsilon_{n \bm k} - m \hbar \omega$, $m \in \mathbb{Z}$. When $H_n$ is nonzero for any $n \neq 0$ there is in general no simple relationship between $\tilde{\epsilon}$ and $ \epsilon_{n \bm k}$. Nonetheless, when $H_n$, $n \neq 0$ can be viewed as perturbation the spectrum of $\tilde{\epsilon}$ is similar to that of $\epsilon_{n \bm k} - m \hbar \omega$, i.e., the eigenenergies $\epsilon_{n \bm k}$ together with infinite number of its copies shifted vertically by $m \hbar \omega$.

The importance of $\tilde{\epsilon}$ is that it completely determines the stroboscopic motion of an arbitrary Floquet wavefunction, i.e.,
\begin{eqnarray}
\psi_{\tilde{\epsilon}} (t + m T) = e^{-i \frac{\tilde{\epsilon} m T}{\hbar}} \psi_{\tilde{\epsilon}} (t),\,\, \forall m\in \mathbb{Z}.
\end{eqnarray}
Since $\{\psi_{\tilde{\epsilon}}(t)\}$ is a complete set at time $t$, the stroboscopic evolution of an arbitrary wavefunction governed by $H(t)$ is 
\begin{eqnarray}
\Psi(t + m T) = \sum_{\tilde{\epsilon}} C_{\tilde{\epsilon}} e^{-i \frac{\tilde{\epsilon} m T}{\hbar}} \psi_{\tilde{\epsilon}} (t),
\end{eqnarray}
where $\Psi(t) =  \sum_{\tilde{\epsilon}} C_{\tilde{\epsilon}} \psi_{\tilde{\epsilon}} (t)$. The full time-evolution operator $\hat{U}(t_1,t_0)$ is therefore
\begin{eqnarray}\label{eq:Uevolve}
\hat{U}(t_1,t_0) = \sum_{\tilde{\epsilon}}|\psi_{\tilde{\epsilon}} (t_1)\rangle \langle \psi_{\tilde{\epsilon}} (t_0) | = \sum_{\tilde{\epsilon}} |u_{\tilde{\epsilon}} (t_1)\rangle \langle u_{\tilde{\epsilon}} (t_0) | e^{-i\frac{\tilde{\epsilon}(t_1 - t_0)}{\hbar}}.
\end{eqnarray}
Now we introduce two operators
\begin{eqnarray}\label{eq:UFt1t0}
\hat{U}^F(t_1,t_0) \equiv \sum_{\tilde{\epsilon}} |u_{\tilde{\epsilon}} (t_1)\rangle \langle u_{\tilde{\epsilon}} (t_0) |,
\end{eqnarray}
and 
\begin{eqnarray}\label{eq:HFt0}
\hat{H}^F_{t_0} \equiv \sum_{\tilde{\epsilon}} |u_{\tilde{\epsilon}} (t_0)\rangle\tilde{\epsilon} \langle u_{\tilde{\epsilon}} (t_0) |,
\end{eqnarray}
which allows us to rewrite Eq.~\ref{eq:Uevolve} as
\begin{eqnarray}
\hat{U}(t_1,t_0) = \hat{U}_F(t_1,t_0) \exp\left[ -i\frac{(t_1 - t_0)\hat{H}^F_{t_0} }{\hbar}  \right] = \exp\left[ -i\frac{(t_1 - t_0)\hat{H}^F_{t_1} }{\hbar}  \right] \hat{U}_F(t_1,t_0). 
\end{eqnarray}
Namely, the full time evolution is separated into two parts: $\hat{H}^F_{t_0}$ governs the stroboscopic evolution \emph{with the starting time} $t_0$, since 
\begin{eqnarray}
 \exp\left[ -i\frac{m T \hat{H}^F_{t_0} }{\hbar}  \right] \psi_{\tilde{\epsilon}}(t_0) = e^{-i\frac{m T \tilde{\epsilon}}{\hbar} } \psi_{\tilde{\epsilon}}(t_0) = \psi_{\tilde{\epsilon}}(t_0 + m T),
\end{eqnarray}
while $\hat{U}_F (t_1, t_0)$ evolves the periodic part of the Floquet wavefunctions. $\hat{H}^F_{t_0} $ and $\hat{U}_F (t_1, t_0)$ are respectively called the Floquet Hamiltonian and the micromotion operator.   

The most unsettling property of $\hat{H}^F_{t_0} $ is its dependence on $t_0$. To get rid of it we note that Eq.~\ref{eq:psitildee} implies
\begin{eqnarray}
|u_{\tilde{\epsilon}}(t) \rangle =  \sum_{\alpha} \left(\sum_m c_{m \tilde{\epsilon}}^{\alpha} e^{-i m\omega t} \right)|\alpha\rangle \equiv\sum_\alpha  |\alpha\rangle U_{\alpha,\tilde{\epsilon}} (t) ,
\end{eqnarray}
where the time-independent basis $|\alpha\rangle$ spans the Hilbert space of $H(t)$, and $U (t)$ is a time-dependent unitary matrix satisfying $U(t+T) = U(t)$. Substituting this $|u_{\tilde{\epsilon}}(t) \rangle$ into Eq.~\ref{eq:SchrHt} gives
\begin{eqnarray}
{\rm Diag}[\{\tilde{\epsilon}\}] = U^\dag H (t) U - i\hbar U^\dag \partial_t U = U^\dag \bar{Q} U,
\end{eqnarray} 
where ${\rm Diag}[\{\tilde{\epsilon}\}]$ is a diagonal matrix with its eigenvalues being $\tilde{\epsilon}$. Comparing this with the effect of a time-dependent unitary transformation of the wavefunction $\psi' = U^\dag \psi$ in the Schr\"{o}dinger equation:
\begin{eqnarray}
i\hbar \partial_t \psi' = (U^\dag H U - i\hbar U^\dag \partial_t U)\psi' \equiv H' \psi',
\end{eqnarray}
we can see that $U$ essentially transforms $H(t)$ to an effective Hamiltonian $H' = U^\dag \bar{Q}U$ which is time independent. The time evolution of $\psi$ can thus obtained as
\begin{eqnarray}
\psi(t_1) &=& U(t_1) \psi'(t_1) = U(t_1) \exp\left[ -i \frac{H' (t_1 - t_0)}{\hbar}  \right] \psi'(t_0) \\\nonumber
&=&  U(t_1) \exp\left[ -i \frac{H' (t_1 - t_0)}{\hbar}  \right] U^\dag(t_0) \psi(t_0)\\\nonumber
&=&\hat{U}(t_1,t_0)\psi(t_0). 
\end{eqnarray}
We therefore define
\begin{eqnarray}
\hat{H}_F \equiv U^\dag \bar{Q} U = H'
\end{eqnarray} 
as the Floquet effective Hamiltonian, which gives the time-evolution operator
\begin{eqnarray}\label{eq:Ut1t0}
\hat{U}(t_1,t_0) =  U(t_1) \exp\left[ -i \frac{\hat{H}_F (t_1 - t_0)}{\hbar}  \right] U^\dag(t_0).
\end{eqnarray}
Intuitively, this means that the time evolution is obtained by first doing a gauge transformation to the time-independent gauge, evolving the system, and finally gauge-transforming back to the original gauge. 

Although we have been assuming that $U(t)$ diagonalizes $\bar{Q}$, this is not necessary. Any time-independent unitary transformation multiplied to $U(t)$ can still make $\hat{H}_F$ time independent. To make connection between the $t_0$ dependent Floquet Hamiltonian $\hat{H}^F_{t_0}$ in Eq.~\ref{eq:HFt0} and the effective Hamiltonian $\hat{H}_F$, we use a minimal $U(t)$ that is independent of the basis of $\hat{H}(t)$:
\begin{eqnarray}
U_F(t) =\sum_m c_m e^{-im\omega t}, 
\end{eqnarray}
which is a time-dependent scalar function. In the matrix form of $\bar{Q}$, this $U_F(t)$ block-diagonalizes $\bar{Q}$. All the diagonal blocks have the form $H_F - m\hbar \omega \mathbbm{1}$. Here we removed the hat of $H_F$ to indicate that it is a matrix written in certain representation instead of an operator. In this particular representation or gauge, $|\alpha(t)\rangle = |\alpha\rangle U_F(t)$. We thus have
\begin{eqnarray}
\hat{H}_{t_0}^F = \sum_{\tilde{\epsilon}}|u_{\tilde{\epsilon}}(t_0)\rangle \tilde{\epsilon} \langle u_{\tilde{\epsilon}} (t_0)| = \sum_{\alpha\beta} U_F(t_0) |\alpha\rangle (H_F)_{\alpha\beta} \langle \beta | U_F^\dag(t_0).  
\end{eqnarray}  
Or loosely speaking $\hat{H}_{t_0}^F =  U_F(t_0) \hat{H}_F U_F^\dag(t_0)$. Therefore the $t_0$ dependence in $\hat{H}_{t_0}^F$ is only due to a gauge transformation and is not physical. The complete information of time evolution can be obtained from $H_F$ \emph{and} $U_F$ according to Eq.~\ref{eq:Ut1t0}.

In practice, to obtain the quasienergy spectrum or $H_F$ we simply start from the eigenvalue problem Eq.~\ref{eq:cepseqn} for $\bar{Q}\equiv \bar{H} + \bar{Q}_0$, where $\bar{H}_{m,m+n} = H_n$ and $(\bar{Q}_0)_{m,m+n} = -m\hbar \omega \delta_{n0}$. We can either use perturbation theory and treat $\bar{H}$ as perturbation, which is accurate in the high-frequency limit, or directly diagonalize $\bar{Q}$ with a large enough cutoff. The first several terms in the perturbation series of $H_F$ are given in Eqs.~86-89 in \cite{eckardt_2015} ($m$ there should be $-m$ in our notation). 

\section{Including a spatially and temporally varying vector potential in a tight-binding model}

In this section we discuss how to include a spatially and temporally varying vector potential in a tight-binding model and to set up the matrix of $\bar{Q}$ for numerical diagonalization.

A tight-binding Hamiltonian is in general written as a polynomial of creation and annihilation operators of Wannier states, denoted by $a_{i,s}^\dag$ and $a_{i,s}$, where $i$ labels sites, and $s$ labels internal degrees of freedom. Assume that the external electromagnetic fields represented by a vector potential $\bm A(\bm r, t)$ vary smoothly in space and time, the fields can be included in the tight-binding model through a Peierls phase
\begin{eqnarray}
a_{i,s}^\dag \rightarrow a_{i,s}^\dag \exp\left[- i\frac{e}{\hbar} \int_{\bm r_0}^{\bm r_i}  \bm A(\bm r, t) \cdot d\bm l \right],
\end{eqnarray}
which leads to a change of the hopping term
\begin{eqnarray}\label{eq:tintA}
t_{ij,ss'} a_{i,s}^\dag a_{j,s'} \rightarrow t_{ij,ss'}  \exp\left[- i\frac{e}{\hbar} \int_{\bm r_j}^{\bm r_i}  \bm A(\bm r, t) \cdot d\bm l \right] a_{i,s}^\dag a_{j,s'} \equiv \tilde{t}_{ij,ss'}  a_{i,s}^\dag a_{j,s'}.
\end{eqnarray}
The phase factor is path dependent. This is not a problem if the spatial variation of $\bm A$ is smooth on the lattice scale. In the limit of smooth variation we can approximate $\tilde{t}_{ij,ss'} $ by 
\begin{eqnarray}\label{eq:tAapprox}
\tilde{t}_{ij,ss'} \approx t_{ij,ss'}  \exp\left[- i\frac{e}{\hbar} \bm A(\bm r_{ij}, t) \cdot \bm d_{ij} \right] \equiv t_{ij,ss'} e^{-i \phi_{ij}(t)},
\end{eqnarray}
where $\bm r_{ij} \equiv (\bm r_i + \bm r_j)/2$, and $\bm d_{ij} \equiv \bm r_i - \bm r_j$. Eq.~\ref{eq:tAapprox} is the main result to be used in the next section.

\section{Tight-binding models}
In this section we give two tight-binding models of the Floquet quantum Hall effect, respectively for Schr\"{o}dinger and Dirac electrons.

\subsection{Schr\"{o}dinger electron}
We consider a nearest-neighbor single-orbital tight-binding model on a square lattice
\begin{eqnarray}
H_S = -t \sum_{\langle ij \rangle} c_{i}^\dag c_j + {\rm h.c.} 
\end{eqnarray}
where $\langle ij \rangle$ means sites $i, j$ are nearest neighbors. $t > 0$. We assume the lattice constant is $a$ and the lattice sites have coordinates
\begin{eqnarray}
\bm r_{i} = x_i a \hat{x} + y_i a \hat{y},\,\, x_i, y_i \in \mathbb{Z}.
\end{eqnarray} 
In the case that the system is infinite in both directions one can Fourier transform the Hamiltonian and obtain the eigenenergy $\epsilon_{\bm k} = -4t\cos(k_x a + k_y a)$, with $k_x, k_y \in [-\pi/a, \pi/a]$. For long wavelength $|\bm k|\ll 1/a$ we have $\epsilon_{\bm k} \approx 2ta^2 k^2 -4t$, same as that of a Schr\"{o}dinger electron with mass $m = \hbar^2/(4ta^2)$.

To get the Floquet QHE effect we consider a vector potential due to two linearly polarize light
\begin{eqnarray}
\bm E_1 = E \cos(\omega t) \hat{x},\,\, \bm E_2 = E\cos(Kx) \sin(\omega t) \hat{y},
\end{eqnarray}
which is 
\begin{eqnarray}
\bm A(t) = -\frac{E}{\omega} \sin(\omega t) \hat{x} + \frac{E}{\omega} \cos(Kx) \cos(\omega t) \hat{y}.
\end{eqnarray}
The hopping term ($t_{i,j}$) in this case is
\begin{equation}
	t_{i,j}=-\exp\left[  -\frac{ie}{\hbar}\bm A(\frac{\bm{r}
		_{i}+\bm{r}_{j}}{2}, t) \cdot \bm d_{ij}\right]  .
\end{equation}
With the help of vector potential, above equation can be written as
\begin{eqnarray}
t_{i,j} = \begin{cases}
-\exp\left[  \mp\frac
{ie}{\hbar}\left(  \bm{-}\frac{Ea}{\omega}\sin(\omega t)\right)
\right]  \equiv\exp\left[  \pm i\theta\right] ,& \text{if } \bm{r}_{j}-\bm{r}_{i}=\pm a \hat{x}\\
-\exp\left[  \mp\frac
{ie}{\hbar}\left(  \frac{Ea}{\omega}\cos(Kx_{i})\cos(\omega t)\right)
\right]  \equiv\exp\left[  \pm i\phi_{x}\right], & \text{if } \bm{r}_{j}-\bm{r}_{i}=\pm a \hat{y} 
\end{cases}
\end{eqnarray}
where we have
\begin{equation}
	\phi_{x}=-\frac{e}{\hbar}\left(  \frac{Ea}{\omega}\cos(Kx_{i})\cos(\omega
	t)\right)  ,\theta=\frac{e}{\hbar}\left(  \frac{Ea}{\omega}\sin(\omega
	t)\right)  ,\phi_{0}=\frac{eEa}{\hbar\omega}
\end{equation}
The Hamiltonian is written $(\bm r_{i} = x_i a \hat{x} + y_i a \hat{y})$ as
\begin{align}
	H_{S}^{F} &  =
	{\displaystyle\sum\limits_{x}}
	{\displaystyle\sum\limits_{y}}
	\left[  C_{x,y}^{\dagger}C_{x,y+a}\exp\left[  i\phi_{x}\right]  +C_{x,y}
	^{\dagger}C_{x,y-a}\exp\left[  -i\phi_{x}\right]  \right]  \\
	&  +
	{\displaystyle\sum\limits_{x}}
	{\displaystyle\sum\limits_{y}}
	\left[  C_{x,y}^{\dagger}C_{x+a,y}\exp\left[  +i\theta\right]  +C_{x,y}
	^{\dagger}C_{x-a,y}\exp\left[  -i\theta\right]  \right]  \nonumber
\end{align}
Using eigenstates of the form
\begin{equation}
	C_{x,y}^{\dagger}=
	{\displaystyle\sum\limits_{k}}
	e^{iky}C_{x,k}^{\dagger},
\end{equation}
For fixed $k$, we arrive at
\begin{equation}
		H_{S}^{F}(k)=
	{\displaystyle\sum\limits_{x}}
	\left[  2\cos[\phi_{x}-ka]C_{x,k}^{\dagger}C_{x,k}+C_{x,k}^{\dagger}
	C_{x+a,k}\exp\left[  +i\theta\right]  +C_{x,k}^{\dagger}C_{x-a,k}\exp\left[
	-i\theta\right]  \right]
\end{equation}
Now in terms of $x=ja$, above equation can be written as
%\begin{equation}
%	H_{i,j}(k);i,j\epsilon\lbrack1,2r_{c}+1],
%\end{equation}
%where we have
\begin{align}
	H_{j,j}(k) &  =-2\cos\left[  \frac{e}{\hbar}\frac{Ea}{\omega}\cos
	(Kaj)\cos(\omega t)+ka\right]  \label{12}\\
	H_{j,j+1}(k) &  =-\exp\left[  i\frac{e}{\hbar}\left(  \frac{Ea}{\omega}
	\sin(\omega t)\right)  \right]  \nonumber\\
	H_{j,j-1}(k) &  =-\exp\left[  -i\frac{e}{\hbar}\left(  \frac{Ea}{\omega}
	\sin(\omega t)\right)  \right]  \nonumber
\end{align}
Now we need to perform time Fourier transform of above equation as
\begin{align}
	H_{j,j,n} &  =\frac{1}{T}
	{\displaystyle\int\limits_{0}^{T}}
	H_{j,j}(k)e^{-in\omega t}dt\\
	&  =\frac{-1}{2\pi}
	{\displaystyle\int\limits_{0}^{2\pi}}
	2\cos\left[  \phi_{0}\cos(Kaj)\cos(\tau)+ka\right]  e^{-in\tau}d\tau
	\nonumber\\
\end{align}
we have used the property of the Bessel function
\begin{equation}
J_{n}(x)  =\frac{1}{2\pi}
{\displaystyle\int\limits_{0}^{2\pi}}
e^{ix\sin\tau-in\tau}d\tau\Longrightarrow\frac{1}{2\pi}
{\displaystyle\int\limits_{0}^{2\pi}}
e^{ix\cos\tau-in\tau}d\tau=J_{n}(x)e^{\frac{in\pi}{2}}
\end{equation}
and the fact that $\tau\rightarrow\tau+\pi/2;\sin\tau=\sin\tau
+\pi/2=\cos\tau$. Therefore, we arrive at
\begin{equation}
	H_{j,j,n}=-\left[  J_{n}\left(  \phi_{0}\cos(Kaj)\right)  e^{ika}+J_{n}\left(
	-\phi_{0}\cos(Kaj)\right)  e^{-ika}\right]  e^{\frac{in\pi}{2}}\label{14}
\end{equation}
similarly, we have
\begin{align}
	H_{j,j+1,n} &  =-\frac{1}{2\pi}
	{\displaystyle\int\limits_{0}^{2\pi}}
	e^{i\phi_{0}\sin\tau-in\tau}d\tau=-J_{n}(\phi_{0})\label{15}\\
	H_{j,j-1,n} &  =-J_{n}(-\phi_{0})\nonumber
\end{align}
%Therefore, using above three equations, we arrive at the final Hamiltonian
%\begin{align}
%	H_{S}^{F}(k,n)  & =H_{j,j,n}+H_{j,j+1,n}+H_{j,j-1,n}\label{16}\\
%	& =-\left[  J_{n}\left(  \phi_{0}\cos(Kaj)\right)  e^{ika}+J_{n}\left(
%	-\phi_{0}\cos(Kaj)\right)  e^{-ika}\right]  e^{\frac{in\pi}{2}}\nonumber\\
%	& -J_{n}(\phi_{0})-J_{n}(-\phi_{0})\nonumber
%\end{align}

We can now construct the matrix of $\bar{Q}$. To this end we choose a cutoff for $m$ ($m\hbar \omega$ in the diagonal blocks):
\begin{eqnarray}
|m|\le m_c,
\end{eqnarray}
where $m_c$ is a positive integer. This means that there are $N_m = 2 m_c + 1$ diagonal blocks, and each block is a $N_S\times N_S$ matrix. Therefore $\bar{Q}$ is a $N_m N_S \times N_m N_S$ matrix. Each $N_S \times N_S$ block, labeled by $\bar{Q}_{m,m+n}$, is
\begin{eqnarray}
\bar{Q}_{m,m+n} = H_{S}^{F}(k,n) - m\hbar \omega \delta_{n0} \mathbbm{1}_{N_S\times N_S},
\end{eqnarray} 
where the $N_S \times N_S$ matrix $H_n$ has matrix elements shown as

\begin{eqnarray}
H_{S}^{F}(k,n)  =\frac{1}{T}
{\displaystyle\int\limits_{0}^{T}}
H_{S}^{F}(k,t)e^{-in\omega t}dt
\end{eqnarray}

To make convergence with respect to $m_c$ faster one can choose $\hbar \omega \gg 8t$, where $8t$ is the band width of the tight-binding model. For $m_c = 4$, $r_c = 7$, the dimension of $\bar{Q}$ is $N_m N_S = 2025$. 
%Diagonalization of such a matrix should be doable on a personal computer. Plotting the eigenvalues $\epsilon$ versus $\phi_0$ should give something similar to the %Hofstadter butterfly.

\subsection{Dirac electron}
We consider a nearest-neighbor single-orbital tight-binding model
\begin{eqnarray}
	H_D = -t_{i\alpha,j\beta} \sum_{\langle i\alpha,j\beta \rangle} c_{i\alpha}^\dag c_{j\beta} + {\rm h.c.} 
\end{eqnarray}
where $\langle i\alpha,j\beta \rangle$ means sites $i, j$ are nearest neighbors with sublattices $\alpha$, $\beta$ and $t > 0$ being the hopping parameter. We assume the lattice constant is $a$ and the lattice sites have coordinates
\begin{eqnarray}
	\bm r_{i\alpha} = m_{i} \bm a_1 \hat{x} + n_{i} \bm a_2 \hat{y} + \bm \tau_{\alpha},\,\, m_{i}, n_{i} \in \mathbb{Z}.
\end{eqnarray} 
%In the case that the system is infinite in both directions one can Fourier transform the Hamiltonian and obtain the eigenenergy $\epsilon_{\bm k} = -4t\cos(k_x a + k_y a)$, with $k_x, k_y \in [-\pi/a, \pi/a]$. For long wavelength $|\bm k|\ll 1/a$ we have $\epsilon_{\bm k} \approx 2ta^2 k^2 -4t$, same as that of a Schr\"{o}dinger electron with mass $m = \hbar^2/(4ta^2)$.

To get the Floquet QHE effect we consider a vector potential due to two linearly polarize light
\begin{eqnarray}
	\bm E_1 = E \cos(\omega t) \hat{x},\,\, \bm E_{2}^{\prime} = E\sin(Kx) \sin(2\omega t) \hat{y},
\end{eqnarray}
which is 
\begin{eqnarray}
	\bm A^{\prime} (t) = -\frac{eE}{\omega} \sin(\omega t) \hat{x} + \frac{eE}{2\omega} \sin(Kx) \cos(2\omega t) \hat{y}.
\end{eqnarray}
Note that $\nabla \cdot \bm A = 0$. For simplicity we consider the long wavelength limit
\begin{eqnarray}
	\bm A^{\prime} (t) \approx -\frac{eE}{\omega} \sin(\omega t) \hat{x} + \frac{eE}{2\omega} \left(K x\right) \cos(2\omega t) \hat{y}.
\end{eqnarray}

To include $\bm A^{\prime}$ in the tight-binding model, we consider a finite system defined by 
\begin{eqnarray}
	\max(|x_{i\alpha}|,|y_{i\beta}|)\le r_c,
\end{eqnarray}
where $r_c$ is a positive integer. The Hamiltonian $H_D$ in the tight-binding basis is a $N_S \times N_S$ square matrix with $N_S = (2r_c+1)^2$ and its matrix elements
\begin{eqnarray}
	H_{i\alpha,j\beta} = -t_{i\alpha,j\beta}, \,\, {\rm if }\, |\bm r_{i\alpha} -\bm r_{j\beta}| = a
\end{eqnarray}
and $0$ otherwise. 

Including the vector potential using Eq.~\ref{eq:tAapprox} corresponding to replacing $H_{i\alpha,j\beta}$ by 
\begin{eqnarray}\label{eq:HijD}
	H_{i\alpha,j\beta} = - t \exp\left\{ -i\frac{eEa}{\hbar\omega}\left[-(x_{i\alpha} - x_{j\beta})\sin(\omega t) + \left(\frac{Ka(x_{i\alpha}+x_{j\beta})}{2}\right) (y_{i\alpha} - y_{j\beta}) \cos(2\omega t) \right] \right\},
\end{eqnarray}
if $|\bm r_{i\alpha} -\bm r_{j\beta}| = a$, and $H_{i\alpha,j\beta} = 0$ otherwise. For simplicity we use $a$ as the length unit and $t$ as the energy unit. $K$ is thus in units of $1/a$. Eq.~\ref{eq:HijD} is then simplified as 
\begin{eqnarray}\label{eq:HijDdimenless}
	H_{i\alpha,j\beta} = - \exp \left\{ -i \phi_0 \left[-(x_{i\alpha} - x_{j\beta})\sin(\omega t) + \left(\frac{K(x_i+x_j)}{2}\right ) (y_{i\alpha} - y_{j\beta}) \cos(2\omega t) \right] \right\},
\end{eqnarray}
where $\phi_0 \equiv eEa/\hbar \omega = (eEa/t)/(\hbar \omega/t)$ is dimensionless. Here we essentially use $t/ea$ as the units of $E$ and $t/\hbar$ as the units of $\omega$. If the long-wavelength limit is not taken at this stage we have instead of Eq.~\ref{eq:HijDdimenless}
\begin{eqnarray}
	H_{i\alpha,j\beta} = - \exp \left\{ -i \phi_0 \left[-(x_{i\alpha} - x_{j\beta})\sin(\omega t) + \frac{1}{2} \sin\left(K(x_{i\alpha}+x_{j\beta})\right ) (y_{i\alpha} - y_{j\beta}) \cos(2\omega t) \right] \right\}.
\end{eqnarray}
We will use this expression below, since one can always get the long-wavelength limit from it.

We next construct the quasienergy operator $\bar{Q}$. For this we first need to calculate $H_{i\alpha,j\beta,n}$:
\begin{eqnarray}\label{eq:HijDn}
	H_{i\alpha,j\beta,n} &=& \frac{1}{T} \int_0^T H_{i\alpha,j\beta} e^{-in\omega t} dt \\\nonumber
	&=& -\frac{1}{2\pi} \int_0^{2\pi} \exp\left[ iX_1\sin(\tau) +iX_2\cos(2\tau) - i n\tau \right] d\tau \\\nonumber
	&=& -J_n(X)e^{in \phi},
\end{eqnarray}
where we have used the property of the Bessel function
\begin{eqnarray}
	J_n(x) = \frac{1}{2\pi} \int_0^{2\pi} e^{ix\sin(\tau)-in\tau} d\tau.
\end{eqnarray}
In the result of $H_{i\alpha,j\beta,n}$ we have defined
\begin{eqnarray}
	X_1\equiv \phi_0(x_{i\alpha}-x_{j\beta}),\,\,\, X_2\equiv -\phi_0\left(\frac{K(x_i+x_j)}{2}\right ) (y_{i\alpha} - y_{j\beta}),
\end{eqnarray}
which gives
\begin{eqnarray}
	\cos(2x) = 2\cos(x)\cos(x) -1.
\end{eqnarray}
Using this result it is easy to calculate $e^{in\phi} = (\cos \phi + i \sin \phi)^n$ numerically.

We can now construct the matrix of $\bar{Q}$. To this end we choose a cutoff for $m$ ($m\hbar \omega$ in the diagonal blocks):
\begin{eqnarray}
	|m|\le m_c,
\end{eqnarray}
where $m_c$ is a positive integer. This means that there are $N_m = 2 m_c + 1$ diagonal blocks, and each block is a $N_S\times N_S$ matrix. Therefore $\bar{Q}$ is a $N_m N_S \times N_m N_S$ matrix. Each $N_S \times N_S$ block, labeled by $\bar{Q}_{m,m+n}$, is
\begin{eqnarray}
	\bar{Q}_{m,m+n} = H_n - m\hbar \omega \delta_{n0} \mathbbm{1}_{N_S\times N_S},
\end{eqnarray} 
where the $N_S \times N_S$ matrix $H_n$ has matrix elements shown in Eq.~\ref{eq:Hijn}
\begin{eqnarray}
	(H_n)_{i\alpha,j\beta} = \begin{cases}
		-J_n(X)e^{in \phi},& \text{if } |\bm r_{i\alpha} - \bm r_{j\beta}| = 1\\
		0, & \text{otherwise} 
	\end{cases}
\end{eqnarray}

To make convergence with respect to $m_c$ faster one can choose $\hbar \omega \gg 8t$, where $8t$ is the band width of the tight-binding model. For $m_c = 4$, $r_c = 7$, the dimension of $\bar{Q}$ is $N_m N_S = 2025$.
%Diagonalization of such a matrix should be doable on a personal computer. Plotting the eigenvalues $\epsilon$ versus $\phi_0$ should give something similar to the Hofstadter butterfly.

\begin{thebibliography}{99}

\bibitem{eckardt_2015} A. Eckardt and E. Anisimovas \textbf{17}, 093039 (2015).
\end{thebibliography}
\end{document}
